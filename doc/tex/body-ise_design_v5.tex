% =============================================================================

\begin{figure}[t]
\begin{math}
\begin{tikzpicture}
\matrix [matrix of math nodes,right delimiter={\rbrack},left delimiter={\lbrack}] (S) {
  \AESRND{s}{r}_{0,0} & \AESRND{s}{r}_{0,1} & \AESRND{s}{r}_{0,2} & \AESRND{s}{r}_{0,3} \\
  \AESRND{s}{r}_{1,0} & \AESRND{s}{r}_{1,1} & \AESRND{s}{r}_{1,2} & \AESRND{s}{r}_{1,3} \\
  \AESRND{s}{r}_{2,0} & \AESRND{s}{r}_{2,1} & \AESRND{s}{r}_{2,2} & \AESRND{s}{r}_{2,3} \\
  \AESRND{s}{r}_{3,0} & \AESRND{s}{r}_{3,1} & \AESRND{s}{r}_{3,2} & \AESRND{s}{r}_{3,3} \\
} ;

\matrix at ([xshift={+2.00cm}]  S.east) [matrix of math nodes,right delimiter={\rbrack},left delimiter={\lbrack},anchor={west}] (C0) {
  \AESRND{s}{r}_{0,0} \\ \AESRND{s}{r}_{1,0} \\ \AESRND{s}{r}_{0,1} \\ \AESRND{s}{r}_{1,1} \\
} ;
\matrix at ([xshift={+0.50cm}] C0.east) [matrix of math nodes,right delimiter={\rbrack},left delimiter={\lbrack},anchor={west}] (C1) {
  \AESRND{s}{r}_{0,2} \\ \AESRND{s}{r}_{1,2} \\ \AESRND{s}{r}_{0,3} \\ \AESRND{s}{r}_{1,3} \\
} ;
\matrix at ([xshift={+0.50cm}] C1.east) [matrix of math nodes,right delimiter={\rbrack},left delimiter={\lbrack},anchor={west}] (C2) {
  \AESRND{s}{r}_{2,0} \\ \AESRND{s}{r}_{3,0} \\ \AESRND{s}{r}_{2,1} \\ \AESRND{s}{r}_{3,1} \\
} ;
\matrix at ([xshift={+0.50cm}] C2.east) [matrix of math nodes,right delimiter={\rbrack},left delimiter={\lbrack},anchor={west}] (C3) {
  \AESRND{s}{r}_{2,2} \\ \AESRND{s}{r}_{3,2} \\ \AESRND{s}{r}_{2,3} \\ \AESRND{s}{r}_{3,3} \\
} ;

\node at ($(S.east)!0.5!(C0.west)$) {$\mapsto$} ; \node at ([xshift={-0.50cm}] S.west) [anchor={east}] {$\AESRND{s}{r} = $} ;

\node [inner sep={-2pt},fit=(S-1-1) (S-2-2),  fill={red},   fill opacity={0.2}] {} ;
\node [inner sep={-2pt},fit=(S-1-3) (S-2-4),  fill={green}, fill opacity={0.2}] {} ;
\node [inner sep={-2pt},fit=(S-3-1) (S-4-2),  fill={blue},  fill opacity={0.2}] {} ;
\node [inner sep={-2pt},fit=(S-3-3) (S-4-4),  fill={orange},fill opacity={0.2}] {} ;

\node [inner sep={-2pt},fit=(C0-1-1) (C0-4-1),fill={red},   fill opacity={0.2}] {} ;
\node [inner sep={-2pt},fit=(C1-1-1) (C1-4-1),fill={green}, fill opacity={0.2}] {} ;
\node [inner sep={-2pt},fit=(C2-1-1) (C2-4-1),fill={blue},  fill opacity={0.2}] {} ;
\node [inner sep={-2pt},fit=(C3-1-1) (C3-4-1),fill={orange},fill opacity={0.2}] {} ;
\end{tikzpicture}
\end{math}
\caption{An illustration of quadrant-packed representation, as applied to a state matrix.}
\label{fig:ise:v5:quadpack}
\end{figure}

% -----------------------------------------------------------------------------

\ISE{5}
assumes 
$\RVXLEN = 32$
and adopts a 
novel, {\em quadrant}-packed 
representation of state and round key matrices:
per
\REFFIG{fig:ise:v5:quadpack}
for example, doing so packs each $4$-element quadrant of the state into 
a $32$-bit word.  Note that {\em either} two rows {\em or} two columns 
of the state can be accessed by accessing two quadrants: the intuition,
based on this fact, is that such a representation can
1) afford advantages of {\em both} row- and column-packed alternatives,
   {\em and}
2) allow an instruction format that 
   includes $2$ source and $1$ destination register address.
However, it also implies a need to convert any input into (resp. output 
from) {\em quadrant}-packed representation; although such conversion is
amortised by $Nr$ rounds of computation, it represents an overhead vs.
other variants.

As detailed in
\REFFIG{fig:v5:mnemonics}
and
\REFFIG{fig:v5:pseudo},
\ISE{5}
adds
$ 7$
instructions ($3$ for encryption, $3$ for decryption, and $1$ auxiliary).
For example
\VERB{saes.v5.esrsub.lo}
applies
\AESFUNC{SubBytes} and \AESFUNC{ShiftRow}  
to the lower
row spanning    two packed quadrants,
\VERB{saes.v5.esrsub.hi}
applies
\AESFUNC{SubBytes} and \AESFUNC{ShiftRow}  
to the upper
row spanning    two packed quadrants,
and
\VERB{saes.v5.emix}
applies 
\AESFUNC{MixColumn} 
to a
column spanning two packed quadrants.

\REFFIG{fig:v5:round}
demonstrates that use of 
\ISE{5} 
to implement AES encryption requires
$16$ instructions per round:
$ 4$ \VERB{lw}
     instructions to load the round key,
$ 4$ \VERB{xor}
     instructions to apply \AESFUNC{AddRoundKey},
$ 4$ \VERB{saes.v5.esrsub.[lo|hi]}
     instructions to apply \AESFUNC{SubBytes} and \AESFUNC{ShiftRows},
     and
$ 4$ \VERB{saes.v5.emix}
     instructions to apply \AESFUNC{MixColumns}.
Note that conversion into (resp. from) quadrant-packed representation
requires a further
$12$ instructions;
     this can be reduced to
$ 4$ \VERB{pack[h]}
     instructions using the 
     standard 
     B~\cite[Section 17]{RV:ISA:I:19}
     extension.

% =============================================================================

