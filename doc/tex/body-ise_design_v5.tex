% =============================================================================

\REFSEC{sec:pseudo:v5} shows the mnemonics and pseudo-code functions
for variant \ISE{5}.
These instructions use a {\em tiled} approach to representing the
AES state.
Figure ({\bf TODO}) shows how the traditional column-wise representation
of AES is changed such that each {\em quadrant} of the 16-byte state
is kept in a single $32$-bit register.

\[
\begin{tikzpicture}
\matrix [matrix of math nodes,right delimiter={\rbrack},left delimiter={\lbrack}] (S) {
  \AESRND{s}{r}_{0,0} & \AESRND{s}{r}_{0,1} & \AESRND{s}{r}_{0,2} & \AESRND{s}{r}_{0,3} \\
  \AESRND{s}{r}_{1,0} & \AESRND{s}{r}_{1,1} & \AESRND{s}{r}_{1,2} & \AESRND{s}{r}_{1,3} \\
  \AESRND{s}{r}_{2,0} & \AESRND{s}{r}_{2,1} & \AESRND{s}{r}_{2,2} & \AESRND{s}{r}_{2,3} \\
  \AESRND{s}{r}_{3,0} & \AESRND{s}{r}_{3,1} & \AESRND{s}{r}_{3,2} & \AESRND{s}{r}_{3,3} \\
} ;

\node [inner sep={-2pt},fit=(S-1-1) (S-2-2),fill={red},   fill opacity={0.2}] {} ;
\node [inner sep={-2pt},fit=(S-1-3) (S-2-4),fill={green}, fill opacity={0.2}] {} ;
\node [inner sep={-2pt},fit=(S-3-1) (S-4-2),fill={blue},  fill opacity={0.2}] {} ;
\node [inner sep={-2pt},fit=(S-3-3) (S-4-4),fill={orange},fill opacity={0.2}] {} ;

\node at ([xshift={-0.25cm}] S.west) [anchor={east}] {$\AESRND{s}{r} = $} ;
\end{tikzpicture}
\]

We can now compute the next round state of any quadrant by sourcing
only two other quadrants (registers) at a time, thus keeping within
the $2$-read-$1$-write constraint.

The state matrix and must be re-arranged before and after applying
the round functions, which adds a small overhead to this variant.
Similarly, the KeySchedule words must also be re-arranged to allow
\AESFUNC{AddRoundKey} to be performed efficiently.
This can be done as a post-processing step in the key expansion.

A single encryption round for this variant requires
four load-word instructions to fetch the round key,
four {\tt xor} instructions to perform \AESFUNC{AddRoundKey},
four {\tt saes.v5.ersub.[lo|hi]} instructions to compute
    \AESFUNC{SubBytes}, \AESFUNC{ShiftRows} for each quadrant
and
four {\tt saes.v5.emix} instructions to compute \AESFUNC{MixColumns} for each
quadrant.
This would make it equivalent to variant 2, however we must also
account for the effort spent packing and un-packing the AES
state into the quadrant representation.
For the base ISA, this would take $12$ instructions to (un-)pack the state.
We note that if the {\tt pack[h]} instructions from the draft
Bit-manipulation extension were included, then packing and unpacking
would be reduced to four instructions.
All packing and un-packing occurs outside the performance critical
loop sections.
\REFFIG{fig:round:v5} shows an example AES encrypt round function
using this variant.

% =============================================================================

