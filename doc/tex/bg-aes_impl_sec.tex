% =============================================================================

% TODO

Although the security of AES against a cryptanalytic attack is defined by
the design, and so out of scope, the concept of implementation attacks is
of central importance.
In essence, an implementation attack focuses on
the concrete, in practice implementation
rather than
the abstract, on paper     specification:
fundamentally, this suggests countermeasures against such attacks must be
considered alongside the implementation they relate to.

AES is an important target, 

 active (e.g., fault injection)
or
passive (i.e., side-channel),
attack techniques.

with the latter sub-divided into those dependent on
analogue,
e.g., power-based~\cite{},
or
discrete, 
e.g.,  time-based~\cite{KoeQui:99},
leakage.

From a positive perspective, use of ISEs
{\em can} provide some inherent protection against certain attacks.
For example,
ISEs typically yield constant (i.e., data-oblivious) execution latency,
so prevent some classes of time- or micro-architectural
(see, e.g.,~\cite[Section 4]{Szefer:19} and ~\cite[Section 4]{GYCH:18})
(e.g., cache-based) attack techniques.
From a negative perspective, however,
use of ISEs presents some unique challenges.
For example, 
Saab et al.\cite{SaaRohHam:16}
discuss power-based attacks on ISEs, AES-NI specifically; they conclude
that naive use of AES-NI will yield exploitable leakage.  Mitigation of
such leakage, in general, demands that the ISE to be flexible enough to
compose with appropriate
(e.g., hiding~\cite[Chapter 7]{dpabook} or masking~\cite[Chapter 10]{dpabook})
countermeasures.

% TODO: Tillich et al.~\cite{TilHerMan:07}

% =============================================================================
