% =============================================================================

While the security of AES against a cryptanalytic attack is defined by
the design, and so is out of scope, {\em implementation} attacks are
of central importance.
An implementation attack focuses on the concrete instance of a construct
rather than the abstract specification.
Countermeasures against such attacks must therefore be
considered alongside implementations they relate to.
Since AES is an important target, a significant body of literature exists
around implementation attacks on it, including both
 active (e.g., fault injection)
or
passive (i.e., side-channel monitoring)
attack techniques.
The latter can be sub-divided into those dependent on
analogue
(power-based~\cite{ManOswPop:07})
or
discrete 
(time-based~\cite{KoeQui:99})
leakage.

Use of ISEs
{\em can} provide some inherent protection against certain attacks.
For example,
ISEs typically yield constant time execution,
preventing some classes of timing or micro-architectural
attack techniques
(see~\cite[Section 4]{Szefer:19} and~\cite[Section 4]{GYCH:18}).
Unfortunately,
use of ISEs also presents some unique challenges.
For example, 
Saab et al. ~\cite{SaaRohHam:16}
discuss power-based attacks on AES-NI; concluding
that naive use of AES-NI yields exploitable information leakage.
Mitigation of such leakage demands the ISE
address instances where the leakage stems from ``inside'' the ISE,
and work with appropriate countermeasures
(e.g., hiding~\cite[Chapter 7]{ManOswPop:07} or masking~\cite[Chapter 10]{ManOswPop:07}).
Tillich et al.~\cite{TilHerMan:07}
consider this problem to an extent, including an ISE-based option in
their investigation of hardened AES implementations. However, the challenge
of developing suitable ISEs is under-studied in general.

% =============================================================================
