
\subsection{Evaluation Discussion}
\label{sec:eval:results}

\begin{table}[]
\centering
\begin{tabular}{l|cc|cc|cc|cc}
& \multicolumn{2}{c}{\begin{tabular}[c]{@{}c@{}}AES 128 Block\\ Encrypt\end{tabular}}
& \multicolumn{2}{c}{\begin{tabular}[c]{@{}c@{}}AES 128 Block\\ Decrypt\end{tabular}}
& \multicolumn{2}{c}{\begin{tabular}[c]{@{}c@{}}KeySchedule\\ Encrypt\end{tabular}} 
& \multicolumn{2}{c}{\begin{tabular}[c]{@{}c@{}}KeySchedule\\ Decrypt\end{tabular}}
\\
Variant     &  iret & cycles & iret & cycles & iret & cycles & iret & cycles\\
\hline
V1 (Latency)&       &        &      &        &      &        &      &      \\
V2 (Latency)&       &        &      &        &      &        &      &      \\
V5 (Latency)&       &        &      &        &      &        &      &      \\
V1 (Size)   &       &        &      &        &      &        &      &      \\
V2 (Size)   &       &        &      &        &      &        &      &      \\
V3          &       &        &      &        &      &        &      &      \\
V4          &       &        &      &        &      &        &      &      \\
V5 (Size)   &       &        &      &        &      &        &      &      \\
\end{tabular}
\caption{Comparison of performance improvement unit-area for each
variant. Obtained by dividing the speedup in terms of instructions
executed (relative to the baseline T-table implementation) by the
normalised size of the dedicated ISE logic, as shown in
\REFTAB{tab:eval:hw}.
We deliberately omit the size of the host core from our calculation,
as this dominates the total size of the system and detracts from
the comparison.
}
\end{table}

{\bf TODO:}
Note performance / unit area.
V3/ttable is the best candidate. Only one SBox, very fast per gate etc.
