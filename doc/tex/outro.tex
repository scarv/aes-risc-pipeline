% =============================================================================

Although differing in nature, both AES and RISC-V represent important
standards.  In this paper, we have addressed the challenge of secure, 
efficient implementation of AES on RISC-V: our approach harnesses the
modularity afforded by RISC-V, through a focus on the use of ISEs.

Specifically, and motivated by ongoing efforts to standardise support 
for AES in RISC-V, we have implemented and evaluated five ISE designs 
on two different RISC-V compliant base micro-architectures.
Our conclusion is that
1) \ISE{3}
   is the best option for 
   AES on $32$-bit cores,
2) \ISE{4}
   is the best option for 
   AES on $64$-bit cores,
   and
3) the
   standard 
   B~\cite[Section 17]{RV:ISA:I:19}
   extension
   can combine with either option to support AES-GCM.

% TODO: maybe try to add some insight into ISA design constraints ...
%
%The requirement for $3$-address (i.e., $2$ source and $1$ destination)
%instruction format prevented some further optimisations, 
%e.g., the integration of \AESFUNC{AddRoundKey} in \ISE{4}.

% =============================================================================
