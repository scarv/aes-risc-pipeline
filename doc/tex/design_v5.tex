% =============================================================================

\REFSEC{sec:pseudo:v5} shows the mnemonics and pseudo-code functions
for variant \ISE{5}.
These instructions use a {\em tiled} approach to representing the
AES state.
Figure ({\bf TODO}) shows how the traditional column-wise representation
of AES is changed such that each {\em quadrant} of the 16-byte state
is kept in a single $32$-bit register.

We can now compute the next round state of any quadrant by sourcing
only two other quadrants (registers) at a time, thus keeping within
the $2$-read-$1$-write constraint.

The state matrix and must be re-arranged before and after applying
the round functions, which adds a small overhead to this variant.
Similarly, the KeySchedule words must also be re-arranged to allow
AddRoundKey to be performed efficiently.
This can be done as a post-processing step in the key expansion.

A single encryption round for this variant requires
$4$ load-word instructions to fetch the round key,
$4$ {\tt xor} instructions to perform AddRoundKey,
$4$ {\tt saes.v5.ersub.[lo|hi]} instructions to compute
    SubBytes, ShiftRows for each quadrant
and
$4$ {\tt saes.v5.emix} instructions to compute MixColumns for each
quadrant.
This would make it equivalent to variant 2, however we must also
account for the effort spent packing and un-packing the AES
state into the quadrant representation.
For the base ISA, this would take $12$ instructions to pack and
$12$ instructions to unpack the state.
We note that if the {\tt pack[h]} instructions from the draft
Bit-manipulation extension were included, then packing and unpacking
would be reduced to $4$ instructions.
\REFFIG{fig:round:v5} shows an example AES encrypt round function
using this variant.

% =============================================================================

