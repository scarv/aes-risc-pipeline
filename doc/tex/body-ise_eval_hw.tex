% =============================================================================

Each ISE variant was evaluated on the host cores
described in \REFSEC{sec:ise:imp}.
The 32-bit designs 
(\ISE{1},\ISE{2},\ISE{3},\ISE{5}) were implemented on both the
\CORE{1} and \CORE{2} cores.
The 64-bit design (\ISE{4}) was only evaluated on the 64-bit configuration
of the \CORE{1} core.
Table \ref{tab:eval:hw} shows the hardware implementation costs.

For \ISE{1}, \ISE{2} and \ISE{5}, two implementations are evaluated.
The {\em Size} optimised implementations instantiate only a single
Forward/Inverse S-box circuit and take multiple cycles
to produce a result.
The {\em Latency} optimised implementations instantiate $4$ S-box
\AESFUNC{MixColumn} circuits to produce their results in a single processor 
clock cycle.

The {\em Size} columns of Table \ref{tab:eval:hw} 
record the number of NAND2 equivalent gates of each variant,
instantiated independently from any wider system.
The LTP column gives the Longest Topological Path of the synthesised
functional unit circuit from input to combinatorial output.
The \CORE{2} Size column gives the size in NAND2 equivalent gates of the
\CORE{2}, with the AES functional units integrated.
The ``Baseline'' row gives the size of the core without any of the
ISEs integrated.
We found that none of the proposed ISEs affected the critical
path of the \CORE{2} or \CORE{1} cores.


% =============================================================================
