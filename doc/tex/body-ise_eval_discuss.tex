% =============================================================================

\begin{table}[p]
\centering
\begin{tabular}{lrrrr}
Variant     &     ISE Size &     ISE LTP & \CORE{2} Size   & \CORE{1} Size \\
\hline
Baseline    &     -        &     -       & 37375 ($1.00$x) &               \\
V1 (Latency)&     3472     &{\bf 19}     & 41723 ($1.12$x) &               \\
V2 (Latency)&     3547     &{\bf 19}     & 41199 ($1.10$x) &               \\
V5 (Latency)&     4121     &     22      & 42070 ($1.13$x) &               \\
V3          &{\bf 1157}    &     30      & 38610 ($1.03$x) &               \\
V4          &     8312     &     27      & N/A             & {\bf TODO}    \\
V1 (Size)   &     2174     &     22      & 40161 ($1.07$x) &               \\
V2 (Size)   &     1381     &     21      & 38885 ($1.04$x) &               \\
V5 (Size)   &     1927     &     23      & 39251 ($1.05$x) &               \\
\end{tabular}
\caption{
Hardware implementation costs of each ISE.
The ISE columns list the size and depth of the ISE Verilog hardware
descriptions when instantiated in isolation.
The \CORE{2} and \CORE{1} columns give the size (in NAND2 equivalent gates)
of the ISEs when integrated into each core.
Gate counts and topological path lengths are obtained using the
Yosys\cite{yosys} tool suite.
}
\label{tab:eval:hw}
\end{table}

\begin{table}[p]
\centering
\begin{tabular}{l|r|r|r|r|r}
Variant &
{\begin{tabular}[c]{@{}c@{}}AES 128 Block\\ Encrypt\end{tabular}} &
{\begin{tabular}[c]{@{}c@{}}AES 128 Block\\ Decrypt\end{tabular}} &
{\begin{tabular}[c]{@{}c@{}}KeySchedule\\ Encrypt\end{tabular}}   &
{\begin{tabular}[c]{@{}c@{}}KeySchedule\\ Decrypt\end{tabular}}   &
.data   \\ \hline
%Byte   &           &          &     312     &  -         & 522   \\
T-table &     804   &     804  &     154     & $174\star$ & 5120  \\
\hline
V1      &     424   &     424  &{\bf 68     }&  -         & 10    \\
V2      &{\bf 234}  &{\bf 238} &{\bf 68     }& $62\star$  & 10    \\
V3      &     290   &     290  &     86      & $64\star$  & 10    \\
V4      &     268   &     268  &     168     & $100\star$ &  0    \\
V5      &     266   &     278  &    $82+208 $&  -         & 10    \\
\end{tabular}
\caption{
Static code size metrics for each variant, measured in bytes.
}
\label{tab:eval:sw:size}
\end{table}

\begin{table}[p]
\centering
\begin{tabular}{l|rr|rr|rr|rr}
& \multicolumn{2}{c|}{\begin{tabular}[c]{@{}c@{}}AES 128 Block\\ Encrypt\end{tabular}}
& \multicolumn{2}{c|}{\begin{tabular}[c]{@{}c@{}}AES 128 Block\\ Decrypt\end{tabular}}
& \multicolumn{2}{c|}{\begin{tabular}[c]{@{}c@{}}KeySchedule\\ Encrypt\end{tabular}} 
& \multicolumn{2}{c }{\begin{tabular}[c]{@{}c@{}}KeySchedule\\ Decrypt\end{tabular}}
\\
Variant           &     iret &     cycles &     iret &     cycles &     iret &     cycles &     iret &    cycles \\ \hline
T-table           &     998  &    1076    &     998  &    1103    &     466  &     554    &    1747  &     2346  \\ \hline
\ISE{1} (Latency) &     518  &     593    &     518  &     607    &{\bf 198} &{\bf 291}   &{\bf 204} &{\bf  310} \\
\ISE{2} (Latency) &{\bf 221} &     301    &{\bf 222} &     303    &{\bf 198} &     302    &     335  &      616  \\
\ISE{5} (Latency) &     233  &     304    &     233  &     309    &     332  &     447    &     338  &      466  \\
\ISE{3}           &     238  &{\bf 291}   &     238  &{\bf 286}   &     219  &     312    &     659  &     1118  \\
\ISE{1} (Size)    &     518  &     753    &     518  &     775    &{\bf 198} &     331    &{\bf 204} &      350  \\
\ISE{2} (Size)    &     221  &     538    &     222  &     540    &{\bf 198} &     332    &     335  &      754  \\
\ISE{5} (Size)    &     233  &     556    &     233  &     550    &     332  &     477    &     338  &      496
\end{tabular}                
\caption{                    
Performance results for the \CORE{2} core.
Note the $64$-bit \ISE{4} is absent.
}
\label{tab:eval:sw:perf:scarv}
\end{table}

\begin{table}[p]
\centering
\begin{tabular}{l|rr|rr|rr|rr}
& \multicolumn{2}{c|}{\begin{tabular}[c]{@{}c@{}}AES 128 Block\\ Encrypt\end{tabular}}
& \multicolumn{2}{c|}{\begin{tabular}[c]{@{}c@{}}AES 128 Block\\ Decrypt\end{tabular}}
& \multicolumn{2}{c|}{\begin{tabular}[c]{@{}c@{}}KeySchedule\\ Encrypt\end{tabular}} 
& \multicolumn{2}{c }{\begin{tabular}[c]{@{}c@{}}KeySchedule\\ Decrypt\end{tabular}}
\\
Variant     &      iret &     cycles &     iret &      cycles &    iret &     cycles &     iret &     cycles\\
\hline
%Byte       &           &            &          &             &         &            &          &          \\
T-table     &     948   &      1143  &     949  &       1025  &     444  &     478    &     1726 &    1977  \\
\hline
V1 (Latency)&     528   &      685   &     529  &       680   &{\bf 212} &     341    &{\bf 214} &{\bf 290} \\
V2 (Latency)&{\bf 231}  &{\bf  359}  &{\bf 233} &       368   &{\bf 212} &{\bf 315}   &     350  &     508  \\
V5 (Latency)&     243   &      414   &     244  &{\bf   319}  &     346  &     427    &     348  &     424  \\
V3          &     253   &      445   &     254  &       445   &     233  &     470    &     674  &    2425  \\
V1 (Size)   &     528   &      804   &     529  &       744   &     212  &     357    &     214  &     335  \\
V2 (Size)   &{\bf 231}  &      511   &{\bf 233} &       520   &     212  &     345    &     350  &     646  \\
V5 (Size)   &     243   &      585   &     244  &       543   &     346  &     504    &     348  &     454  \\
\hline
V4          &     81    &      119   &      82  &       125   &      66  &     204    &     136  &     306  \\
\end{tabular}
\caption{
Performance results for the \CORE{1} core.
Note that \ISE{4} uses the 64-bit base core, all others use the 32-bit base.
}
\label{tab:eval:sw:perf:rocket}
\end{table}

% -----------------------------------------------------------------------------

\begin{table}[p]
\centering
\begin{tabular}{l|rr|rr|rr|rr}
& \multicolumn{2}{c}{\begin{tabular}[c]{@{}c@{}}AES 128 Block\\ Encrypt\end{tabular}}
& \multicolumn{2}{c}{\begin{tabular}[c]{@{}c@{}}AES 128 Block\\ Decrypt\end{tabular}}
& \multicolumn{2}{c}{\begin{tabular}[c]{@{}c@{}}KeySchedule\\ Encrypt\end{tabular}} 
& \multicolumn{2}{c}{\begin{tabular}[c]{@{}c@{}}KeySchedule\\ Decrypt\end{tabular}}
\\
Variant           &     iret    &     cycle  &      iret   &    cycles  &       iret  &     cycles  &      iret   &      cycles   \\
\hline                                                                                                                                            
\ISE{1} (Latency) &       4.61  &      4.34  &       4.61  &      4.35  &       5.39  &       6.06  &      20.50  &      18.12    \\
\ISE{2} (Latency) &      10.58  &      8.38  &      10.53  &      8.53  &       5.28  &       4.30  &      12.22  &       8.92    \\
\ISE{5} (Latency) &       8.64  &      7.14  &       8.64  &      7.20  &       2.71  &       2.50  &      10.43  &      10.15    \\
\ISE{3}           & {\bf 30.12} &{\bf 26.56} & {\bf 30.12} &{\bf 27.71} & {\bf 14.63} & {\bf 12.76} &      19.04  &      15.08    \\
\ISE{4}           &      12.32  &      9.04  &      12.17  &      8.82  &       6.76  &       2.72  &      12.85  &       7.67    \\
\ISE{1} (Size)    &       7.37  &      5.46  &       7.37  &      5.44  &       8.61  &       6.40  & {\bf 32.74} & {\bf 25.63}   \\
\ISE{2} (Size)    &      27.18  &     12.04  &      27.06  &     12.29  &      13.56  &      10.04  &      31.39  &      18.73    \\
\ISE{5} (Size)    &      18.48  &      8.35  &      17.64  &      8.30  &       5.79  &       5.01  &      22.29  &      20.40
\end{tabular}
\caption{Comparison of performance improvement unit-area for each
variant. 
}
\label{tab:eval:results}
\end{table}

% -----------------------------------------------------------------------------

For the $32$-bit ISEs 
(\ISE{1},\ISE{2},\ISE{3},\ISE{5})
our primary metric is
performance/area efficiency, since these ISEs must be sensible to
implement in small, embedded class cores.
To this, end we calculate an efficiency score by calculating the speedup
of each variant relative to the T-table baseline, and dividing it
by the normalised size of each variant
(see ``ISE Size'' column of \REFTAB{tab:eval:hw}).
This is done separately for the $32$-bit and $64$-bit variants so as
to compare like with like.

\REFTAB{tab:eval:results} shows the performance / area efficiency results.
The \CORE{2} core results were used for these numbers, but the
same conclusions may be drawn from the \CORE{1} results.
We deliberately omit the size of the host core from our calculation,
as this dominates the total size of the system and detracts from
the comparison.
Qualitatively, we place more weight on the block encrypt/decrypt results
than the KeySchedule results, as one typically performs many more
block operations than KeySchedules.

\ISE{3} gives the best performance per unit area
for Encryption, Decryption and encrypt KeySchedule generation.
Its decryption KeySchedule performance suffers slightly for needing
to use the equivilent inverse cipher transform.
Based on these results, we believe ISE \ISE{3} makes the best candidate for
standardisationfor $32$-bit RISC-V cores,
based on its efficiency, raw performance and simplicity to implement.

For the $64$-bit case, \ISE{4} is an obvious choice for making
sensible use of the wider data-path within the constraints of RISC-V.
We note the relativley poor performance of \ISE{4} in terms of cycles
per instruction executed for the Rocket core - a $50\%$ overhead for
Encrypt/Decrypt.
This is attributed to the forwarding behaviour of the \CORE{1}:
ROCC instruction results are not
forwarded to subsequent instructions as quickly as base ISA
instructions.

% =============================================================================
