% =============================================================================

By reproducing~\cite[Section 4.2]{TilGro:06},
\ISE{1}
assumes 
$w = 32$
and adopts a 
column-packed 
representation of state and round key matrices.
As detailed in
\REFFIG{fig:mnemonics:v1}
and
\REFFIG{fig:pseudo:v1},
it adds
$ 4$
instructions ($2$ for encryption, $2$ for decryption).
For example
\VERB{saes.v1.encs}
applies \AESFUNC{SubBytes}  to elements in a packed column,
and
\VERB{saes.v1.encm}
applies \AESFUNC{MixColumn} to             a packed column (which is optionally rotated).
Both 
\VERB{saes.v1.encs}
and
\VERB{saes.v1.encm}
use an instruction format with $1$ source and $1$ destination register address.

As detailed in
\REFFIG{fig:round:v1},
using this variant to implement AES encryption requires
$47$
instructions per round:
$ 4$ \VERB{saes.v1.encs} instructions to apply \AESFUNC{SubBytes},
$ 4$ \VERB{saes.v1.encm} instructions to apply \AESFUNC{MixColumns},
$ 4$ \VERB{ lw}          instructions to load the round key,
$31$                     instructions to apply \AESFUNC{ShiftRows},
and
$ 4$ \VERB{xor}          instructions to apply \AESFUNC{AddRoundKey}.

% =============================================================================
