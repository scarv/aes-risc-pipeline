% =============================================================================

\REFSEC{sec:pseudo:v1} shows the mnemonics and pseudo-code functions
for \ISE{1}.
This variant is the simplest conceptual approach, and
is also described in~\cite[Section 4.2]{TilGro:06}.
The AES state is stored column-wise in four $32$-bit words.
Each instruction applies the forward/inverse AES \AESFUNC{SubBytes} or
\AESFUNC{MixColumn} function to a single word.
The instructions have only a single source and destination register,
making them very compact in terms of encoding points.
\REFFIG{fig:design:fu_block:v1} shows a block diagram of these
instructions.

The authors of \cite{TilGro:06} note that these instructions do not
efficiently support the \AESFUNC{ShiftRows} operation of AES.  We reproduce their
optimisation addressing this in the \ISE{2} instructions.

A single encryption round using this variant requires
four {\tt saes.v1.encs} instructions to perform \AESFUNC{SubBytes},
four {\tt saes.v1.encm} instructions to perform \AESFUNC{MixColumns},
four load-word instructions to fetch the round key
and
four {\tt xor} instructions to perform \AESFUNC{AddRoundKey}.
As noted, the inefficiency of these instructions is the
lack of \AESFUNC{ShiftRows} support, which takes $31$ instructions using the
base {\tt rv32i} ISA.
\REFFIG{fig:round:v1} shows an example AES encrypt round function
using this variant.

% =============================================================================
