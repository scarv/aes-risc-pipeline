\documentclass[submission]{iacrtrans}

\usepackage{paper}

\title{AES on RISC-V}
\keywords{...}

\ifbool{anonymous}{%
\author{}
\institute{}
}{%
\author{}
\institute{}
}%

\begin{document}

% =============================================================================

\maketitle

\begin{abstract}
AES is one of the most widely used block ciphers in the world.
RISC-V is a popular open Instruction Set Architecture (ISA) used
for new CPU designs and research in both industry and academia.
Secure and efficient execution of AES is an essential property for any ISA,
particularly in resource constrained environments such as embedded and IoT
class CPUs.
We survey existing approaches to accelerating AES using Instruction Set
Extensions (ISEs) from academia and industry, create RISC-V targeted variants,
and evaluate them for software performance, code size and hardware
implementation cost.
Our work informs on-going efforts to standardise the proposed Cryptographic
ISEs for RISC-V.
\end{abstract}

% =============================================================================

\section{Introduction}
\label{sec:intro}
% =============================================================================

\paragraph{Implementing the Advanced Encryption Standard (AES).}

In $2001$, NIST pronounced Rijndael~\cite{DaeRij:98,DaeRij:02}, 
a design due to Daemen and Rijmen, 
as winner of a $5$-year standardisation process~\cite{NBBBDFR:01} instigated 
to identify a replacement for the incumbent
Data     Encryption Standard (DES)~\cite{FIPS:46} 
block cipher; the resulting 
Advanced Encryption Standard (AES)~\cite{FIPS:197} 
was announced in $2001$.

Compared to more general cases, cryptographic kernels such as AES present a
significant challenge wrt. implementation, because they often
a) involve computationally intensive, somewhat niche functionality,
b) are deployed in a diverse range of contexts,
   and
c) form a central target in what is a complex attack surface.
The demand for efficiency, however measured, forms an overarching example
with at least two dimensions.
First,
cryptography often represents an enabling technology vs. a feature: it will
likely be viewed as overhead, therefore, when viewed from the perspective 
of a user.  Put another way,  (even {\em perceived}) {\em in}efficiency may
be problematic in terms of fitness for purpose.  Addressing this is further 
complicated by any constraints associated with the context, e.g., a demand 
for
high-volume, 
 low-latency, 
high-throughput, 
 low-footprint, 
and/or 
 low-power.
Second,
and although efficiency is a goal in and of itself therefore, it {\em also} 
acts as an enabler for security.  This is because one cannot 
(or at least {\em should} not) 
compromise security to meet efficiency requirements, implying that delivery 
of higher efficiency offers greater margin within which to deliver suitable
countermeasures against attack.

AES represents an interesting case-study wrt. techniques which attempt to 
address the challenge of efficient, secure implementation.  For example,
per the request for candidates announcement\footnote{%
\url{https://www.govinfo.gov/content/pkg/FR-1997-09-12/pdf/97-24214.pdf}
}, the AES process was instrumental in popularising a model in which
{\em both}
``security''
(e.g., resilience against cryptanalytic attack)
{\em and}
``algorithm and implementation characteristics''
(e.g., computational efficiency, memory requirements, suitability for different platform classes)
form important quality metrics for the {\em design}, in order to facilitate
techniques for higher quality {\em implementations} of it.
In addition,
both the design {\em and} implementations of it are long-lived.
From a positive perspective, 
the central importance of AES has yielded special emphasis on related
research and development effort before, during, and, most significantly, 
after the AES process.
From a negative perspective, however,
the $20+$ year period since standardisation has forced an evolution of 
implementation technique, e.g., to match evolution of both the technology 
and attack landscape.  For example,~\cite[Section 3.6]{NBBBDFR:01} covers
implementation (e.g., side-channel) attacks: this field has become richer,
and the associated threat more potent during said period.

% -----------------------------------------------------------------------------

\paragraph{Support via Instruction Set Extensions (ISEs).}

A potentially large space of implementation techniques will often exist
for a given cryptographic kernel.  One could classify instances as being 
   kernel-agnostic
   or
   kernel-specific,
and based on the use of   
   hardware              only,
                software only,
   or
   hardware and software (i.e., a hybrid of the two).
The latter class includes instances based on the concept of an
Instruction Set Extension (ISE)~\cite{GalBer:11,BarGioMar:09,RegIen:16}.
The idea is to identify, e.g., through benchmarking, a set of additional 
instructions that allow the kernel to leverage
special-purpose, domain-specific functionality
in the resulting ISE,
vs. 
general-purpose                  functionality
in the base Instruction Set Architecture (ISA),
and thereby deliver improvement wrt. pertinent quality metrics.  
Although ISEs can be an effective option for {\em both}
high(er)-end, performance-oriented
and
 low(er)-end, constrained
platforms, 
they are particularly effective for the latter by 
improving footprint and latency
vs. a software-only option
while also
improving area      and flexibility
vs. a hardware-only option.

Abstractly, an ISE design constitutes
1) an {\em interface},
   i.e., 
   addition of instructions to some 
   base ISA
   plus
2) an {\em implementation},
   i.e., 
   support for execution of said instructions through changes or addition 
   to a 
   base micro-architecture.
As a fundamental, long-lived computer systems interface, the design of or
changes to an ISA demands careful consideration
(cf.~\cite[Section 4]{Gueron:09}); this implies that the production of a 
concrete ISE design is far from trivial.  
Such a design must deliver a demonstrable improvement wrt. the kernel in 
question, {\em as well as} considering a (non-exhaustive) list of design 
goals such as

\begin{itemize}
\item minimal invasiveness,
      e.g.,
      limit the number and scope (local vs. global) of changes,
\item minimal overhead,
      e.g.,
      avoid instructions that demand support via large or complex hardware components,
\item maximal compatibility,
      e.g.,
      align with constraints in the base ISA, such as use of existing instruction formats,
      and
\item maximal       utility,
      i.e.,
      favour general- vs. {\em too} special-purpose functionality.
\end{itemize}

\noindent
By extending x86, Intel provide archetypal examples of and evidence in
support of ISEs.
For example, various generations of
non-cryptographic
Multi-Media      eXtensions (MMX),
Streaming SIMD  Extensions (SSE),
and
Advanced Vector Extensions (AVX)
support numerical kernels via vector (or SIMD) vs. scalar computation.  
Likewise, the
    cryptographic
Advanced Encryption Standard New Instructions (AES-NI)~\cite{Gueron:09,DruGueKra:19}
ISE
supports AES: it significantly improves latency and throughput
(see, e.g.,~\cite{FazLopOli:18}),
and represents a pertinent case-study wrt. the guidelines above by
a) adding just $6$ additional (vs. $1500+$ total) instructions,
b) reducing overhead by, e.g., sharing the XMM register file,
c) facilitating compatibility via the
   \VERB{CPUID}~\cite[Chapter 20]{X86:1:18}
   feature identification mechanism,
   and
d) offering utility beyond AES specifically:
   it supports AES-specific functionality but is able to maximise utility 
   through more AES-agnostic support of {\em other} kernels
   (see, e.g., use of AES-NI within SHA3~\cite{BBGR:09} and CEASER~\cite[Section 4.1]{AnkAnk:16} candidates).

% -----------------------------------------------------------------------------

\paragraph{Remit.}

In this paper, we address the challenge of supporting AES by designing and
evaluating an associated ISE for RISC-V
(see, e.g.,~\cite{riscv:1,riscv:2}).
Note that we specifically focus on extending the scalar RISC-V instruction 
set, thereby contrasting with work related to cryptographic support in the
standard V (or vector) extension~\cite[Section 21]{RV:ISA:I:19}.

On one hand, 
RISC-V represents an excellent vehicle for such work:
extensibility is a by-design feature in the ISA, whose open nature renders
exploration of such extensions easier by virtue of the range of associated 
(often open-source) implementations.  
Increased commercial deployment of such implementations suggests that work 
on RISC-V timely, and potentially of high impact.
On the other hand, however,
RISC-V also presents some unique challenges vs. previous work.
For example,
RISC-V could in fact be viewed as {\em three} related base ISAs,
 RV32I~\cite[Section 2]{RV:ISA:I:19},
 RV64I~\cite[Section 5]{RV:ISA:I:19},
and
RV128I~\cite[Section 6]{RV:ISA:I:19},
that each support a different word size:
designing ISEs that are applicable (or scale) across these options is a
complicating factor, yet an important design goal.

% TODO: need to be clearer what constraints are for a given design

% -----------------------------------------------------------------------------

\paragraph{Organisation.}

% TODO

\REFSEC{sec:bg} describes the AES cipher, existing implementation strategies
and prior commercial and academic ISE designs.
\REFSEC{sec:design} details five candidate ISEs, including the criteria
used to select and design them.
\REFSEC{sec:eval} gives evaluation results in terms of hardware cost
and software performance.
\REFSEC{sec:psca} shows how our ISE recommendation for highly resource
constrained devices can be further extended to give power side-channel
resistance.
Finally, 
\REFSEC{sec:outro}
presents some conclusions and potential directions for future work.

\ifbool{submission}{%
Note that in order to satisfy the TCHES submission guidelines, we have 
anonymised various resources and references.  We intend to open-source 
such resources post-submission, but could provide them to reviewers, 
if required, to facilitate the review process.
}{}%

% =============================================================================


% =============================================================================

\section{Background}
\label{sec:bg}

% -----------------------------------------------------------------------------

\subsection{AES  specification}
\label{sec:bg:aes_spec}
% =============================================================================

% -----------------------------------------------------------------------------

\paragraph{Syntax.}

As a block cipher, AES defines two algorithms
\[
\begin{array}{lcl}
\ALG{Enc} &:& \SET{ 0, 1 }^{8 \cdot 4 \cdot Nk} \times \SET{ 0, 1 }^{8 \cdot 4 \cdot Nb} \rightarrow \SET{ 0, 1 }^{8 \cdot 4 \cdot Nb} \\
\ALG{Dec} &:& \SET{ 0, 1 }^{8 \cdot 4 \cdot Nk} \times \SET{ 0, 1 }^{8 \cdot 4 \cdot Nb} \rightarrow \SET{ 0, 1 }^{8 \cdot 4 \cdot Nb} \\
\end{array}
\]
such that
$
m = \ALG{Dec}( k, c = \ALG{Enc}( k, m ) ) .
$
That is, given a plaintext $m$ and cipher key $k$, \ALG{Enc} encrypts $m$ 
under $k$; given the same $k$, \ALG{Dec} will invert \ALG{Enc} and so the
{\em same} $m$ can be recovered from the associated ciphertext $c$.  
In addition, it defines an algorithm
\ALG{KeyExp}
that expands~\cite[Section 5.2]{FIPS:197} the cipher key into a sequence 
of round keys then used by
\ALG{Enc}
or
\ALG{Dec};
where appropriate, we use
\[
\begin{array}{lcl}
\ALG{Enc-KeyExp} &:& \SET{ 0, 1 }^{8 \cdot 4 \cdot Nk} \rightarrow \SET{ 0, 1 }^{( 8 \cdot 4 \cdot Nb ) \times ( Nr + 1 )} \\
\ALG{Dec-KeyExp} &:& \SET{ 0, 1 }^{8 \cdot 4 \cdot Nk} \rightarrow \SET{ 0, 1 }^{( 8 \cdot 4 \cdot Nb ) \times ( Nr + 1 )} \\
\end{array}
\]
to denote said algorithm as specialised to suit
\ALG{Enc}
and
\ALG{Dec}
respectively.

% -----------------------------------------------------------------------------

\paragraph{Parameterisation.}

An AES parameter set~\cite[Figure 4]{FIPS:197}
is a triple
$
\TUPLE{ Nk, Nb, Nr }
$
where 
$Nk$ dictates the number of $32$-bit words in $k$,
$Nb$ dictates the number of $32$-bit words in $m$ or $c$ (i.e., a block),
and
$Nr$ dictates the number of rounds.  The standard AES parameter sets are
\[
\begin{array}{lcl}
\mbox{AES-128} &\mapsto& \TUPLE{ 4, 4, 10 } \\
\mbox{AES-192} &\mapsto& \TUPLE{ 6, 4, 12 } \\
\mbox{AES-256} &\mapsto& \TUPLE{ 8, 4, 14 } \\
\end{array}
\]
such that the number of bits in a plaintext (resp. ciphertext) block is fixed to 
$
8 \cdot 4 \cdot Nb = 128 .
$
From here on, we focus wlog. on encryption using AES-128 (other parameter 
sets are catered for naturally, and decryption with minor differences) so
use the terms AES and AES-128 synonymously.

% -----------------------------------------------------------------------------

\paragraph{Design.}

The mathematics underpinning AES are described in ~\cite[Section 4]{FIPS:197}.
In particular, it can be defined in terms of 
operations in the finite field $\F_{2^{  8}}$ constructed as
$
\F_{2}[\IND{x}] / ( \IND{x}^{8} + \IND{x}^{4} + \IND{x}^{3} + \IND{x} + 1 ) .
$
A hexadecimal short-hand~\cite[Section 3.2]{FIPS:197} is used to represent 
field literals, e.g.,
$
\AESCONST{13} ~\mapsto~ \RADIX{13}{16} ~\equiv~ \RADIX{00010011}{2} ~\mapsto~ \IND{x}^4 + \IND{x} + 1 .
$
Field 
      addition, 
multiplication, 
and  
      division
are denoted by
$\AESADD$,
$\AESMUL$,
and
$\AESINV$
respectively,
with the multiplication-by-$\IND{x}$ operation~\cite[Section 4.2.1]{FIPS:197} 
denoted \AESFUNC{xtime}.
Elements of $\F_{2^8}$ are collected into $( 4 \times 4 )$-element state
and round key matrices; the $i$-th row and $j$-th column of such a matrix 
relating to round $r$ is denoted
$\AESRND {s}{r}_{i,j}$
and
$\AESRND{rk}{r}_{i,j}$
respectively, with super- and/or subscripts omitted whenever irrelevant.

AES is an iterative block cipher, based on a substitution-permutation network.
This means encryption using AES can be described~\cite[Section 5.2]{FIPS:197}
as follows:
1)    the  input  plaintext is pre-whitened to yield
      $\AESRND {s}{  0} = m \AESADD \AESRND{rk}{0} = m \AESADD k$,
2)    each $r$-th round, for $1 \leq r \leq Nr$, demands computation of
      $\AESRND {s}{r+1} = \ALG{P-layer}( \ALG{S-layer}( \AESRND{s}{r}                        ) ) \AESADD \AESRND{rk}{r}$,
      and therefore use of round key
      $\AESRND{rk}{r  }$,
3)    the output ciphertext is
      $c = \AESRND{s}{Nr}$.
Note that an alternative round definition, namely
      $\AESRND {s}{r+1} = \ALG{P-layer}( \ALG{S-layer}( \AESRND{s}{r} \AESADD \AESRND{rk}{r} ) )                       $ ,
is plausible: this shifts the 
 pre-whitening step {\em before} 2) 
into an analogous 
post-whitening step {\em  after} 2)
to yield an equivalent result.
At a  low(er) level,
the computation of each round is specified via four round functions (each of 
which has an inverse, to support decryption):

\begin{itemize}

\item \AESFUNC{SubBytes}
      ~\cite[Section 5.1.1]{FIPS:197}
      operates element-wise,
      computing
      $\AESRND{s}{r+1}_{i,j} = \ALG{S-box}( \AESRND{s}{r}_{i,j} )$
      via application of the S-box:
      given an element $x$, this component can be described as
      \[
      \begin{array}{lcl}
      \ALG{S-Box} &:& \left\{\begin{array}{ccc}
                             \F_{2^8} &\rightarrow& \F_{2^8} \\
                             x        &\mapsto    & f(g(x))  \\
                             \end{array}
                      \right.
      \end{array}
      \]
      where 
      $g$ is an inversion, 
      and 
      $f$ is a specially selected affine transformation.
      Where appropriate,
      we overload \AESFUNC{SubBytes} by allowing it to denote application 
      of the S-box to {\em any} collection, 
      e.g., a row, column, or, more generally, a sequence, 
      of elements.

\item \AESFUNC{ShiftRows}
      ~\cite[Section 5.1.2]{FIPS:197}
      operates     row-wise,
      rotating each 
      $i$-th row 
      of 
      $\AESRND{s}{r  }$
      by $i$ elements
      to form 
      the associated row    of
      $\AESRND{s}{r+1}$,
      i.e.,
      $\AESRND{s}{r+1}_{i,j} = \AESRND{s}{r}_{i,j + i \pmod{Nb}}$.
      Where appropriate,
      we use
      \AESFUNC{ShiftRow}
      to denote
      the operation applied to a single 
      row
      within \AESFUNC{ShiftRows}.

\item \AESFUNC{MixColumns}
      ~\cite[Section 5.1.3]{FIPS:197}
      operates  column-wise,
      multiplying each 
      $j$-th column
      of 
      $\AESRND{s}{r  }$
      with a constant MDS matrix
      to form 
      the associated column of
      $\AESRND{s}{r+1}$.
      Where appropriate,
      we use
      \AESFUNC{MixColumn}
      to denote
      the operation applied to a single 
      column 
      within \AESFUNC{MixColumns}, i.e., multiplication of a $4$-element 
      column vector by the constant MDS matrix.
      
\item \AESFUNC{AddRoundKey}
      ~\cite[Section 5.1.4]{FIPS:197}
      operates element-wise,
      computing
      $\AESRND{s}{r+1}_{i,j} = \AESRND{s}{r}_{i,j} \AESADD \AESRND{rk}{r}_{i,j}$ 
      and thereby mixing a round key into the state.

\end{itemize}

\noindent
Note that
$
\ALG{S-layer} = \AESFUNC{SubBytes} ,
$
and
\[
\ALG{P-layer} = \left\{\begin{array}{l@{\;}c@{\;}l lr}
                       \AESFUNC{MixColumns} &\circ& \AESFUNC{ShiftRows} & \mbox{in rounds} & 1 \leq r < Nr \\
                                            &     & \AESFUNC{ShiftRows} & \mbox{in round } &            Nr \\
                       \end{array}
                \right.
\]
i.e., the last, $Nr$-th round differs from the initial $Nr - 1$ rounds.  As
such, a round as defined above is constructed via
$
\AESFUNC{AddRoundKey} \circ \AESFUNC{MixColumns} \circ \AESFUNC{ShiftRows} \circ \AESFUNC{SubBytes} 
$
or
$
\AESFUNC{AddRoundKey} \circ                            \AESFUNC{ShiftRows} \circ \AESFUNC{SubBytes}
$
respectively, where, because \AESFUNC{ShiftRows} and \AESFUNC{SubBytes}
commute, the order they are applied in can be selected to suit.



% -----------------------------------------------------------------------------

\subsection{AES implementation}
\label{sec:bg:aes_impl}

\subsubsection{Representation}
\label{sec:bg:aes_impl_rep}
% =============================================================================

A field element in $\F_{2^8}$ can be represented by an
$8$-bit byte,
where the $i$-th bit of $x$ for $0 \leq i < 8$ represents the $i$-th 
polynomial coefficient.

Beyond this, the state and round key matrices can be represented in
several ways.
The most direct option would be termed
array-based (or unpacked):
the matrix is represented as a $16$-element array of $8$-bit bytes, each
representing field elements.
%FIPS-197~\cite{FIPS:197} defines a word to be st. $w = 32$.
We use $R$ to refer to the register width of a target platform.
For RISC-V, $R = \RVXLEN$ where we consider $\RVXLEN \in {32,64}$.
Where $R \geq  32$,
an entire row or column of the AES state matrix can be packed into each 
register:
we term these
   ``row-packed''  
and
``column-packed''
representations respectively.
Where $R \geq 128$, 
it is plausible to pack
an entire AES state matrix
into a single register: 
we term this a 
 ``fully-packed'' 
representation.

% =============================================================================


\subsubsection{Hardware-only implementations}
\label{sec:bg:aes_impl_hw}
% =============================================================================

In a hardware-only implementation,
execution of 
AES
is 
performed by 
a dedicated hardware module (e.g., a memory-mapped co-processor).
A large design space exists for hardware implementations of AES.
Gaj and Chodowiec~\cite[Section 3.3]{GajCho:00}
give an overview, detailing
iterative,
combinatorial (unrolled),
and
pipelined architectures.
Similarly, ~\cite{PMDW:04,GooBen:05,GajCho:09}
survey concrete implementations on a variety of fabrics including FPGAs
and ASICs.

Although hardware-only designs are not our focus, the associated techniques
can guide ISE-related design choices.
First,
they guide the ISE interface.
For example, some ISEs can be characterised as offering an interface to
hardware constituting one round 
(i.e., aligned with an iterative hardware implementation).
Second,
they guide the ISE implementation.
For example, a significant body of work focuses on efficient hardware 
implementation of the S-box: ~\cite{Canright:05,BoyPer:12,ReyTahAsh:18}.

% =============================================================================

\subsubsection{Software-only implementations}
\label{sec:bg:aes_impl_sw}
% =============================================================================

% TODO

Note that we consider ``vanilla'' ISAs only, so exclude work related to the
use of, e.g., vector-like extensions~\cite{Hamburg:09}.

In combination,
Bernstein and   Schwabe~\cite{BerSch:08}
and
Schwabe   and Stoffelen~\cite{SchSto:16}
present and compare a range of software-based implementation and 
optimisation techniques, across a range of platforms.

% -----------------------------------------------------------------------------

\paragraph{Compute-oriented.}

A compute-oriented implementation of AES favours
 online     computation, 
thus reducing 
memory footprint
at the cost of increased 
latency.
Following~\cite[Section 4.1]{DaeRij:02}, for example, the idea is to simply
1) adopt an
    array-packed
   representation of state and round key matrices,
   then
2) construct a round implementation by following the algorithmic description
   of each round function in a direct manner.
Addition in $\F_{2^8}$ can be realised using a native XOR instruction; this
native support is seldom afforded to multiplication and inversion, however.
As a result, it is common to pre-compute the \ALG{S-box} and \AESFUNC{xtime} 
functions:
doing so demands pre-computation and storage of a
$
\SI{256}{\byte}
$
look-up table per function, but significantly reduces execution latency.

On platforms where $w = 32$,
Bertoni et al.~\cite{BBFMM:02}
further improve execution latency by exploiting the wider data-path.  Their
idea is to
1) adopt a 
      row-packed
   representation of state and round key matrices,
2) implement
   \AESFUNC{ShiftRows}
   by using native rotation instructions to act on the packed
   rows,
3) implement
   \AESFUNC{MixColumns}
   by harnessing the SIMD Within A Register (SWAR) paradigm:
   by applying
   \AESFUNC{xtime}
   across a packed row in parallel,
   a carefully organised scheme for evaluating
   \AESFUNC{MixColumns}
   can be constructed.

% -----------------------------------------------------------------------------

\paragraph  {Table-oriented.}

A  table-oriented implementation of AES favours
offline pre-computation,
thus reducing 
latency
at the cost of increased 
memory footprint.
The archetypal example of this technique is use of so-called
T-tables~\cite[Section 4.2]{DaeRij:02}.
In short, doing so means
1) adopting a 
   column-packed
   representation of state and round key matrices,
2) pre-computing
   $
   \AESFUNC{MixColumn} \circ \AESFUNC{SubBytes}
   $
   using the tables
   \[
   \begin{array}{cc}
   \begin{array}{lcl}
   T_0[x] &=& \left[\begin{array}{c}
                    \RADIX{02}{16} \AESMUL \ALG{S-box}( x ) \\
                    \RADIX{01}{16} \AESMUL \ALG{S-box}( x ) \\
                    \RADIX{01}{16} \AESMUL \ALG{S-box}( x ) \\
                    \RADIX{03}{16} \AESMUL \ALG{S-box}( x ) \\
                    \end{array} \right]
   \end{array}
   &
   \begin{array}{lcl}
   T_1[x] &=& \left[\begin{array}{c}
                    \RADIX{03}{16} \AESMUL \ALG{S-box}( x ) \\
                    \RADIX{02}{16} \AESMUL \ALG{S-box}( x ) \\
                    \RADIX{01}{16} \AESMUL \ALG{S-box}( x ) \\
                    \RADIX{01}{16} \AESMUL \ALG{S-box}( x ) \\
                    \end{array} \right]
   \end{array}
   \\\\
   \begin{array}{lcl}
   T_2[x] &=& \left[\begin{array}{c}
                    \RADIX{01}{16} \AESMUL \ALG{S-box}( x ) \\
                    \RADIX{03}{16} \AESMUL \ALG{S-box}( x ) \\
                    \RADIX{02}{16} \AESMUL \ALG{S-box}( x ) \\
                    \RADIX{01}{16} \AESMUL \ALG{S-box}( x ) \\
                    \end{array} \right]                 
   \end{array}
   &
   \begin{array}{lcl}
   T_3[x] &=& \left[\begin{array}{c}
                    \RADIX{01}{16} \AESMUL \ALG{S-box}( x ) \\
                    \RADIX{01}{16} \AESMUL \ALG{S-box}( x ) \\
                    \RADIX{03}{16} \AESMUL \ALG{S-box}( x ) \\
                    \RADIX{02}{16} \AESMUL \ALG{S-box}( x ) \\
                    \end{array} \right]
   \end{array}
   \end{array}
   \]
   for $x \in \F_{2^8}$,
3) computing each $j$-th column of $\AESRND{s}{r+1}$ as
   \[
   T_0[ \AESRND{s}{r}_{i, j + i \pmod{Nb}} ] \AESADD
   T_1[ \AESRND{s}{r}_{i, j + i \pmod{Nb}} ] \AESADD
   T_2[ \AESRND{s}{r}_{i, j + i \pmod{Nb}} ] \AESADD
   T_3[ \AESRND{s}{r}_{i, j + i \pmod{Nb}} ]
   \]
   where extraction of elements caters for \AESFUNC{ShiftRows}, then XOR'ing 
   the $j$-th column of $\AESRND{rk}{r}$ to cater for \AESFUNC{AddRoundKey}.

As such, each round amounts to a sequence of look-ups into $T_i$, plus XORs 
to combine their result; 
doing so demands pre-computation and storage of a
$
256 \cdot \SI{4}{\byte} = \SI{1}{\kilo\byte}
$
look-up table per $T_i$.
However, note that the overhead related to extraction of each element from 
packed columns representing $\AESRND{s}{r}$ 
(to form look-table offsets) 
is not insignificant:
Fiskiran and Lee~\cite{FisLee:01}
analyse the impact of different addressing modes on this issue, with
Stoffelen~\cite[Section 3.1]{Stoffelen:19}
concluding that RISC-V is (relatively) ill-equipped to reduce said overhead,
due to the provision of a (relatively) sparse set of addressing modes.

% -----------------------------------------------------------------------------

\paragraph{Use of bit-slicing.}

% TODO

\cite{MatNak:07,Konighofer:08,KasSch:09}

\cite{Stoffelen:19}

% =============================================================================

\subsubsection{Hybrid        implementations}
\label{sec:bg:aes_impl_ise}
% =============================================================================

Here, we survey AES-related ISE designs split into
1) industry-specified ISEs,
   which are {\em     standard} extensions,
   and
2) academia-specified ISEs,
   which are {\em non-standard} extensions,
wrt. a given base ISA.
   Each ISE is classified as either
   workload-specific,
   if it is only useful for AES,
   or
   workload-agnostic,
   if it is      useful for AES and other workloads.
Note that we exclude work where an ISE for another workload can be applied 
{\em  to} AES
but was not designed 
{\em for} AES
(see, e.g., Tillich and Gro{\ss}sch\"{a}dl~\cite{TilGro:04} who apply an ISE intended for ECC to AES).

% =============================================================================

\subsubsection{Standard, industry-specified ISEs}

\noindent
{\bf Intel}
      introduced support for AES in 
      x86
      per~\cite[Section 12.13]{X86:1:18}.
      Instructions use a
          destructive $2$-address ($1$ source, $1$ source/destination)  
      or
      non-destructive $3$-address ($2$ source, $1$        destination)
      format
      depending on the variant (e.g., XMM- vs. AVX-based),
      and operate on data housed in the pre-existing
      vector 
      register file, implying $R = 128$.
      AES is implemented by
      1) adopting a 
          fully-packed
         representation of state and round key matrices,
         then
      2) using
             \VERB{AESENC}         ~\cite[Page 3-54]{X86:2:18}
         to construct a round implementation as
         \[
         \VERB{AESENC} \mapsto \AESFUNC{AddRoundKey} \circ \AESFUNC{MixColumns} \circ \AESFUNC{SubBytes} \circ \AESFUNC{ShiftRows}
         \]
%     Note that
%            \VERB{AESENCLAST}     ~\cite[Page 3-56]{X86:2:18}
%     supports 
%     the $Nr$-th round;
%     additional instructions are provided to 
%     support
%     decryption
%     (i.e., \VERB{AESDEC}         ~\cite[Page 3-50]{X86:2:18}
%            and
%            \VERB{AESDECLAST}     ~\cite[Page 3-52]{X86:2:18})
%     and
%     key expansion
%     (i.e., \VERB{AESKEYGENASSIST}~\cite[Page 3-59]{X86:2:18}
%            and
%            \VERB{AESIMC}         ~\cite[Page 3-58]{X86:2:18}).

\noindent
{\bf IBM}
      introduced support for AES in 
      POWER
      per~\cite[Section 6.11.1]{POWER:18}.
      Instructions use a
      non-destructive $3$-register ($2$ source, $1$        destination)
      format,
      and operate on data housed in the pre-existing
      vector 
      register file, implying $R = 128$.
      AES is implemented by
      1) adopting a 
          fully-packed
         representation of state and round key matrices,
         then
      2) using
             \VERB{vcipher}     ~\cite[Page 304]{POWER:18}
         to construct a round implementation as
         \[
         \VERB{vcipher} \mapsto \AESFUNC{AddRoundKey} \circ \AESFUNC{MixColumns} \circ \AESFUNC{ShiftRows} \circ \AESFUNC{SubBytes}
         \]
%     Note that
%            \VERB{vcipherlast} ~\cite[Page 304]{POWER:18}
%     supports 
%     the $Nr$-th round;
%     additional instructions are provided to 
%     support
%     decryption
%     (i.e., \VERB{vncipher}    ~\cite[Page 305]{POWER:18}
%            and
%            \VERB{vncipherlast}~\cite[Page 305]{POWER:18})
%     and
%     key expansion
%     (i.e., \VERB{vsbox}       ~\cite[Page 305]{POWER:18}).

\noindent
{\bf ARM}
      introduced support for AES in 
      ARMv8-A
      per~\cite[Section A2.3]{ARMv8-A:20}.
      Instructions use a
          destructive $2$-address ($1$ source, $1$ source/destination)  
      format,
      and operate on data housed in the pre-existing
      vector 
      register file, implying $R = 128$.
      AES is implemented by
      1) adopting a 
          fully-packed
         representation of state and round key matrices,
         then
      2) using
             \VERB{AESE}  ~\cite[Section C7.2.8 ]{ARMv8-A:20}
             and
             \VERB{AESMC} ~\cite[Section C7.2.10]{ARMv8-A:20}
         to construct a round implementation as
         \[
         \VERB{AESMC} \circ \VERB{AESE} \mapsto \AESFUNC{MixColumns} \circ ( \AESFUNC{SubBytes} \circ \AESFUNC{ShiftRows} \circ \AESFUNC{AddRoundKey} ) ,
         \]
%         where the alternative round definition from 
%         \REFSEC{sec:bg:aes_spec} 
%         is assumed to cater for the order of application.
%     Note that
%     additional instructions are provided to 
%     support
%     decryption
%     (i.e., \VERB{AESD}  ~\cite[Section C7.2.7 ]{ARMv8-A:20}
%            and
%            \VERB{AESIMC}~\cite[Section C7.2.9 ]{ARMv8-A:20}),
%     but none are required to 
%     support
%     the $Nr$-th round:
%     \VERB{AESE} obviously lacks \AESFUNC{MixColumns}, and the post-whitening 
%     step is naturally supported via XOR. 

\noindent
{\bf Oracle}
      introduced support for AES in 
      SPARC 
      per~\cite[Sections 7.3+7.4]{SPARC:16}.
      Instructions use a
      non-destructive $4$-address ($3$ source, $1$        destination)
      format,
      and operate on data housed in the pre-existing
      general-purpose
      register file, implying $R =  64$.
      AES is implemented by
      1) using a 
         column-packed
         representation of state and round key matrices,
         then
      2) using
             \VERB{AES_EROUND01}     ~\cite[Page 109]{SPARC:16}
             and
             \VERB{AES_EROUND23}     ~\cite[Page 109]{SPARC:16}
         to construct a round implementation as
         \[
         ( \VERB{AES_EROUND01};\VERB{AES_EROUND23} ) \mapsto \AESFUNC{AddRoundKey} \circ \AESFUNC{MixColumns} \circ \AESFUNC{ShiftRows} \circ \AESFUNC{SubBytes} 
         \]
         in two steps:
         the first  step processes columns $0$ and $1$ via \VERB{AES_EROUND01}
         whereas
         the second step processes columns $2$ and $3$ via \VERB{AES_EROUND23}.
%     Note that
%            \VERB{AES_EROUND01_LAST}~\cite[Page 109]{SPARC:16}
%            and
%            \VERB{AES_EROUND23_LAST}~\cite[Page 109]{SPARC:16}
%     support 
%     the $Nr$-th round;
%     additional instructions are provided to 
%     support
%     decryption
%     (i.e., \VERB{AES_DROUND01}     ~\cite[Page 109]{SPARC:16},
%            \VERB{AES_DROUND23}     ~\cite[Page 109]{SPARC:16},
%            \VERB{AES_DROUND01_LAST}~\cite[Page 109]{SPARC:16},
%            and
%            \VERB{AES_DROUND23_LAST}~\cite[Page 109]{SPARC:16})
%     and
%     key expansion
%     (i.e., \VERB{AES_KEXPAND0}     ~\cite[Page 112]{SPARC:16},
%            \VERB{AES_KEXPAND1}     ~\cite[Page 109]{SPARC:16},
%            and
%            \VERB{AES_KEXPAND2}     ~\cite[Page 112]{SPARC:16}).

% -----------------------------------------------------------------------------

\subsubsection{Non-standard, academia-specified ISEs}

% workload-agnostic

      Burke et al.~\cite{BurMcDAus:00}
      propose 
      a workload-agnostic ISE
      based on workload characterisation for the
      DEC Alpha architecture \cite{alpha2014alpha}.
      Per~\cite{BurMcDAus:00}, pertinent examples
      for AES
      include
      a) \VERB{ROL}
         and
         \VERB{ROR},
         which perform
         left- and right-rotate,
         and
      b) \VERB{SBOX},
         which 
         extracts elements to form look-up table offsets.
         In one configuration,
         the resulting memory accesses are supported by a
         set of special-purpose ``S-box caches''.

      Fiskiran and Lee~\cite{FisLee:05}
      propose 
      a workload-agnostic ISE
      that employs a so-called
      Parallel Table Lookup Module (PTLU) for a ``{\em RISC like}''
      instruction set.
      For AES, 
      this accelerates implementations based on T-tables 
      by affording an addressing mode that
      a) integrates 
         extraction of elements to form look-up table offsets,
         and
      b) performs the associated table look-ups in parallel, supported by
         a dedicated scratch-pad memory.

      Biham et al.~\cite[Page 232]{BihAndKnu:98}
      propose (in theory)
      and
      Grabher et al.~\cite{GraGroPag:08}
      explore  (in practice)
      a workload-agnostic ISE
      that supports bit-sliced implementations for their custom
      CRISP (``{\em RISC like}'') architecture.
      The ISE allows computation using 
      {\em configurable} $4$-input, $2$-output 
      Boolean functions, vs. 
      {\em fixed}        $2$-input, $1$-output alternatives such as NOT, AND, OR, and XOR.
      Sequences of native Boolean instructions, which dominate bit-sliced
      implementations, can thereby be ``compressed'' into use of the ISE.
      Doing so improves both latency and footprint.
      \cite[Section 4]{GraGroPag:08} details the application to AES.

% workload-specific

      Nadehara et al.~\cite{NadIkeKur:04} 
      propose 
      a workload-specific ISE
       that could be described as 
      ``hardware-assisted T-tables'':
      observing that $\forall x, i \neq j$, $T_i[ x ]$ is a rotation of
      $T_j[ x ]$, they support on-the-fly computation (vs. via look-up)
      of T-table entries.
      The ISE constitutes a single instruction
      $\VERB{AESENC} \mapsto T_i$,
      supported by a dedicated hardware module
      (see~\cite[Figure 6]{NadIkeKur:04}).
      Instances of \VERB{AESENC}
      1) extract an   input element from a 
         packed  input column
      2) use the input to compute an output element equivalent to a
         look-up from the T-table,
         and
      3) store   the output element into a
         packed output column.
      This approach was reapplied by Saarinen~\cite{Saarinen:20}
      within the context of RISC-V.

      Tillich et al.~\cite{TilGroSze:05}
      propose 
      a workload-specific ISE
       that could be described as 
      ``hardware-assisted S-box'' for the SPARC V8 architecture.
      The ISE constitutes a single instruction
      $\VERB{sbox} \mapsto \AESFUNC{SubBytes}$,
      supported by a dedicated hardware module
      (see~\cite[Figure 1]{TilGroSze:05}).
      Instances of \VERB{sbox}
      1) extract an   input element from a packed  input row or column,
      2) use the input to compute an output element equivalent to a
         look-up from the S-box,
         and
      3)  insert the output element into a packed output row or column.
         Using insert vs. overwrite semantics allows
         \AESFUNC{ShiftRows} to be computed {\em for free}.

      Bertoni et al.~\cite{BBFR:06}
      propose 
      a workload-specific ISE
       that could be described as 
      ``hardware-assisted round functions''.
      The ISE includes
      1) zero-overhead rotation (similar to ARM),
         and
      2) byte- and word-oriented variants of
         $\VERB{SMix} \mapsto \AESFUNC{MixColumn} \circ \AESFUNC{SubBytes}$.
      
      Tillich and Gro{\ss}sch\"{a}dl~\cite{TilGro:06}
      propose 
      a workload-specific ISE
       that could be described as 
      ``hardware-assisted round functions'' for the SPARC V8 architecture.
      The ISE includes
         byte- and word-oriented variants of
         $\VERB  {sbox[4][s|r]} \mapsto \AESFUNC{SubBytes} $
         and
         $\VERB{mixcol[4][s]  } \mapsto \AESFUNC{MixColumn}$;
      per~\cite[Section 4.3]{TilGro:06},
      the most efficient variant allows
         a zero-overhead implementation of \AESFUNC{ShiftRows} to be realised.


% =============================================================================


\subsubsection{Security}
\label{sec:bg:aes_impl_sec}
% =============================================================================

While the security of AES against a cryptanalytic attack is defined by
the design, and so is out of scope, {\em implementation} attacks are
of central importance.
An implementation attack focuses on the concrete instance of a construct
rather than the abstract specification.
Countermeasures against such attacks must therefore be
considered alongside implementations they relate to.
Since AES is an important target, a significant body of literature exists
around implementation attacks on it, including both
 active (e.g., fault injection)
or
passive (i.e., side-channel monitoring)
attack techniques.
The latter can be sub-divided into those dependent on
analogue
(power-based~\cite{ManOswPop:07})
or
discrete 
(time-based~\cite{KoeQui:99})
leakage.

Use of ISEs
{\em can} provide some inherent protection against certain attacks.
For example,
ISEs typically yield constant time execution,
preventing some classes of timing or micro-architectural
attack techniques
(see~\cite[Section 4]{Szefer:19} and~\cite[Section 4]{GYCH:18}).
Unfortunately,
use of ISEs also presents some unique challenges.
For example, 
Saab et al. ~\cite{SaaRohHam:16}
discuss power-based attacks on AES-NI; concluding
that naive use of AES-NI yields exploitable information leakage.
Mitigation of such leakage demands the ISE
address instances where the leakage stems from ``inside'' the ISE,
and work with appropriate countermeasures
(e.g., hiding~\cite[Chapter 7]{ManOswPop:07} or masking~\cite[Chapter 10]{ManOswPop:07}).
Tillich et al.~\cite{TilHerMan:07}
consider this problem to an extent, including an ISE-based option in
their investigation of hardened AES implementations. However, the challenge
of developing suitable ISEs is under-studied in general.

% =============================================================================


% -----------------------------------------------------------------------------

\subsection{RISC-V}
\label{sec:bg:riscv}
% =============================================================================

RISC-V is a (relatively) new ISA, with its origins at UC Berkley
\cite{riscv:1}.
The core ISA is extremely simple, consisting of only $50$ instructions.
The ISA defines $32$ general purpose registers, with register $0$ always
tied to the value $0$.
The base ISA comes in $32$, $64$ and $128$-bit wide variants.
For this work, we focus on the $32$ and $64$ bit variants, since these
are the more mature and commercially relevant ones at present.

Various optional extensions are used to support more complex and specialised
functionality (e.g. the Floating point {\tt F} extension),
or to optimise for certain goals (e.g. code density and size
with the Compressed {\tt C} extension).
This work is aimed at informing work on the forthcoming Cryptographic
standard extensions for the ISA.

Further, RISC-V is a free to use, open standard managed by the independent
RISC-V Foundation.
This is in opposition to existing architectures such as ARM, which required
significant license fees to use.
This extensibility and openness makes RISC-V an excellent target for
computer architecture research.


% =============================================================================


% =============================================================================

\section{Design and implementation}
\label{sec:design}


%TODO: 64-bit core?

The ISE designs were integrated with the \CORE{2} core as a base platform
for evaluation.
The \CORE{2}\footnote{%
  \ifbool{anonymous}{Details of this core have been anonymised to comply with the TCHES submission guidelines.}{}
} core
implements the {\tt rv32imc} instruction set: 32-bit base ISA, with the
Multiply and Compressed instruction set extensions.
A block diagram of the core is shown in~\REFFIG{fig:design:cpu_block:2}.
A standard 5-stage, in-order pipeline is used, which
means that each stage of execution
(namely fetch, decode, execute, memory access, and write-back)
occurs in {\em parallel} for multiple different in-flight instructions.

Note there are two memory interfaces, one for (instruction) fetch and one for
(data) memory accesses;
no form of cache hierarchy or branch prediction is implemented.
The core implements various performance counters,
and
elements of the
RISC-V Privileged Resource Architecture (PRA)~\cite[Chapter 3]{RV:ISA:II:17}
related to exception and interrupt handling.

To support the AES ISE variants, two modifications were made to the core:
1) The Instruction Decode module was modifed to support identification and
   operand selection for the new instructions. 
2) A new ``AES" functional unit (AES FU) was added in the execute stage to
   perform the instruction computations.
Because all of the variants read at most two,
and write one general purpose register, no new structural datapaths
were needed.

All of the AES variants share the same sets of inputs, so the interface
to the AES FU is kept constant for every variant.
A synthesis time parameter was then added to switch between different
ISE variants.
The following sections describe each ISE variant using pseudo-code, and
give a simplified datapath diagram for the AES FU internals.

\begin{figure}
\centering
\includegraphics[scale=0.45,angle=90]{diagrams/scarv-cpu-uarch.png}
\caption{Core $2$: \CORE{2}.}
\label{fig:design:cpu_block:2}
\end{figure}

\begin{figure}
\centering
\begin{subfigure}[t]{0.40\textwidth}
    \centering
    \includegraphics[width=\textwidth]{diagrams/ise-datapath-v1.png}
    \caption{Variant 1}
    \label{fig:design:fu_block:v1}
\end{subfigure}
\begin{subfigure}[t]{0.40\textwidth}
    \centering
    \includegraphics[width=\textwidth]{diagrams/ise-datapath-v2.png}
    \caption{Variant 2}
    \label{fig:design:fu_block:v2}
\end{subfigure}

\begin{subfigure}[t]{0.40\textwidth}
    \centering
    \includegraphics[width=\textwidth]{diagrams/ise-datapath-v3.png}
    \caption{Variant 3}
    \label{fig:design:fu_block:v3}
\end{subfigure}
\begin{subfigure}[t]{0.40\textwidth}
    \centering
    \includegraphics[width=\textwidth]{diagrams/ise-datapath-v5.png}
    \caption{Variant 5}
    \label{fig:design:fu_block:v5}
\end{subfigure}

\caption{
Micro-architecture block diagrams for the AES functional units, for
each ISE variant.
}
\end{figure}



\subsection{Variant $1$: column-wise acceleration}
\label{sec:design:v1}
% =============================================================================

\subsubsection{ISE interface}

% TODO: instruction definition(s)

% -----------------------------------------------------------------------------

\subsubsection{ISE implementation}

\begin{lstlisting}[language=pseudo,style=block]
saes.v1.SubBytes(rs1, fwd):
  rd.8[i] = SubBytes(rs1.8[i], fwd) for i=0..3

saes.v1.MixColumns(rs1, fwd):
  rd.8[i] = MixColumn(ROTL32(rs1.32, 8*i)), fwd) for i=0..3
\end{lstlisting}

% TODO: micro-architecture diagram

% =============================================================================


\subsection{Variant $2$: combined row/column acceleration}
\label{sec:design:v2}
% =============================================================================

\REFSEC{sec:pseudo:v2} shows the mnemonics and pseudo-code functions
for \ISE{2}.
Here we reproduce the instructions proposed in Section $4.3$ of
\cite{TilGro:06}.
We continue to store the AES column-wise in four $32$-bit words.
By using two source registers however,
the ShiftRows transformation can be implicitly performed by selecting
appropriate bytes from each source word, as shown in
\REFFIG{fig:design:fu_block:v2}.
Executing $4$  {\tt v2.encs}/{\tt v2.encm} instructions each hence
performs the entire SubBytes, ShiftRows and MixColumns steps.
The {\tt v2.encs} instruction can be used for the KeySchedule by
making {\tt rs1} equal to {\tt rs2}.

A single encryption round using this variant requires $16$ instructions
in total:
four {\tt saes.v2.sub.enc} instructions to perform SubBytes and part of
shift rows,
four {\tt saes.v2.mix.enc} instructions to perform MixColumns and the
remainder of shift rows,
four load-word instructions to fetch the round key
and
four {\tt xor} instructions to add the round key.
\REFFIG{fig:round:v2} shows an example AES encrypt round function
using this variant.

Because the final round does not include MixColumns, we must
complete the final ShiftRows operation with an additional
$12$ {\tt and}/{\tt or} instructions.

% =============================================================================


\subsection{Variant $3$: T-tables acceleration}
\label{sec:design:v3}
% =============================================================================

\begin{figure}
\begin{subfigure}{\textwidth}
\begin{lstlisting}[language=pseudo,style=block]
saes.v3.encs  rd, rs1, rs2, bs : v3.Proc(rd, rs1, rs2, bs, fwd=1, mix=0)
saes.v3.encsm rd, rs1, rs2, bs : v3.Proc(rd, rs1, rs2, bs, fwd=1, mix=1)
saes.v3.decs  rd, rs1, rs2, bs : v3.Proc(rd, rs1, rs2, bs, fwd=0, mix=0)
saes.v3.decsm rd, rs1, rs2, bs : v3.Proc(rd, rs1, rs2, bs, fwd=0, mix=1)
\end{lstlisting}
\caption{
}
\label{fig:mnemonics:v3}
\end{subfigure}
\begin{subfigure}{\textwidth}
\begin{lstlisting}[language=pseudo,style=block]
lw              a0, 16(RK)      // Load Round Key
lw              a1, 20(RK)
lw              a2, 24(RK)
lw              a3, 28(RK)      // t0,t1,t2,t3 contains current round state.
saes.v3.encsm   a0, a0, t0, 0   // Next state for column 0.
saes.v3.encsm   a0, a0, t1, 1   // Current column 0 in t0.
saes.v3.encsm   a0, a0, t2, 2   // Next column 0 accumulates in a0
saes.v3.encsm   a0, a0, t3, 3
saes.v3.encsm   a1, a1, t1, 0   // Next state for column 1.
saes.v3.encsm   a1, a1, t2, 1
saes.v3.encsm   a1, a1, t3, 2
saes.v3.encsm   a1, a1, t0, 3
saes.v3.encsm   a2, a2, t2, 0   // Next state for column 2.
saes.v3.encsm   a2, a2, t3, 1
saes.v3.encsm   a2, a2, t0, 2
saes.v3.encsm   a2, a2, t1, 3
saes.v3.encsm   a3, a3, t3, 0   // Next state for column 3.
saes.v3.encsm   a3, a3, t0, 1
saes.v3.encsm   a3, a3, t1, 2
saes.v3.encsm   a3, a3, t2, 3   // a0,a1,a2,a3 contains new round state
\end{lstlisting}
\caption{
}
\label{fig:round:v3}
\end{subfigure}
\caption{
    Menmonics, pseudo code mappings and example encryption
    round function for variant 3.
    See \REFSEC{sec:pseudo}, \REFFIG{fig:pseudo:v3} for detailed
    descriptions of the pseudo-code functions.
}
\end{figure}

\REFFIG{fig:mnemonics:v3} shows the mnemonics and pseudo-code functions
for variant 3.
These instructions are based on
\cite{NadIkeKur:04,BBFR:06} and \cite{Saarinen:20},
which implement a T-tables based representation of AES \cite{DaeRij:02}.
The AES state is stored column-wise in $4$ $32$-bit words, and
each instruction selects a single byte of {\tt rs2} to operate on
using the $2$-bit {\tt bs} immediate.
This byte is used as the input to a standard T-table lookup operation,
but the table entry is calculated in hardware.
\REFFIG{fig:design:fu_block:v3} shows the data-path for these instructions.
The result of the T-table lookup is then xor'd with {\tt rs1} to
accumulate the results of the round transformation.

These instructions require only one SBox implementation to be implemented,
which is a clear advantage in resource constrained applications.
While the previous designs could be implemented with a single SBox, they
would not

We also note that \cite{Saarinen:20} improves on \cite{BBFR:06}
by using an extra source register and allowing the AddRoundKey step to be
performed implicitly, thus saving four instructions per round.

A single encryption round using this variant requires
$4$ load-word instructions to fetch the round key and
$16$ {\tt saes.v3.encs[m]} instructions to perform AddRoundKey,
SubBytes, ShiftRows and (optionally) MixColumns.
\REFFIG{fig:round:v3} shows an example AES encrypt round function
using this variant.

% =============================================================================


\subsection{Variant $4$: $64$-bit}
\label{sec:design:v4}
% =============================================================================


\begin{lstlisting}[language=pseudo,style=block]
saes.v4.ks1       rd rs1 rcon : rd = v4.ks1(rs1, rcon)
saes.v4.ks2       rd rs1 rs2  : rd = v4.ks2(rs1, rs2 )
saes.v4.imix      rd rs1      : rd = v4.InvMix(rs1)
saes.v4.encsm     rd rs1 rs2  : rd = v4.Enc(rs1, rs2, mix=1)
saes.v4.encs      rd rs1 rs2  : rd = v4.Enc(rs1, rs2, mix=0)
saes.v4.decsm     rd rs1 rs2  : rd = v4.Dec(rs1, rs2, mix=1)
saes.v4.decs      rd rs1 rs2  : rd = v4.Dec(rs1, rs2, mix=0)
\end{lstlisting}


\begin{lstlisting}[language=pseudo,style=block]
v4.ks1(rs1, enc_rcon):     // KeySchedule: SubBytes, Rotate, Round Const
    temp.32   = rs1.32[1]
    rcon      = 0x0
    if(enc_rcon != 0xA):
        temp.32 = ROTR32(temp.32, 8)
        rcon    = RoundConstants.8[enc_rcon]
    temp.8[i] = SubByte(temp.8[i])  for i=0..3
    temp.8[0] = temp.8[0] ^ rcon
    rd.64     = {temp.32, temp.32}

v4.ks2(rs1, rs2):           // KeySchedule: XOR
    rd.32[0]  = rs1.32[1] ^ rs2.32[0]
    rd.32[1]  = rs1.32[1] ^ rs2.32[0] ^ rs2.32[1]

v4.Enc(rs1, rs2, mix): // SubBytes, ShiftRows, MixColumns
    t1.128    = ShiftRows({rs2, rs1})
    t2.64     = t1.64[0]
    t3.8[i]   = SubByte(t2.8[i]) for i=0..7
    rd.32[i]  = MixColumn(t3.32[i]) if mix else t3.32[i] for i=0..1

v4.Dec(rs1, rs2, mix, hi): // InvSubBytes, InvShiftRows, InvMixColumns
    t1.128    = InvShiftRows(rs2 || rs1)
    t2.64     = t1.64[0]
    t3.8[i]   = SBox(t2.8[i]) for i=0..7
    rd.32[i]  = InvMixColumn(t3.32[i]) if mix else t3.32[i] for i=0..1

v4.InvMix(rs1):             // Inverse MixColumns
    rd.32[i]  = MixColumn(rs1.32[i]) for i=0..1
\end{lstlisting}

% TODO: micro-architecture diagram

% =============================================================================


\subsection{Variant $5$: tiled}
\label{sec:design:v5}
% =============================================================================

\REFSEC{sec:pseudo:v5} shows the mnemonics and pseudo-code functions
for variant \ISE{5}.
These instructions use a {\em tiled} approach to representing the
AES state.
Figure ({\bf TODO}) shows how the traditional column-wise representation
of AES is changed such that each {\em quadrant} of the 16-byte state
is kept in a single $32$-bit register.

We can now compute the next round state of any quadrant by sourcing
only two other quadrants (registers) at a time, thus keeping within
the $2$-read-$1$-write constraint.

The state matrix and must be re-arranged before and after applying
the round functions, which adds a small overhead to this variant.
Similarly, the KeySchedule words must also be re-arranged to allow
AddRoundKey to be performed efficiently.
This can be done as a post-processing step in the key expansion.

A single encryption round for this variant requires
$4$ load-word instructions to fetch the round key,
$4$ {\tt xor} instructions to perform AddRoundKey,
$4$ {\tt saes.v5.ersub.[lo|hi]} instructions to compute
    SubBytes, ShiftRows for each quadrant
and
$4$ {\tt saes.v5.emix} instructions to compute MixColumns for each
quadrant.
This would make it equivalent to variant 2, however we must also
account for the effort spent packing and un-packing the AES
state into the quadrant representation.
For the base ISA, this would take $12$ instructions to pack and
$12$ instructions to unpack the state.
We note that if the {\tt pack[h]} instructions from the draft
Bit-manipulation extension were included, then packing and unpacking
would be reduced to $4$ instructions.
\REFFIG{fig:round:v5} shows an example AES encrypt round function
using this variant.

% =============================================================================



% =============================================================================

\section{Evaluation}
\label{sec:eval}
\input{tex/eval-hosts.tex}
% =============================================================================

\subsection{Hardware Evaluation}
\label{sec:eval:sw}

Each ISE variant was evaluated on the host cores
described in \REFSEC{sec:design}.
The 32-bit designs (V1,V2,V3,V5) were implemented on both the
\CORE{1} and \CORE{2} cores.
The 64-bit design (v4) was only evaluated on the 64-bit configuration
of the \CORE{2} core.

Table \ref{tab:eval:hw}
shows the hardware implementation costs, while
Table \ref{tab:eval:sw}
shows performance and code size results for
each ISE, with software only versions of AES used as a baseline.

For variants 1, 2 and 5, two implementations are evaluated.
The ``Size" optimised implementations instantiate only a single
Forward/Inverse SBox and MixColumns circuit and take multiple cycles
to produce a result.
The ``Latency" optimised implementations instantiate $4$ SBox and
MixColumn circuits to produce their results in a single processor cycle.

The ISE Size column of Table \ref{tab:eval:hw} 
records the size in NAND2 equivalent gates of each variant,
instantiated independently from any wider system.
The LTP column gives the Longest Topological Path of the synthesised
functional unit circuit from input to combinatorial output.
The \CORE{2} Size column gives the size in NAND2 equivalent gates of the
\CORE{2}, with the various AES functional units integrated.
The ``Baseline" row gives the size of the core without any of the
ISEs integrated.
We found that none of the proposed ISEs affected the critical
path of the \CORE{2} core.

% ------------------------------------------------------------

\begin{table}
\centering
\begin{tabular}{lrrrr}
Variant     & ISE Size & ISE LTP & \CORE{2} Size & Size Overhead \\ \hline
Baseline    & -        & -       & 37375         & -             \\
V1 (Latency)& 3472     & 19      & 41723         & $  \%$        \\
V2 (Latency)& 3547     & 19      & 41199         & $  \%$        \\
V5 (Latency)& 4121     & 22      & 42070         & $  \%$        \\
V3          & 1157     & 30      & 38610         & $  \%$        \\
V4          &          &         &               & $  \%$        \\
V1 (Size)   & 2174     & 22      & 40161         & $  \%$        \\
V2 (Size)   & 1381     & 21      & 38885         & $  \%$        \\
V5 (Size)   & 1927     & 23      & 39251         & $  \%$        \\
\end{tabular}
\caption{
Hardware implementation costs based on the 32-bit \CORE{2} CPU core.
Gate counts and topological path lengths are obtained using the
Yosys\cite{yosys} tool suite.
}
\label{tab:eval:hw}
\end{table}



% =============================================================================


\subsection{Software Evaluation}
\label{sec:eval:sw}

To evaluate the software performance, we implemented AES 128, and
measure the static code size, instruction execution counts and cycle
counts of the Key Schedule, Encrypt and Decrypt functions.
We also separate generation of the KeySchedule for Encrypt and Decrypt.

\REFTAB{tab:eval:sw:size} shows the static code size for each
function.
We see that... ({\bf TODO}).

\REFTAB{tab:eval:sw:perf} gives cycle and instruction counts for each
variant.
Each implementation uses word-aligned state, meaning the input blocks
can be loaded with $4$ load-word instructions on $32$-bit host cores,
or $2$ load-double instructions on the $64$-bit host cores.

\begin{table}
\centering
\begin{tabular}{l|c|c|c|c|c}
Variant &
KeySchedule Enc  &
KeySchedule Dec  &
Encrypt Block    &
Decrypt Block    &
.data   \\ \hline
%Byte   & 312   &  -    &       &       & 522   \\
T-Table  & 154   &  174  & 804   & 804   & 5120  \\
V1      & 68    &  -    & 424   & 424   & 10    \\
V2      & 68    &  62*  & 234   & 238   & 10    \\
V3      & 86    &  64   & 290   & 290   & 10    \\
V4      & 168   &  100  & 268   & 268   &  0    \\
V5      & 82+208&  -    & 266   & 278   & 10    \\
\end{tabular}
\caption{
Static code size metrics for each variant, measured in bytes.
These are measured targeting the {\tt rv32imc} base ISA for all variants,
except for V4, which targets {\tt rv64imc}.
}
\label{tab:eval:sw:size}
\end{table}


%
% Commented out this table because in the real world, you'd make sure that
% your state is well aligned!
%
% \begin{table}[pt]
% \centering
% \begin{tabular}{l|cc|cc|cc|cc}
% & \multicolumn{2}{c}{\begin{tabular}[c]{@{}c@{}}KeyExpand\\ Encrypt\end{tabular}} 
% & \multicolumn{2}{c}{\begin{tabular}[c]{@{}c@{}}KeyExpand\\ Decrypt\end{tabular}}
% & \multicolumn{2}{c}{\begin{tabular}[c]{@{}c@{}}AES 128\\ Encrypt\end{tabular}}
% & \multicolumn{2}{c}{\begin{tabular}[c]{@{}c@{}}AES 128\\ Decrypt\end{tabular}} \\
% Variant     &  iret & cycles & iret & cycles & iret & cycles & iret & cycles \\ \hline
%  Byte       &  926  & 3887   & 926  & 3886   & 4228 & 7061   & 7652 & 11587 \\
%  T-Table     &  481  & 591    & 1762 & 2238   & 1013 & 1144   & 1013 & 1118  \\
% V1 (Latency)&  249  & 369    & 255  & 386    & 593  & 698    & 593  & 707   \\
% V2 (Latency)&  249  & 382    & 386  & 694    & 296  & 404    & 297  & 404   \\
% V3          &  269  & 388    & 719  & 1145   & 321  & 413    & 321  & 408   \\
% V5 (Latency)&  383  & 524    & 389  & 541    & 308  & 408    & 308  & 409   \\
% V1 (Size)   &  249  & 409    & 255  & 426    & 593  & 858    & 593  & 875   \\
% V2 (Size)   &  249  & 412    & 386  & 832    & 296  & 641    & 297  & 641   \\
% V5 (Size)   &  383  & 554    & 389  & 571    & 308  & 660    & 308  & 650   \\
% \end{tabular}
% \caption{
% Performance metrics for byte aligned state.
% }
% \label{tab:eval:sw:perf:byte}
% \end{table}

\begin{table}
\centering
\begin{tabular}{l|cc|cc|cc|cc}
& \multicolumn{2}{c}{\begin{tabular}[c]{@{}c@{}}AES 128 Block\\ Encrypt\end{tabular}}
& \multicolumn{2}{c}{\begin{tabular}[c]{@{}c@{}}AES 128 Block\\ Decrypt\end{tabular}}
& \multicolumn{2}{c}{\begin{tabular}[c]{@{}c@{}}KeySchedule\\ Encrypt\end{tabular}} 
& \multicolumn{2}{c}{\begin{tabular}[c]{@{}c@{}}KeySchedule\\ Decrypt\end{tabular}}
\\
Variant     &  iret & cycles & iret & cycles & iret & cycles & iret & cycles \\ \hline
%Byte       & 4228 & 7061 & 7652 & 11587& 926 & 3887 & 926 & 3886   \\
 T-Table     & 953  & 1078 & 953  & 1058 & 445 & 555 & 1726 & 2202    \\
V1 (Latency)& 533  & 635  & 533  & 647  & 213 & 333 & 219  & 350     \\
V2 (Latency)& 236  & 343  & 237  & 343  & 213 & 344 & 350  & 656     \\
V5 (Latency)& 248  & 346  & 248  & 349  & 347 & 489 & 353  & 506     \\
V1 (Size)   & 533  & 795  & 533  & 815  & 213 & 373 & 219  & 390     \\
V2 (Size)   & 236  & 580  & 237  & 580  & 213 & 374 & 350  & 794     \\
V3          & 261  & 351  & 261  & 346  & 233 & 352 & 683  & 1109    \\
V5 (Size)   & 248  & 598  & 248  & 590  & 347 & 519 & 353  & 536  
\end{tabular}
\caption{
Performance results for the \CORE{2} core.
Note the absence of variant 4, as it is designed for 64-bit targets only.
}
\label{tab:eval:sw:perf:scarv}
\end{table}

\begin{table}
\centering
\begin{tabular}{l|cc|cc|cc|cc}
& \multicolumn{2}{c}{\begin{tabular}[c]{@{}c@{}}AES 128 Block\\ Encrypt\end{tabular}}
& \multicolumn{2}{c}{\begin{tabular}[c]{@{}c@{}}AES 128 Block\\ Decrypt\end{tabular}}
& \multicolumn{2}{c}{\begin{tabular}[c]{@{}c@{}}KeySchedule\\ Encrypt\end{tabular}} 
& \multicolumn{2}{c}{\begin{tabular}[c]{@{}c@{}}KeySchedule\\ Decrypt\end{tabular}}
\\
Variant     &  iret & cycles & iret & cycles & iret & cycles & iret & cycles\\
\hline
%Byte       &       &        &      &        &      &        &      &      \\
T-Table     & 948   &  1143  & 949  &  1025  & 444  & 478    & 1726 & 1977 \\
V1 (Latency)& 528   &  685   & 529  &  680   & 212  & 341    & 214  & 290  \\
V2 (Latency)& 231   &  359   & 233  &  368   & 212  & 315    & 350  & 508  \\
V5 (Latency)& 243   &  414   & 244  &  319   & 346  & 427    & 348  & 424  \\
V1 (Size)   & 528   &  804   & 529  &  744   & 212  & 357    & 214  & 335  \\
V2 (Size)   & 231   &  511   & 233  &  520   & 212  & 345    & 350  & 646  \\
V3          & 253   &  895   & 254  &  873   & 233  & 470    & 674  & 2425 \\
V4          & TODO  &        &      &        &      &        &      &      \\
V5 (Size)   & 243   &  585   & 244  &  543   & 346  & 504    & 348  & 454  \\
\end{tabular}
\caption{
Performance results for the \CORE{1} core.
}
\label{tab:eval:sw:perf:scarv}
\end{table}

% ------------------------------------------------------------

We can see from the tables that ({\bf TODO}).

% =============================================================================


% =============================================================================

\section{Power Side-channel Resistance}
\label{sec:psca}

Many embedded implementations include power side-channels in their threat
model.
Having identified ISE variant $3$ detailed in
\REFSEC{sec:design:v3} as a strong candidate for embedded $32$-bit
RISC-V cores, we now show a possible
way of extending the ISE further to add power side-channel resistance
with minimal area and performance overheads.
We aim for $1$'st order side-channel security.

As described in \cite{TilGro:07}, there are several broad approaches
to adding power side-channel security to ISEs which we summarise here:

1) The data path can be implemeted in a secure logic style, which always
consumes energy at the same rate, regardless of the values being computed on.
This comes with a significant overhead in terms of silicon area and
gate-level performance.
It also requires that secret values only be stored in secure logic
elements.
This can be undermined when loading keys from memory, since they must
pass through any register stages in the memory hierarchy, which are
unlikely to be implemented in a secure logic style.

2) Random pre-charging involves sandwiching every instruction which
manipulates a secret value with instructions which operate on random values.
As the authors of \cite{TilGro:07} note, it provides only a modest
improvemnt in security, and comes with a $100\%$ performance overhead;
since every secret manipulating instruction must be accompanied by
a random-precharging partner.

3) The authors also describe a small ``secure zone'' within the processor,
implemented in a secure logic style and responsible for
mask storage, generation and all computations on secret data.
The aim is to keep the secure zone as small and separate from the main
processor datapath as possible.

\subsection{Design Overview}

We extend ISE variant $3$ to support 1'st order masking.
To this end, the secret key is represented as two boolean masked shares.
An implementation of the AES block encrypt/decrypt function requires
eight registers: four for the current round state, four to load the
next round key into and then accumulate the next round state.
Doubling this requirement to store shares of each secret variable
in the General Purpose Register (GPR) file is un-reasonable.
It would require drastic modifications to the instruction definitions and
reigster file to read four registers (two sources, of two shares each) and
write two registers.

Instead, we define a new, $8$-element ``Mask Register File'' (MRF).
Each mask register $M_i$ is $32$-bits wide, and stores the mask for
one of the GPRs.
We use a fixed mapping between GPRs and mask registers;
not all GPRs have a corresponding mask register.
({\bf TODO:} define this mapping. E.g. \{a0..a3,t0..t4\} onto MRF \{0..7\}).

Share $0$ of each secret value is loaded into the GPRs.
We define a new Load Mask instruction {\tt lm rd, imm(rs1)} which
loads {\em the mask for GPR {\tt rd}} from memory into the MRF.
A corresponding Store Mask instruction {\tt sm rs2, imm(rs1)} writes
the mask correspoding to GPR {\tt rs2} to memory.
We reqire that the secret values be stored in shared form in memory
(rather than splitting them into shares upon being loaded)
to extend the SCA prottection boundary outside the CPU.
Otherwise, the hamming weight of un-masked secret values would be
leaked by memory-hierarchy registers outside the CPU.

When an ISE instruction is executed such that any of its GPR source
registers also map onto an MRF register, both GPR and MRF are
read simultaneously and fed to the AES functional unit.
If any GPR source does not map to an MRF register, we assume that
operand is un-masked and represent the other share as $0$.

Within the AES FU the instruction result is computed entired in it's
masked representation.

The result is re-masked using a randomness source.

Share 0 goes to GPR, Share 1 to the MRF.

If destination is not a GPR, result is written back un-masked.


% =============================================================================

\section{Conclusion}
\label{sec:outro}
% =============================================================================

We have surveyed and evaluated several ISEs for accelerating
the AES block cipher in the context of RISC-V.
We find that \ISE{3} has clear advantages for embedded class
$32$-bit cores, while \ISE{4} is a natural choice for taking
advantage of the wider data-path on $64$-bit systems.
For the $32$-bit case, we have also shown that with reasonable additional
hardware, it is possible to create a $1$'st order masked implementation with
modest performance and resource overheads.

% =============================================================================


% ============================================================================

\ifbool{anonymous}{}{%
\section*{Acknowledgements}

We would like to thank the anonymous reviewers for their helpful and 
constructive comments.
This work has been supported in part by EPSRC via grant EP/R012288/1, 
under the RISE (\url{http://www.ukrise.org}) programme.
}%

% =============================================================================

\bibliographystyle{alpha}
\bibliography{paper}

% =============================================================================

\appendix

\section{ISE Instruction Pseudo Code}
\label{sec:pseudo}

%
% V1
% ------------------------------------------------------------

\begin{figure}[!htb]
\begin{lstlisting}[language=pseudo,style=block]
v1.SubByte(rd, rs1, fwd):
    rd.8[i] = AESSBox[rs1.8[i]] if fwd else AESInbSBox[rs1.8[i]] for i=0..3

v1.MixColumn(rd, rs1, fwd)
    for i=0..3:
        tmp.32  = ROTL32(rs1.32, 8*i)
        rd.8[i] = AESMixColumn(tmp.32) if fwd else AESInvMixColumn(tmp.32)
\end{lstlisting}
\caption{
    Variant 1 psuedo code.
    See section \REFSEC{sec:design:v1} for the instruction
    descriptions.
}
\label{fig:pseudo:v1}
\end{figure}

%
% V2
% ------------------------------------------------------------

\begin{figure}[!htb]
\begin{lstlisting}[language=pseudo,style=block]
v2.SubBytes(rd, rs1, rs2, fwd):
  t1.32  = {rs1.8[0], rs2.8[1], rs1.8[2], rs2.8[3]}
  rd.8[i]= AESSBox[t1.8[i]] if fwd else AESInvSBox[t1.8[i]] for i=0..3

v2.MixColumns(rd, rs1, rs2, fwd):
  t1.32  = {rs1.8[0], rs1.8[1], rs2.8[2], rs2.8[3]}
  for i=0..3:
      tmp.32 = ROTL32(rs1.32, 8*i)
      rd.8[i]= AESMixColumn(tmp.32) if fwd else AESInvMixColumn(tmp.32)
\end{lstlisting}
\caption{
    Variant 2 psuedo code.
    See section \REFSEC{sec:design:v2} for the instruction
    descriptions.
}
\label{fig:pseudo:v2}
\end{figure}

%
% V3
% ------------------------------------------------------------

\begin{figure}[!htb]
\begin{lstlisting}[language=pseudo,style=block]
v3.Proc(rd, rs1, rs2, bs, fwd, mix):
  x     = AESSBox[rs2.8[bs]] if fwd else AESInvSBox[rs2.8[bs]]
  if   mix and  fwd: t1.32 = {GFMUL(x, 3),      x    ,      x   ,GFMUL(x, 2)}
  elif mix and !fwd: t1.32 = {GFMUL(x,11),GFMUL(x,13),GFMUL(x,9),GFMUL(x,14)}
  else             : t1.32 = {0, 0, 0, x}
  rd.32 = ROTL32(t1.32, 8*bs) ^ rs1
\end{lstlisting}
\caption{
    Variant 3 psuedo code.
    See section \REFSEC{sec:design:v3} for the instruction
    descriptions.
}
\label{fig:pseudo:v3}
\end{figure}

%
% V4
% ------------------------------------------------------------

\begin{figure}[!htb]
\begin{lstlisting}[language=pseudo,style=block]
v4.ks1(rd, rs1, enc_rcon):     // KeySchedule: SubBytes, Rotate, Round Const
    temp.32   = rs1.32[1]
    rcon      = 0x0
    if(enc_rcon != 0xA):
        temp.32 = ROTR32(temp.32, 8)
        rcon    = RoundConstants.8[enc_rcon]
    temp.8[i] = AESSBox[temp.8[i]]  for i=0..3
    temp.8[0] = temp.8[0] ^ rcon
    rd.64     = {temp.32, temp.32}

v4.ks2(rd, rs1, rs2):           // KeySchedule: XOR
    rd.32[0]  = rs1.32[1] ^ rs2.32[0]
    rd.32[1]  = rs1.32[1] ^ rs2.32[0] ^ rs2.32[1]

v4.Enc(rd, rs1, rs2, mix): // SubBytes, ShiftRows, MixColumns
    t1.128    = ShiftRows({rs2, rs1})
    t2.64     = t1.64[0]
    t3.8[i]   = AESSBox[t2.8[i]] for i=0..7
    rd.32[i]  = AESMixColumn(t3.32[i]) if mix else t3.32[i] for i=0..1

v4.Dec(rd, rs1, rs2, mix, hi): // InvSubBytes, InvShiftRows, InvMixColumns
    t1.128    = InvShiftRows(rs2 || rs1)
    t2.64     = t1.64[0]
    t3.8[i]   = AESInvSBox[t2.8[i]] for i=0..7
    rd.32[i]  = AESInvMixColumn(t3.32[i]) if mix else t3.32[i] for i=0..1

v4.InvMix(rd, rs1):             // Inverse MixColumns
    rd.32[i]  = AESInvMixColumn(rs1.32[i]) for i=0..1
\end{lstlisting}
\caption{
    Variant 4 psuedo code.
    See section \REFSEC{sec:design:v4} for the instruction
    descriptions.
}
\label{fig:pseudo:v4}
\end{figure}

%
% V5
% ------------------------------------------------------------

\begin{figure}[!htb]
\begin{lstlisting}[language=pseudo,style=block]
v5.SrSub(rd, rs1, rs2, fwd, hi):
  if(fwd):
    if hi: tmp.32 = {rs1.8[3], rs2.8[0], rs2.8[1], rs2.8[2]}
    else : tmp.32 = {rs2.8[3], rs1.8[1], rs1.8[0], rs1.8[2]}
    tmp.8[i]      =    AESSBox[tmp.8[i]] for i=0..3
  else:
    if hi: tmp.32 = {rs2.8[3], rs2.8[0], rs1.8[1], rs2.8[2]}
    else : tmp.32 = {rs1.8[3], rs2.8[1], rs1.8[0], rs1.8[2]}
    tmp.8[i]      = InvAESSBox[tmp.8[i]] for i=0..3
  if(hi): rd.32 = {tmp.8[2],tmp.8[3],tmp.8[0],tmp.8[1]}
  else  : rd.32 = {tmp.8[1],tmp.8[3],tmp.8[0],tmp.8[2]}

v5.mix(rd, rs1, rs2, fwd):
  col0.32 = {rs1.8[2], rs1.8[3], rs2.8[2], rs2.8[3]}
  col1.32 = {rs1.8[0], rs1.8[1], rs2.8[0], rs2.8[1]}
  n0.8    = AESMixColumn(       col0   ) if fwd else AESInvMixColumn(       col0   )
  n1.8    = AESMixColumn(ROTL32(col0,8)) if fwd else AESInvMixColumn(ROTL32(col0,8))
  n2.8    = AESMixColumn(       col1   ) if fwd else AESInvMixColumn(       col1   )
  n3.8    = AESMixColumn(ROTL32(col1,8)) if fwd else AESInvMixColumn(ROTL32(col1,8))
  rd.32 = {n2, n3, n0, n1}
\end{lstlisting}
\caption{
    Variant 5 psuedo code.
    See section \REFSEC{sec:design:v5} for the instruction
    descriptions.
}
\label{fig:pseudo:v5}
\end{figure}



% =============================================================================

\end{document}
