\documentclass[preprint]{iacrtrans}

\usepackage{paper}

\title{An exploration of ISEs for AES on RISC-V}
\keywords{ISE, AES, RISC-V}

\ifbool{anonymous}{%
\author{}
\institute{}
}{%
\author{
Andy Glew\inst{1}                   \and
Ben Marshall\inst{2}                \and
G. Richard Newell\inst{3}           \and
Dan Page\inst{2}                    \and
Markku-Juhani O. Saarinen\inst{4}   \and
Barry Spinney\inst{5}               \and
Claire Wolf\inst{6}
}
\institute{
SiFive, Inc. \\ \email{andy.glew@sifive.com}
\and
Department of Computer Science, University of Bristol \\ \email{{ben.marshall,daniel.page}@bristol.ac.uk}
\and
Microchip Technology Inc., USA \\ \email{richard.newell@microchip.com}
\and
PQShield, UK \\ \email{mjos@pqshield.com}
\and
Nvidia Corporation \\ \email{first.last@nvidia.com}
\and
Symbiotic EDA \\ \email{claire@symbioticeda.com}
}
}%

\begin{document}

% =============================================================================

\maketitle

\begin{abstract}
Secure, efficient execution of AES is an essential requirement for most
computing platforms, often motivating inclusion of dedicated
Instruction Set Extensions (ISEs).
RISC-V is a (relatively) new ISA that lacks such a standardised ISE.
We survey the state-of-the-art industrial and academic ISEs for AES,
implement and evaluate five different ISEs, one of which is novel,
and make recommendations for standardisation.
We consider the side-channel security implications of the ISE designs,
demonstrating how an implementation of one candidate ISE can be hardened
against DPA-style attacks.
We also explore how the proposed standard Bit-manipulation extension
to RISC-V can be harnessed for efficient implementation of AES-GCM.
Our work supports the ongoing RISC-V cryptography extension standardisation
process.
\end{abstract}

% =============================================================================

\section{Introduction}
\label{sec:intro}
% =============================================================================

\paragraph{Implementing the Advanced Encryption Standard (AES).}

In $2001$, NIST pronounced Rijndael~\cite{DaeRij:98,DaeRij:02}, 
a design due to Daemen and Rijmen, 
as winner of a $5$-year standardisation process~\cite{NBBBDFR:01} instigated 
to identify a replacement for the incumbent
Data     Encryption Standard (DES)~\cite{FIPS:46} 
block cipher; the resulting 
Advanced Encryption Standard (AES)~\cite{FIPS:197} 
was announced in $2001$.

Compared to more general cases, cryptographic kernels such as AES present a
significant challenge wrt. implementation, because they often
a) involve computationally intensive, somewhat niche functionality,
b) are deployed in a diverse range of contexts,
   and
c) form a central target in what is a complex attack surface.
The demand for efficiency, however measured, forms an overarching example
with at least two dimensions.
First,
cryptography often represents an enabling technology vs. a feature: it will
likely be viewed as overhead, therefore, when viewed from the perspective 
of a user.  Put another way,  (even {\em perceived}) {\em in}efficiency may
be problematic in terms of fitness for purpose.  Addressing this is further 
complicated by any constraints associated with the context, e.g., a demand 
for
high-volume, 
 low-latency, 
high-throughput, 
 low-footprint, 
and/or 
 low-power.
Second,
and although efficiency is a goal in and of itself therefore, it {\em also} 
acts as an enabler for security.  This is because one cannot 
(or at least {\em should} not) 
compromise security to meet efficiency requirements, implying that delivery 
of higher efficiency offers greater margin within which to deliver suitable
countermeasures against attack.

AES represents an interesting case-study wrt. techniques which attempt to 
address the challenge of efficient, secure implementation.  For example,
per the request for candidates announcement\footnote{%
\url{https://www.govinfo.gov/content/pkg/FR-1997-09-12/pdf/97-24214.pdf}
}, the AES process was instrumental in popularising a model in which
{\em both}
``security''
(e.g., resilience against cryptanalytic attack)
{\em and}
``algorithm and implementation characteristics''
(e.g., computational efficiency, memory requirements, suitability for different platform classes)
form important quality metrics for the {\em design}, in order to facilitate
techniques for higher quality {\em implementations} of it.
In addition,
both the design {\em and} implementations of it are long-lived.
From a positive perspective, 
the central importance of AES has yielded special emphasis on related
research and development effort before, during, and, most significantly, 
after the AES process.
From a negative perspective, however,
the $20+$ year period since standardisation has forced an evolution of 
implementation technique, e.g., to match evolution of both the technology 
and attack landscape.  For example,~\cite[Section 3.6]{NBBBDFR:01} covers
implementation (e.g., side-channel) attacks: this field has become richer,
and the associated threat more potent during said period.

% -----------------------------------------------------------------------------

\paragraph{Support via Instruction Set Extensions (ISEs).}

A potentially large space of implementation techniques will often exist
for a given cryptographic kernel.  One could classify instances as being 
   kernel-agnostic
   or
   kernel-specific,
and based on the use of   
   hardware              only,
                software only,
   or
   hardware and software (i.e., a hybrid of the two).
The latter class includes instances based on the concept of an
Instruction Set Extension (ISE)~\cite{GalBer:11,BarGioMar:09,RegIen:16}.
The idea is to identify, e.g., through benchmarking, a set of additional 
instructions that allow the kernel to leverage
special-purpose, domain-specific functionality
in the resulting ISE,
vs. 
general-purpose                  functionality
in the base Instruction Set Architecture (ISA),
and thereby deliver improvement wrt. pertinent quality metrics.  
Although ISEs can be an effective option for {\em both}
high(er)-end, performance-oriented
and
 low(er)-end, constrained
platforms, 
they are particularly effective for the latter by 
improving footprint and latency
vs. a software-only option
while also
improving area      and flexibility
vs. a hardware-only option.

Abstractly, an ISE design constitutes
1) an {\em interface},
   i.e., 
   addition of instructions to some 
   base ISA
   plus
2) an {\em implementation},
   i.e., 
   support for execution of said instructions through changes or addition 
   to a 
   base micro-architecture.
As a fundamental, long-lived computer systems interface, the design of or
changes to an ISA demands careful consideration
(cf.~\cite[Section 4]{Gueron:09}); this implies that the production of a 
concrete ISE design is far from trivial.  
Such a design must deliver a demonstrable improvement wrt. the kernel in 
question, {\em as well as} considering a (non-exhaustive) list of design 
goals such as

\begin{itemize}
\item minimal invasiveness,
      e.g.,
      limit the number and scope (local vs. global) of changes,
\item minimal overhead,
      e.g.,
      avoid instructions that demand support via large or complex hardware components,
\item maximal compatibility,
      e.g.,
      align with constraints in the base ISA, such as use of existing instruction formats,
      and
\item maximal       utility,
      i.e.,
      favour general- vs. {\em too} special-purpose functionality.
\end{itemize}

\noindent
By extending x86, Intel provide archetypal examples of and evidence in
support of ISEs.
For example, various generations of
non-cryptographic
Multi-Media      eXtensions (MMX),
Streaming SIMD  Extensions (SSE),
and
Advanced Vector Extensions (AVX)
support numerical kernels via vector (or SIMD) vs. scalar computation.  
Likewise, the
    cryptographic
Advanced Encryption Standard New Instructions (AES-NI)~\cite{Gueron:09,DruGueKra:19}
ISE
supports AES: it significantly improves latency and throughput
(see, e.g.,~\cite{FazLopOli:18}),
and represents a pertinent case-study wrt. the guidelines above by
a) adding just $6$ additional (vs. $1500+$ total) instructions,
b) reducing overhead by, e.g., sharing the XMM register file,
c) facilitating compatibility via the
   \VERB{CPUID}~\cite[Chapter 20]{X86:1:18}
   feature identification mechanism,
   and
d) offering utility beyond AES specifically:
   it supports AES-specific functionality but is able to maximise utility 
   through more AES-agnostic support of {\em other} kernels
   (see, e.g., use of AES-NI within SHA3~\cite{BBGR:09} and CEASER~\cite[Section 4.1]{AnkAnk:16} candidates).

% -----------------------------------------------------------------------------

\paragraph{Remit.}

In this paper, we address the challenge of supporting AES by designing and
evaluating an associated ISE for RISC-V
(see, e.g.,~\cite{riscv:1,riscv:2}).
Note that we specifically focus on extending the scalar RISC-V instruction 
set, thereby contrasting with work related to cryptographic support in the
standard V (or vector) extension~\cite[Section 21]{RV:ISA:I:19}.

On one hand, 
RISC-V represents an excellent vehicle for such work:
extensibility is a by-design feature in the ISA, whose open nature renders
exploration of such extensions easier by virtue of the range of associated 
(often open-source) implementations.  
Increased commercial deployment of such implementations suggests that work 
on RISC-V timely, and potentially of high impact.
On the other hand, however,
RISC-V also presents some unique challenges vs. previous work.
For example,
RISC-V could in fact be viewed as {\em three} related base ISAs,
 RV32I~\cite[Section 2]{RV:ISA:I:19},
 RV64I~\cite[Section 5]{RV:ISA:I:19},
and
RV128I~\cite[Section 6]{RV:ISA:I:19},
that each support a different word size:
designing ISEs that are applicable (or scale) across these options is a
complicating factor, yet an important design goal.

% TODO: need to be clearer what constraints are for a given design

% -----------------------------------------------------------------------------

\paragraph{Organisation.}

% TODO

\REFSEC{sec:bg} describes the AES cipher, existing implementation strategies
and prior commercial and academic ISE designs.
\REFSEC{sec:design} details five candidate ISEs, including the criteria
used to select and design them.
\REFSEC{sec:eval} gives evaluation results in terms of hardware cost
and software performance.
\REFSEC{sec:psca} shows how our ISE recommendation for highly resource
constrained devices can be further extended to give power side-channel
resistance.
Finally, 
\REFSEC{sec:outro}
presents some conclusions and potential directions for future work.

\ifbool{submission}{%
Note that in order to satisfy the TCHES submission guidelines, we have 
anonymised various resources and references.  We intend to open-source 
such resources post-submission, but could provide them to reviewers, 
if required, to facilitate the review process.
}{}%

% =============================================================================

% =============================================================================

\paragraph{RISC-V}
\label{sec:bg:riscv}

RISC-V is a (relatively) new ISA, with academic origins~\cite{riscv:1,riscv:2}.
Unlike alternatives such as x86 or ARMv8-A, RISC-V is a free to use, 
open standard: this is managed by the independent
RISC-V International Foundation.
The base ISA is extremely simple, consisting of only $50$ instructions,
and adopts {\em strongly} RISC-oriented design principles.  However, it
is also highly modular: a general-purpose base ISA can (optionally) be
supplemented using sets of special-purpose, standard or non-standard
extensions to
support additional functionality 
(e.g., floating-point, 
       via the 
       standard F~\cite[Section 11]{RV:ISA:I:19}
                and
                D~\cite[Section 12]{RV:ISA:I:19}
       extension),
or 
satisfy specific optimisation goals
(e.g., code density, 
       via the 
       standard C~\cite[Section 16]{RV:ISA:I:19}
       extension).
The RISC-V International Foundation delegates the development of
extensions to a dedicated task group.
The Cryptographic Extensions Task
Group\footnote{
  \url{https://lists.riscv.org/g/tech-crypto-ext}
} provides some specific context for this paper, through their remit to 
develop scalar and vector extensions to support cryptography.

We focus wlog. on extending either
RV32I~\cite[Section 2]{RV:ISA:I:19}
or
RV64I~\cite[Section 5]{RV:ISA:I:19},
i.e., the
$32$-bit 
or
$64$-bit 
integer RISC-V base ISA.
Let $\GPR[*][ i ]$, for $0 \leq i < 32$, denote the $i$-th entry of the 
General-Purpose Register (GPR) file.  Note that 
$
\GPR[*][ 0 ] = 0 ,
$
meaning the $0$-th register is fixed to $0$.
RISC-V uses \RVXLEN to denote the word size; we adopt the same approach, 
but by focusing on 
RV32I 
or
RV64I 
assume a focus on 
$\RVXLEN = 32$
or
$\RVXLEN = 64$.

% =============================================================================


\paragraph{Remit and organisation.}

In this paper we address the challenge of supporting AES on the RISC-V base
ISA
(see, e.g.,~\cite{riscv:1,riscv:2}),
and therefore inform on-going efforts to standardise cryptographic ISEs for 
RISC-V.  In specific terms, our contributions are as follows:

\begin{enumerate}

\item In 
      \REFSEC{sec:bg}
      we capture some background, including a limited form of
      Systematisation of Knowledge (SoK)
      wrt. ISEs for AES.

\item In 
      \REFSEC{sec:ise}
      we implement and evaluate five different ISEs for AES on two different 
      RISC-V compliant base micro-architectures.
      As well as exploring existing ISE designs, 
      \REFSEC{sec:ise:design:v5}
      introduces what is, to the best of our knowledge, a novel ISE design 
      that leverages a quadrant-packed representation of the state.

\item In
      \REFSEC{sec:gcm}
      we evaluate how the
      standard 
      B 
      extension~\cite[Section 21]{RV:ISA:I:19}
      to RISC-V can be harnessed for efficient implementation of AES-GCM.

\item In
      \REFSEC{sec:sca}
      we select one candidate ISE design from 
      \REFSEC{sec:ise},
      and demonstrate how the associated implementation can be hardened
      against DPA-style attacks.

\end{enumerate}

\noindent
On one hand, 
RISC-V represents an excellent vehicle for such work:
extensibility is a by-design feature in the ISA, whose open nature renders
exploration of such extensions easier by virtue of the range of associated 
(often open-source) implementations.  
Increased commercial deployment of such implementations suggests that work 
on RISC-V timely, and potentially of high impact.
On the other hand, however,
RISC-V also presents some unique challenges vs. previous work.
For example,
RISC-V could in fact be viewed as {\em three} related base ISAs,
 RV32I~\cite[Section 2]{RV:ISA:I:19},
 RV64I~\cite[Section 5]{RV:ISA:I:19},
and
RV128I~\cite[Section 6]{RV:ISA:I:19},
that each support a different word size:
designing ISEs that are applicable (or scale) across these options is a
complicating factor.

\ifbool{submission}{%
Note that in order to satisfy the TCHES submission guidelines, we have 
anonymised various resources and references.  We intend to open-source 
such resources post-submission, but could provide them to reviewers, 
if required, to facilitate the review process.
}{}%



% =============================================================================

\section{Background}
\label{sec:bg}

FIPS-197~\cite{FIPS:197} represents the definitive specification
of AES. 
An overview of related design rationale is offered in ~\cite{DaeRij:02}.
We endeavour to follow the notation set out in ~\cite{FIPS:197}
in referencing specific parts of AES functionality.

% -----------------------------------------------------------------------------

%\subsection{AES specification}
%\label{sec:bg:aes_spec}
%% =============================================================================

% -----------------------------------------------------------------------------

\paragraph{Syntax.}

As a block cipher, AES defines two algorithms
\[
\begin{array}{lcl}
\ALG{Enc} &:& \SET{ 0, 1 }^{8 \cdot 4 \cdot Nk} \times \SET{ 0, 1 }^{8 \cdot 4 \cdot Nb} \rightarrow \SET{ 0, 1 }^{8 \cdot 4 \cdot Nb} \\
\ALG{Dec} &:& \SET{ 0, 1 }^{8 \cdot 4 \cdot Nk} \times \SET{ 0, 1 }^{8 \cdot 4 \cdot Nb} \rightarrow \SET{ 0, 1 }^{8 \cdot 4 \cdot Nb} \\
\end{array}
\]
such that
$
m = \ALG{Dec}( k, c = \ALG{Enc}( k, m ) ) .
$
That is, given a plaintext $m$ and cipher key $k$, \ALG{Enc} encrypts $m$ 
under $k$; given the same $k$, \ALG{Dec} will invert \ALG{Enc} and so the
{\em same} $m$ can be recovered from the associated ciphertext $c$.  
In addition, it defines an algorithm
\ALG{KeyExp}
that expands~\cite[Section 5.2]{FIPS:197} the cipher key into a sequence 
of round keys then used by
\ALG{Enc}
or
\ALG{Dec};
where appropriate, we use
\[
\begin{array}{lcl}
\ALG{Enc-KeyExp} &:& \SET{ 0, 1 }^{8 \cdot 4 \cdot Nk} \rightarrow \SET{ 0, 1 }^{( 8 \cdot 4 \cdot Nb ) \times ( Nr + 1 )} \\
\ALG{Dec-KeyExp} &:& \SET{ 0, 1 }^{8 \cdot 4 \cdot Nk} \rightarrow \SET{ 0, 1 }^{( 8 \cdot 4 \cdot Nb ) \times ( Nr + 1 )} \\
\end{array}
\]
to denote said algorithm as specialised to suit
\ALG{Enc}
and
\ALG{Dec}
respectively.

% -----------------------------------------------------------------------------

\paragraph{Parameterisation.}

An AES parameter set~\cite[Figure 4]{FIPS:197}
is a triple
$
\TUPLE{ Nk, Nb, Nr }
$
where 
$Nk$ dictates the number of $32$-bit words in $k$,
$Nb$ dictates the number of $32$-bit words in $m$ or $c$ (i.e., a block),
and
$Nr$ dictates the number of rounds.  The standard AES parameter sets are
\[
\begin{array}{lcl}
\mbox{AES-128} &\mapsto& \TUPLE{ 4, 4, 10 } \\
\mbox{AES-192} &\mapsto& \TUPLE{ 6, 4, 12 } \\
\mbox{AES-256} &\mapsto& \TUPLE{ 8, 4, 14 } \\
\end{array}
\]
such that the number of bits in a plaintext (resp. ciphertext) block is fixed to 
$
8 \cdot 4 \cdot Nb = 128 .
$
From here on, we focus wlog. on encryption using AES-128 (other parameter 
sets are catered for naturally, and decryption with minor differences) so
use the terms AES and AES-128 synonymously.

% -----------------------------------------------------------------------------

\paragraph{Design.}

The mathematics underpinning AES are described in ~\cite[Section 4]{FIPS:197}.
In particular, it can be defined in terms of 
operations in the finite field $\F_{2^{  8}}$ constructed as
$
\F_{2}[\IND{x}] / ( \IND{x}^{8} + \IND{x}^{4} + \IND{x}^{3} + \IND{x} + 1 ) .
$
A hexadecimal short-hand~\cite[Section 3.2]{FIPS:197} is used to represent 
field literals, e.g.,
$
\AESCONST{13} ~\mapsto~ \RADIX{13}{16} ~\equiv~ \RADIX{00010011}{2} ~\mapsto~ \IND{x}^4 + \IND{x} + 1 .
$
Field 
      addition, 
multiplication, 
and  
      division
are denoted by
$\AESADD$,
$\AESMUL$,
and
$\AESINV$
respectively,
with the multiplication-by-$\IND{x}$ operation~\cite[Section 4.2.1]{FIPS:197} 
denoted \AESFUNC{xtime}.
Elements of $\F_{2^8}$ are collected into $( 4 \times 4 )$-element state
and round key matrices; the $i$-th row and $j$-th column of such a matrix 
relating to round $r$ is denoted
$\AESRND {s}{r}_{i,j}$
and
$\AESRND{rk}{r}_{i,j}$
respectively, with super- and/or subscripts omitted whenever irrelevant.

AES is an iterative block cipher, based on a substitution-permutation network.
This means encryption using AES can be described~\cite[Section 5.2]{FIPS:197}
as follows:
1)    the  input  plaintext is pre-whitened to yield
      $\AESRND {s}{  0} = m \AESADD \AESRND{rk}{0} = m \AESADD k$,
2)    each $r$-th round, for $1 \leq r \leq Nr$, demands computation of
      $\AESRND {s}{r+1} = \ALG{P-layer}( \ALG{S-layer}( \AESRND{s}{r}                        ) ) \AESADD \AESRND{rk}{r}$,
      and therefore use of round key
      $\AESRND{rk}{r  }$,
3)    the output ciphertext is
      $c = \AESRND{s}{Nr}$.
Note that an alternative round definition, namely
      $\AESRND {s}{r+1} = \ALG{P-layer}( \ALG{S-layer}( \AESRND{s}{r} \AESADD \AESRND{rk}{r} ) )                       $ ,
is plausible: this shifts the 
 pre-whitening step {\em before} 2) 
into an analogous 
post-whitening step {\em  after} 2)
to yield an equivalent result.
At a  low(er) level,
the computation of each round is specified via four round functions (each of 
which has an inverse, to support decryption):

\begin{itemize}

\item \AESFUNC{SubBytes}
      ~\cite[Section 5.1.1]{FIPS:197}
      operates element-wise,
      computing
      $\AESRND{s}{r+1}_{i,j} = \ALG{S-box}( \AESRND{s}{r}_{i,j} )$
      via application of the S-box:
      given an element $x$, this component can be described as
      \[
      \begin{array}{lcl}
      \ALG{S-Box} &:& \left\{\begin{array}{ccc}
                             \F_{2^8} &\rightarrow& \F_{2^8} \\
                             x        &\mapsto    & f(g(x))  \\
                             \end{array}
                      \right.
      \end{array}
      \]
      where 
      $g$ is an inversion, 
      and 
      $f$ is a specially selected affine transformation.
      Where appropriate,
      we overload \AESFUNC{SubBytes} by allowing it to denote application 
      of the S-box to {\em any} collection, 
      e.g., a row, column, or, more generally, a sequence, 
      of elements.

\item \AESFUNC{ShiftRows}
      ~\cite[Section 5.1.2]{FIPS:197}
      operates     row-wise,
      rotating each 
      $i$-th row 
      of 
      $\AESRND{s}{r  }$
      by $i$ elements
      to form 
      the associated row    of
      $\AESRND{s}{r+1}$,
      i.e.,
      $\AESRND{s}{r+1}_{i,j} = \AESRND{s}{r}_{i,j + i \pmod{Nb}}$.
      Where appropriate,
      we use
      \AESFUNC{ShiftRow}
      to denote
      the operation applied to a single 
      row
      within \AESFUNC{ShiftRows}.

\item \AESFUNC{MixColumns}
      ~\cite[Section 5.1.3]{FIPS:197}
      operates  column-wise,
      multiplying each 
      $j$-th column
      of 
      $\AESRND{s}{r  }$
      with a constant MDS matrix
      to form 
      the associated column of
      $\AESRND{s}{r+1}$.
      Where appropriate,
      we use
      \AESFUNC{MixColumn}
      to denote
      the operation applied to a single 
      column 
      within \AESFUNC{MixColumns}, i.e., multiplication of a $4$-element 
      column vector by the constant MDS matrix.
      
\item \AESFUNC{AddRoundKey}
      ~\cite[Section 5.1.4]{FIPS:197}
      operates element-wise,
      computing
      $\AESRND{s}{r+1}_{i,j} = \AESRND{s}{r}_{i,j} \AESADD \AESRND{rk}{r}_{i,j}$ 
      and thereby mixing a round key into the state.

\end{itemize}

\noindent
Note that
$
\ALG{S-layer} = \AESFUNC{SubBytes} ,
$
and
\[
\ALG{P-layer} = \left\{\begin{array}{l@{\;}c@{\;}l lr}
                       \AESFUNC{MixColumns} &\circ& \AESFUNC{ShiftRows} & \mbox{in rounds} & 1 \leq r < Nr \\
                                            &     & \AESFUNC{ShiftRows} & \mbox{in round } &            Nr \\
                       \end{array}
                \right.
\]
i.e., the last, $Nr$-th round differs from the initial $Nr - 1$ rounds.  As
such, a round as defined above is constructed via
$
\AESFUNC{AddRoundKey} \circ \AESFUNC{MixColumns} \circ \AESFUNC{ShiftRows} \circ \AESFUNC{SubBytes} 
$
or
$
\AESFUNC{AddRoundKey} \circ                            \AESFUNC{ShiftRows} \circ \AESFUNC{SubBytes}
$
respectively, where, because \AESFUNC{ShiftRows} and \AESFUNC{SubBytes}
commute, the order they are applied in can be selected to suit.



\subsection{AES implementation}
\label{sec:bg:aes_impl}

\paragraph{Representation.}
\label{sec:bg:aes_impl_rep}
% =============================================================================

A field element in $\F_{2^8}$ can be represented by an
$8$-bit byte,
where the $i$-th bit of $x$ for $0 \leq i < 8$ represents the $i$-th 
polynomial coefficient.

Beyond this, the state and round key matrices can be represented in
several ways.
The most direct option would be termed
array-based (or unpacked):
the matrix is represented as a $16$-element array of $8$-bit bytes, each
representing field elements.
%FIPS-197~\cite{FIPS:197} defines a word to be st. $w = 32$.
We use $R$ to refer to the register width of a target platform.
For RISC-V, $R = \RVXLEN$ where we consider $\RVXLEN \in {32,64}$.
Where $R \geq  32$,
an entire row or column of the AES state matrix can be packed into each 
register:
we term these
   ``row-packed''  
and
``column-packed''
representations respectively.
Where $R \geq 128$, 
it is plausible to pack
an entire AES state matrix
into a single register: 
we term this a 
 ``fully-packed'' 
representation.

% =============================================================================


\paragraph{Hardware-only implementations.}
\label{sec:bg:aes_impl_hw}
% =============================================================================

In a hardware-only implementation,
execution of 
AES
is 
performed by 
a dedicated hardware module (e.g., a memory-mapped co-processor).
A large design space exists for hardware implementations of AES.
Gaj and Chodowiec~\cite[Section 3.3]{GajCho:00}
give an overview, detailing
iterative,
combinatorial (unrolled),
and
pipelined architectures.
Similarly, ~\cite{PMDW:04,GooBen:05,GajCho:09}
survey concrete implementations on a variety of fabrics including FPGAs
and ASICs.

Although hardware-only designs are not our focus, the associated techniques
can guide ISE-related design choices.
First,
they guide the ISE interface.
For example, some ISEs can be characterised as offering an interface to
hardware constituting one round 
(i.e., aligned with an iterative hardware implementation).
Second,
they guide the ISE implementation.
For example, a significant body of work focuses on efficient hardware 
implementation of the S-box: ~\cite{Canright:05,BoyPer:12,ReyTahAsh:18}.

% =============================================================================

\paragraph{Software-only implementations.}
\label{sec:bg:aes_impl_sw}
% =============================================================================

% TODO

Note that we consider ``vanilla'' ISAs only, so exclude work related to the
use of, e.g., vector-like extensions~\cite{Hamburg:09}.

In combination,
Bernstein and   Schwabe~\cite{BerSch:08}
and
Schwabe   and Stoffelen~\cite{SchSto:16}
present and compare a range of software-based implementation and 
optimisation techniques, across a range of platforms.

% -----------------------------------------------------------------------------

\paragraph{Compute-oriented.}

A compute-oriented implementation of AES favours
 online     computation, 
thus reducing 
memory footprint
at the cost of increased 
latency.
Following~\cite[Section 4.1]{DaeRij:02}, for example, the idea is to simply
1) adopt an
    array-packed
   representation of state and round key matrices,
   then
2) construct a round implementation by following the algorithmic description
   of each round function in a direct manner.
Addition in $\F_{2^8}$ can be realised using a native XOR instruction; this
native support is seldom afforded to multiplication and inversion, however.
As a result, it is common to pre-compute the \ALG{S-box} and \AESFUNC{xtime} 
functions:
doing so demands pre-computation and storage of a
$
\SI{256}{\byte}
$
look-up table per function, but significantly reduces execution latency.

On platforms where $w = 32$,
Bertoni et al.~\cite{BBFMM:02}
further improve execution latency by exploiting the wider data-path.  Their
idea is to
1) adopt a 
      row-packed
   representation of state and round key matrices,
2) implement
   \AESFUNC{ShiftRows}
   by using native rotation instructions to act on the packed
   rows,
3) implement
   \AESFUNC{MixColumns}
   by harnessing the SIMD Within A Register (SWAR) paradigm:
   by applying
   \AESFUNC{xtime}
   across a packed row in parallel,
   a carefully organised scheme for evaluating
   \AESFUNC{MixColumns}
   can be constructed.

% -----------------------------------------------------------------------------

\paragraph  {Table-oriented.}

A  table-oriented implementation of AES favours
offline pre-computation,
thus reducing 
latency
at the cost of increased 
memory footprint.
The archetypal example of this technique is use of so-called
T-tables~\cite[Section 4.2]{DaeRij:02}.
In short, doing so means
1) adopting a 
   column-packed
   representation of state and round key matrices,
2) pre-computing
   $
   \AESFUNC{MixColumn} \circ \AESFUNC{SubBytes}
   $
   using the tables
   \[
   \begin{array}{cc}
   \begin{array}{lcl}
   T_0[x] &=& \left[\begin{array}{c}
                    \RADIX{02}{16} \AESMUL \ALG{S-box}( x ) \\
                    \RADIX{01}{16} \AESMUL \ALG{S-box}( x ) \\
                    \RADIX{01}{16} \AESMUL \ALG{S-box}( x ) \\
                    \RADIX{03}{16} \AESMUL \ALG{S-box}( x ) \\
                    \end{array} \right]
   \end{array}
   &
   \begin{array}{lcl}
   T_1[x] &=& \left[\begin{array}{c}
                    \RADIX{03}{16} \AESMUL \ALG{S-box}( x ) \\
                    \RADIX{02}{16} \AESMUL \ALG{S-box}( x ) \\
                    \RADIX{01}{16} \AESMUL \ALG{S-box}( x ) \\
                    \RADIX{01}{16} \AESMUL \ALG{S-box}( x ) \\
                    \end{array} \right]
   \end{array}
   \\\\
   \begin{array}{lcl}
   T_2[x] &=& \left[\begin{array}{c}
                    \RADIX{01}{16} \AESMUL \ALG{S-box}( x ) \\
                    \RADIX{03}{16} \AESMUL \ALG{S-box}( x ) \\
                    \RADIX{02}{16} \AESMUL \ALG{S-box}( x ) \\
                    \RADIX{01}{16} \AESMUL \ALG{S-box}( x ) \\
                    \end{array} \right]                 
   \end{array}
   &
   \begin{array}{lcl}
   T_3[x] &=& \left[\begin{array}{c}
                    \RADIX{01}{16} \AESMUL \ALG{S-box}( x ) \\
                    \RADIX{01}{16} \AESMUL \ALG{S-box}( x ) \\
                    \RADIX{03}{16} \AESMUL \ALG{S-box}( x ) \\
                    \RADIX{02}{16} \AESMUL \ALG{S-box}( x ) \\
                    \end{array} \right]
   \end{array}
   \end{array}
   \]
   for $x \in \F_{2^8}$,
3) computing each $j$-th column of $\AESRND{s}{r+1}$ as
   \[
   T_0[ \AESRND{s}{r}_{i, j + i \pmod{Nb}} ] \AESADD
   T_1[ \AESRND{s}{r}_{i, j + i \pmod{Nb}} ] \AESADD
   T_2[ \AESRND{s}{r}_{i, j + i \pmod{Nb}} ] \AESADD
   T_3[ \AESRND{s}{r}_{i, j + i \pmod{Nb}} ]
   \]
   where extraction of elements caters for \AESFUNC{ShiftRows}, then XOR'ing 
   the $j$-th column of $\AESRND{rk}{r}$ to cater for \AESFUNC{AddRoundKey}.

As such, each round amounts to a sequence of look-ups into $T_i$, plus XORs 
to combine their result; 
doing so demands pre-computation and storage of a
$
256 \cdot \SI{4}{\byte} = \SI{1}{\kilo\byte}
$
look-up table per $T_i$.
However, note that the overhead related to extraction of each element from 
packed columns representing $\AESRND{s}{r}$ 
(to form look-table offsets) 
is not insignificant:
Fiskiran and Lee~\cite{FisLee:01}
analyse the impact of different addressing modes on this issue, with
Stoffelen~\cite[Section 3.1]{Stoffelen:19}
concluding that RISC-V is (relatively) ill-equipped to reduce said overhead,
due to the provision of a (relatively) sparse set of addressing modes.

% -----------------------------------------------------------------------------

\paragraph{Use of bit-slicing.}

% TODO

\cite{MatNak:07,Konighofer:08,KasSch:09}

\cite{Stoffelen:19}

% =============================================================================

\subsection{Existing AES ISEs}
\label{sec:bg:aes_impl_ise}
% =============================================================================

Here, we survey AES-related ISE designs split into
1) industry-specified ISEs,
   which are {\em     standard} extensions,
   and
2) academia-specified ISEs,
   which are {\em non-standard} extensions,
wrt. a given base ISA.
   Each ISE is classified as either
   workload-specific,
   if it is only useful for AES,
   or
   workload-agnostic,
   if it is      useful for AES and other workloads.
Note that we exclude work where an ISE for another workload can be applied 
{\em  to} AES
but was not designed 
{\em for} AES
(see, e.g., Tillich and Gro{\ss}sch\"{a}dl~\cite{TilGro:04} who apply an ISE intended for ECC to AES).

% =============================================================================

\subsubsection{Standard, industry-specified ISEs}

\noindent
{\bf Intel}
      introduced support for AES in 
      x86
      per~\cite[Section 12.13]{X86:1:18}.
      Instructions use a
          destructive $2$-address ($1$ source, $1$ source/destination)  
      or
      non-destructive $3$-address ($2$ source, $1$        destination)
      format
      depending on the variant (e.g., XMM- vs. AVX-based),
      and operate on data housed in the pre-existing
      vector 
      register file, implying $R = 128$.
      AES is implemented by
      1) adopting a 
          fully-packed
         representation of state and round key matrices,
         then
      2) using
             \VERB{AESENC}         ~\cite[Page 3-54]{X86:2:18}
         to construct a round implementation as
         \[
         \VERB{AESENC} \mapsto \AESFUNC{AddRoundKey} \circ \AESFUNC{MixColumns} \circ \AESFUNC{SubBytes} \circ \AESFUNC{ShiftRows}
         \]
%     Note that
%            \VERB{AESENCLAST}     ~\cite[Page 3-56]{X86:2:18}
%     supports 
%     the $Nr$-th round;
%     additional instructions are provided to 
%     support
%     decryption
%     (i.e., \VERB{AESDEC}         ~\cite[Page 3-50]{X86:2:18}
%            and
%            \VERB{AESDECLAST}     ~\cite[Page 3-52]{X86:2:18})
%     and
%     key expansion
%     (i.e., \VERB{AESKEYGENASSIST}~\cite[Page 3-59]{X86:2:18}
%            and
%            \VERB{AESIMC}         ~\cite[Page 3-58]{X86:2:18}).

\noindent
{\bf IBM}
      introduced support for AES in 
      POWER
      per~\cite[Section 6.11.1]{POWER:18}.
      Instructions use a
      non-destructive $3$-register ($2$ source, $1$        destination)
      format,
      and operate on data housed in the pre-existing
      vector 
      register file, implying $R = 128$.
      AES is implemented by
      1) adopting a 
          fully-packed
         representation of state and round key matrices,
         then
      2) using
             \VERB{vcipher}     ~\cite[Page 304]{POWER:18}
         to construct a round implementation as
         \[
         \VERB{vcipher} \mapsto \AESFUNC{AddRoundKey} \circ \AESFUNC{MixColumns} \circ \AESFUNC{ShiftRows} \circ \AESFUNC{SubBytes}
         \]
%     Note that
%            \VERB{vcipherlast} ~\cite[Page 304]{POWER:18}
%     supports 
%     the $Nr$-th round;
%     additional instructions are provided to 
%     support
%     decryption
%     (i.e., \VERB{vncipher}    ~\cite[Page 305]{POWER:18}
%            and
%            \VERB{vncipherlast}~\cite[Page 305]{POWER:18})
%     and
%     key expansion
%     (i.e., \VERB{vsbox}       ~\cite[Page 305]{POWER:18}).

\noindent
{\bf ARM}
      introduced support for AES in 
      ARMv8-A
      per~\cite[Section A2.3]{ARMv8-A:20}.
      Instructions use a
          destructive $2$-address ($1$ source, $1$ source/destination)  
      format,
      and operate on data housed in the pre-existing
      vector 
      register file, implying $R = 128$.
      AES is implemented by
      1) adopting a 
          fully-packed
         representation of state and round key matrices,
         then
      2) using
             \VERB{AESE}  ~\cite[Section C7.2.8 ]{ARMv8-A:20}
             and
             \VERB{AESMC} ~\cite[Section C7.2.10]{ARMv8-A:20}
         to construct a round implementation as
         \[
         \VERB{AESMC} \circ \VERB{AESE} \mapsto \AESFUNC{MixColumns} \circ ( \AESFUNC{SubBytes} \circ \AESFUNC{ShiftRows} \circ \AESFUNC{AddRoundKey} ) ,
         \]
%         where the alternative round definition from 
%         \REFSEC{sec:bg:aes_spec} 
%         is assumed to cater for the order of application.
%     Note that
%     additional instructions are provided to 
%     support
%     decryption
%     (i.e., \VERB{AESD}  ~\cite[Section C7.2.7 ]{ARMv8-A:20}
%            and
%            \VERB{AESIMC}~\cite[Section C7.2.9 ]{ARMv8-A:20}),
%     but none are required to 
%     support
%     the $Nr$-th round:
%     \VERB{AESE} obviously lacks \AESFUNC{MixColumns}, and the post-whitening 
%     step is naturally supported via XOR. 

\noindent
{\bf Oracle}
      introduced support for AES in 
      SPARC 
      per~\cite[Sections 7.3+7.4]{SPARC:16}.
      Instructions use a
      non-destructive $4$-address ($3$ source, $1$        destination)
      format,
      and operate on data housed in the pre-existing
      general-purpose
      register file, implying $R =  64$.
      AES is implemented by
      1) using a 
         column-packed
         representation of state and round key matrices,
         then
      2) using
             \VERB{AES_EROUND01}     ~\cite[Page 109]{SPARC:16}
             and
             \VERB{AES_EROUND23}     ~\cite[Page 109]{SPARC:16}
         to construct a round implementation as
         \[
         ( \VERB{AES_EROUND01};\VERB{AES_EROUND23} ) \mapsto \AESFUNC{AddRoundKey} \circ \AESFUNC{MixColumns} \circ \AESFUNC{ShiftRows} \circ \AESFUNC{SubBytes} 
         \]
         in two steps:
         the first  step processes columns $0$ and $1$ via \VERB{AES_EROUND01}
         whereas
         the second step processes columns $2$ and $3$ via \VERB{AES_EROUND23}.
%     Note that
%            \VERB{AES_EROUND01_LAST}~\cite[Page 109]{SPARC:16}
%            and
%            \VERB{AES_EROUND23_LAST}~\cite[Page 109]{SPARC:16}
%     support 
%     the $Nr$-th round;
%     additional instructions are provided to 
%     support
%     decryption
%     (i.e., \VERB{AES_DROUND01}     ~\cite[Page 109]{SPARC:16},
%            \VERB{AES_DROUND23}     ~\cite[Page 109]{SPARC:16},
%            \VERB{AES_DROUND01_LAST}~\cite[Page 109]{SPARC:16},
%            and
%            \VERB{AES_DROUND23_LAST}~\cite[Page 109]{SPARC:16})
%     and
%     key expansion
%     (i.e., \VERB{AES_KEXPAND0}     ~\cite[Page 112]{SPARC:16},
%            \VERB{AES_KEXPAND1}     ~\cite[Page 109]{SPARC:16},
%            and
%            \VERB{AES_KEXPAND2}     ~\cite[Page 112]{SPARC:16}).

% -----------------------------------------------------------------------------

\subsubsection{Non-standard, academia-specified ISEs}

% workload-agnostic

      Burke et al.~\cite{BurMcDAus:00}
      propose 
      a workload-agnostic ISE
      based on workload characterisation for the
      DEC Alpha architecture \cite{alpha2014alpha}.
      Per~\cite{BurMcDAus:00}, pertinent examples
      for AES
      include
      a) \VERB{ROL}
         and
         \VERB{ROR},
         which perform
         left- and right-rotate,
         and
      b) \VERB{SBOX},
         which 
         extracts elements to form look-up table offsets.
         In one configuration,
         the resulting memory accesses are supported by a
         set of special-purpose ``S-box caches''.

      Fiskiran and Lee~\cite{FisLee:05}
      propose 
      a workload-agnostic ISE
      that employs a so-called
      Parallel Table Lookup Module (PTLU) for a ``{\em RISC like}''
      instruction set.
      For AES, 
      this accelerates implementations based on T-tables 
      by affording an addressing mode that
      a) integrates 
         extraction of elements to form look-up table offsets,
         and
      b) performs the associated table look-ups in parallel, supported by
         a dedicated scratch-pad memory.

      Biham et al.~\cite[Page 232]{BihAndKnu:98}
      propose (in theory)
      and
      Grabher et al.~\cite{GraGroPag:08}
      explore  (in practice)
      a workload-agnostic ISE
      that supports bit-sliced implementations for their custom
      CRISP (``{\em RISC like}'') architecture.
      The ISE allows computation using 
      {\em configurable} $4$-input, $2$-output 
      Boolean functions, vs. 
      {\em fixed}        $2$-input, $1$-output alternatives such as NOT, AND, OR, and XOR.
      Sequences of native Boolean instructions, which dominate bit-sliced
      implementations, can thereby be ``compressed'' into use of the ISE.
      Doing so improves both latency and footprint.
      \cite[Section 4]{GraGroPag:08} details the application to AES.

% workload-specific

      Nadehara et al.~\cite{NadIkeKur:04} 
      propose 
      a workload-specific ISE
       that could be described as 
      ``hardware-assisted T-tables'':
      observing that $\forall x, i \neq j$, $T_i[ x ]$ is a rotation of
      $T_j[ x ]$, they support on-the-fly computation (vs. via look-up)
      of T-table entries.
      The ISE constitutes a single instruction
      $\VERB{AESENC} \mapsto T_i$,
      supported by a dedicated hardware module
      (see~\cite[Figure 6]{NadIkeKur:04}).
      Instances of \VERB{AESENC}
      1) extract an   input element from a 
         packed  input column
      2) use the input to compute an output element equivalent to a
         look-up from the T-table,
         and
      3) store   the output element into a
         packed output column.
      This approach was reapplied by Saarinen~\cite{Saarinen:20}
      within the context of RISC-V.

      Tillich et al.~\cite{TilGroSze:05}
      propose 
      a workload-specific ISE
       that could be described as 
      ``hardware-assisted S-box'' for the SPARC V8 architecture.
      The ISE constitutes a single instruction
      $\VERB{sbox} \mapsto \AESFUNC{SubBytes}$,
      supported by a dedicated hardware module
      (see~\cite[Figure 1]{TilGroSze:05}).
      Instances of \VERB{sbox}
      1) extract an   input element from a packed  input row or column,
      2) use the input to compute an output element equivalent to a
         look-up from the S-box,
         and
      3)  insert the output element into a packed output row or column.
         Using insert vs. overwrite semantics allows
         \AESFUNC{ShiftRows} to be computed {\em for free}.

      Bertoni et al.~\cite{BBFR:06}
      propose 
      a workload-specific ISE
       that could be described as 
      ``hardware-assisted round functions''.
      The ISE includes
      1) zero-overhead rotation (similar to ARM),
         and
      2) byte- and word-oriented variants of
         $\VERB{SMix} \mapsto \AESFUNC{MixColumn} \circ \AESFUNC{SubBytes}$.
      
      Tillich and Gro{\ss}sch\"{a}dl~\cite{TilGro:06}
      propose 
      a workload-specific ISE
       that could be described as 
      ``hardware-assisted round functions'' for the SPARC V8 architecture.
      The ISE includes
         byte- and word-oriented variants of
         $\VERB  {sbox[4][s|r]} \mapsto \AESFUNC{SubBytes} $
         and
         $\VERB{mixcol[4][s]  } \mapsto \AESFUNC{MixColumn}$;
      per~\cite[Section 4.3]{TilGro:06},
      the most efficient variant allows
         a zero-overhead implementation of \AESFUNC{ShiftRows} to be realised.


% =============================================================================


\paragraph{Security}
\label{sec:bg:aes_impl_sec}
% =============================================================================

While the security of AES against a cryptanalytic attack is defined by
the design, and so is out of scope, {\em implementation} attacks are
of central importance.
An implementation attack focuses on the concrete instance of a construct
rather than the abstract specification.
Countermeasures against such attacks must therefore be
considered alongside implementations they relate to.
Since AES is an important target, a significant body of literature exists
around implementation attacks on it, including both
 active (e.g., fault injection)
or
passive (i.e., side-channel monitoring)
attack techniques.
The latter can be sub-divided into those dependent on
analogue
(power-based~\cite{ManOswPop:07})
or
discrete 
(time-based~\cite{KoeQui:99})
leakage.

Use of ISEs
{\em can} provide some inherent protection against certain attacks.
For example,
ISEs typically yield constant time execution,
preventing some classes of timing or micro-architectural
attack techniques
(see~\cite[Section 4]{Szefer:19} and~\cite[Section 4]{GYCH:18}).
Unfortunately,
use of ISEs also presents some unique challenges.
For example, 
Saab et al. ~\cite{SaaRohHam:16}
discuss power-based attacks on AES-NI; concluding
that naive use of AES-NI yields exploitable information leakage.
Mitigation of such leakage demands the ISE
address instances where the leakage stems from ``inside'' the ISE,
and work with appropriate countermeasures
(e.g., hiding~\cite[Chapter 7]{ManOswPop:07} or masking~\cite[Chapter 10]{ManOswPop:07}).
Tillich et al.~\cite{TilHerMan:07}
consider this problem to an extent, including an ISE-based option in
their investigation of hardened AES implementations. However, the challenge
of developing suitable ISEs is under-studied in general.

% =============================================================================


% =============================================================================

\section{Exploring AES ISEs for RISC-V}
\label{sec:ise}

% -----------------------------------------------------------------------------

\label{sec:ise:design}

% =============================================================================

\REFSEC{sec:bg:aes_impl_ise}
outlined a range of ISE designs, demonstrating a large design space of
options that we {\em could} consider.  To narrow the design space into
those we {\em do} consider, we use the requirements outlined below:

\begin{requirement}\label{req:1}
The ISE must support
1) AES encryption {\em and} decryption,
   and
2) {\em all} parameter sets, i.e., AES-128, AES-192, and AES-256.
Support for 
auxiliary operations, e.g., key schedule, 
is an advantage but not a requirement.
\end{requirement}

\begin{requirement}\label{req:2}
The ISE must align with the wider RISC-V design principles.
This means it should 
favour simple building-block operations,
and
use instruction encodings with at most
$2$ source registers and
$1$ destination register.
This avoids the cost of a general-purpose register file with more than $2$
read ports or $1$ write port.
\end{requirement}

\begin{requirement}\label{req:3}
The ISE must use
the RISC-V general-purpose scalar register file 
to store operands and results, rather than
any vector register file.
This requirement excludes the majority of standard ISEs outlined in 
\REFSEC{sec:bg:aes_impl_ise}.
\end{requirement}

\begin{requirement}\label{req:4}
The ISE must not introduce
special-purpose       architectural state, 
nor rely on
special-purpose micro-architectural state
(e.g., caches or scratch-pad memory).
\end{requirement}

\begin{requirement}\label{req:5}
The ISE must enable data-oblivious execution of AES, preventing
timing attacks based on execution latency
(e.g., stemming from accesses to a pre-computed S-box).
\end{requirement}

\begin{requirement}
The ISE must be efficient, in terms of improvement in execution latency 
per area required: this balances the value in {\em both} metrics vs. an 
exclusive preference for one or the other.
Efficiency wrt. 
auxiliary metrics, e.g., memory footprint or instruction encoding points,
is an advantage but not a requirement.
\end{requirement}

\noindent
Overall, the requirements combine to intentionally target the ISE at 
 low(er)-end,
resource-constrained (e.g., embedded) platforms.  
We view such a focus as reasonable, because existing work on adding
cryptographic support to the
standard 
vector extension ~\cite[Section 21]{RV:ISA:I:19}
already caters for
high(er)-end
alternatives.

We arrive at five ISE variants using the requirements, the description of 
which is split into
an 
intuitive 
description in the following \SEC[s]
and
a
technical
description
(e.g., a list of instructions and their semantics)
in an associated \APPX.

% =============================================================================


\subsection{Variant 1 (\ISE{1}): \AESFUNC{SubBytes} $+$ \AESFUNC{MixColumn} $+$ explicit \AESFUNC{ShiftRows}}
\label{sec:ise:design:v1}
% =============================================================================

By reproducing~\cite[Section 4.2]{TilGro:06},
\ISE{1}
assumes 
$\RVXLEN = 32$
and adopts a 
column-packed 
representation of state and round key matrices.
As detailed in
\REFFIG{fig:v1:pseudo},
\ISE{1}
adds
$ 4$
instructions ($2$ for encryption, $2$ for decryption).
For example,
\VERB{saes.v1.encs}
applies 
\AESFUNC{SubBytes}  
to elements in   a packed column,
and
\VERB{saes.v1.encm}
applies 
\AESFUNC{MixColumn} 
to               a packed column;
the instruction format for
\VERB{saes.v1.encs}
and
\VERB{saes.v1.encm}
specifies $1$ source and $1$ destination register.
Since 
\VERB{saes.v1.encs}
requires $4$ applications of the S-box, a trade-off between latency and
area is possible st. 
$n$ physical S-box instances are (re)used in $4/n$ cycles
(e.g., $1$ instance in $4$ cycles, or $4$ instances in $1$ cycle).

\REFFIG{fig:v1:round}
demonstrates that use of \ISE{1} to implement AES encryption requires
$47$ instructions per round:
$ 4$ \VERB{lw}           
     instructions to load the round key,
$ 4$ \VERB{xor}           
     instructions to apply \AESFUNC{AddRoundKey},
$ 4$ \VERB{saes.v1.encs}  
     instructions to apply \AESFUNC{SubBytes},
$31$ instructions to apply \AESFUNC{ShiftRows},
and
$ 4$ \VERB{saes.v1.encm}  
     instructions to apply \AESFUNC{MixColumns}.

% =============================================================================

\subsection{Variant 2 (\ISE{2}): \AESFUNC{SubBytes} $+$ \AESFUNC{MixColumn} $+$ implicit \AESFUNC{ShiftRows}}
\label{sec:ise:design:v2}
% =============================================================================

By reproducing~\cite[Section 4.3]{TilGro:06},
\ISE{2}
assumes 
$w = 32$
and adopts a 
column-packed 
representation of state and round key matrices.

As detailed in
\REFFIG{fig:v2:mnemonics}
and
\REFFIG{fig:v2:pseudo},
it adds
$ 4$
instructions ($2$ for encryption, $2$ for decryption).
For example
\VERB{saes.v2.encs}
\VERB{saes.v2.encs}
 applies \AESFUNC{SubBytes}  to elements in   a packed column,
and
\VERB{saes.v2.encm}
 applies \AESFUNC{MixColumn} to               a packed column (which is optionally rotated);
the instruction format for
\VERB{saes.v2.encs}
and
\VERB{saes.v2.encm}
uses 
includes $1$ source and $1$ destination register address.
\ISE{2} improves \ISE{1} by applying \AESFUNC{ShiftRows} 
{\em implicitly}:
this is possible by careful indexing of elements in source and destination
columns during application of \AESFUNC{SubBytes} and \AESFUNC{MixColumns},
and also permits
\VERB{saes.v2.encs}
to be used within the key schedule.
The same trade-off is possible as in \ISE{1}, whereby
$n$ physical S-box instances are (re)used in $4/n$ cycles
(e.g., $1$ instance in $4$ cycles, or $4$ instances in $1$ cycle).

\REFFIG{fig:v2:round}
demonstrates that use of \ISE{2} to implement AES encryption requires
$16$
instructions per round:
$ 4$ \VERB{saes.v1.encs}  instructions to apply \AESFUNC{SubBytes},
$ 4$ \VERB{saes.v1.encm}  instructions to apply \AESFUNC{MixColumns},
$ 4$ \VERB{ lw}           instructions to load the round key,
and
$ 4$ \VERB{xor}           instructions to apply \AESFUNC{AddRoundKey}.
In the $Nr$-th round, which omits \AESFUNC{MixColumns},
\AESFUNC{ShiftRows} must be applied
{\em explicitly}
using additional 
$12$
instructions.

% =============================================================================

\subsection{Variant 3 (\ISE{3}): hardware-assisted T-tables}
\label{sec:ise:design:v3}
% =============================================================================

\REFSEC{sec:pseudo:v3} shows the mnemonics and pseudo-code functions
for \ISE{3}.
These instructions are based on
\cite{NadIkeKur:04,BBFR:06} and \cite{Saarinen:20},
which implement a T-tables based representation of AES \cite{DaeRij:02}.
The AES state is stored column-wise in four $32$-bit words, and
each instruction selects a single byte of {\tt rs2} to operate on
using the $2$-bit {\tt bs} immediate.
This byte is used as the input to a standard T-table lookup operation,
but the table entry is calculated in hardware.
\REFFIG{fig:design:fu_block:v3} shows the data-path for these instructions.
The result of the T-table lookup is then XOR'ed with {\tt rs1} to
accumulate the results of the round transformation.

These instructions require only one S-box instantiation,
which is a clear advantage in resource constrained applications.
While the previous designs could be implemented with a single S-box, they
would require additional temporary registers and evaluation over multiple
cycles.

We also note that \cite{Saarinen:20} improves on \cite{BBFR:06}
by using an extra source register and allowing the \AESFUNC{AddRoundKey} step to be
performed implicitly, thus saving four instructions per round.

A single encryption round using this variant requires
four load-word instructions to fetch the round key and
$16$ {\tt saes.v3.encs[m]} instructions to perform \AESFUNC{AddRoundKey},
\AESFUNC{SubBytes}, \AESFUNC{ShiftRows} and (optionally) \AESFUNC{MixColumns}.
\REFFIG{fig:round:v3} shows an example AES encrypt round function
using this variant.

% =============================================================================

\subsection{Variant 4 (\ISE{4}): $64$-bit data-path}
\label{sec:ise:design:v4}
% =============================================================================

\ISE{4}
is based on SPARC~\cite[Page 109]{SPARC:16}; it
assumes 
$w = 64$
and adopts a 
{\em double}
column-packed 
representation of state and round key matrices,
i.e., {\em two} columns (or $8$ elements) are packed into a $64$-bit word.
While still adhering to a format that
includes $2$ source and $1$ destination register address,
a single instruction can therefore 
1) accept  all  of the current state as  input,
   and
2) produce half of the next    state as output.

SPARC~\cite[Page 109]{SPARC:16}
adds
$ 9$
instructions ($4$ for encryption, $4$ for decryption, and $1$ auxiliary).
For example
\VERB{AES_EROUND01}
and
\VERB{AES_EROUND23}
produce
columns $0$ and $1$
and
columns $2$ and $3$
respectively.
As detailed in
\REFFIG{fig:v4:mnemonics}
and
\REFFIG{fig:v4:pseudo},
\ISE{4}
refines this slightly by 
adding 
$ 7$
instructions ($2$ for encryption, $2$ for decryption, and $3$ auxiliary).
For example
\VERB{saes.v4.encs}
applies
\AESFUNC{SubBytes}, \AESFUNC{ShiftRow}, and \AESFUNC{MixColumn}  
to elements in   a packed column,
but differs from 
\VERB{AES_EROUND01}
and
\VERB{AES_EROUND23},
because
1) it constitutes
   $1$ (vs. $2$)
   instruction,
   which is possible by observing that swapping the inputs allows 
   computation of either 
   columns $0$ and $1$ 
   or 
   columns $2$ and $3$,
   and
2) it uses 
   $2$ (vs. $3$)
   source register addresses, 
   as a result of opting not to include
   \AESFUNC{AddRoundKey}.

\REFFIG{fig:v4:round}
demonstrates that use of \ISE{4} to implement AES encryption requires
$ 6$ instructions per round:
$ 2$ \VERB{ld}           
     instructions to load the round key,
$ 2$ \VERB{xor}           
     instructions to apply \AESFUNC{AddRoundKey},
$ 2$ \VERB{saes.v4.encsm}  
     instructions to apply \AESFUNC{SubBytes}, \AESFUNC{ShiftRows}, and \AESFUNC{MixColumns}.
In the $Nr$-th round, which omits \AESFUNC{MixColumns},
     \VERB{saes.v4.encsm}
is replaced by 
     \VERB{saes.v4.encs}.

% =============================================================================

\subsection{Variant 5 (\ISE{5}): quadrant-packed}
\label{sec:ise:design:v5}
% =============================================================================

\REFSEC{sec:pseudo:v5} shows the mnemonics and pseudo-code functions
for variant \ISE{5}.
These instructions use a {\em tiled} approach to representing the
AES state.
Figure ({\bf TODO}) shows how the traditional column-wise representation
of AES is changed such that each {\em quadrant} of the 16-byte state
is kept in a single $32$-bit register.

\[
\begin{tikzpicture}
\matrix [matrix of math nodes,right delimiter={\rbrack},left delimiter={\lbrack}] (S) {
  \AESRND{s}{r}_{0,0} & \AESRND{s}{r}_{0,1} & \AESRND{s}{r}_{0,2} & \AESRND{s}{r}_{0,3} \\
  \AESRND{s}{r}_{1,0} & \AESRND{s}{r}_{1,1} & \AESRND{s}{r}_{1,2} & \AESRND{s}{r}_{1,3} \\
  \AESRND{s}{r}_{2,0} & \AESRND{s}{r}_{2,1} & \AESRND{s}{r}_{2,2} & \AESRND{s}{r}_{2,3} \\
  \AESRND{s}{r}_{3,0} & \AESRND{s}{r}_{3,1} & \AESRND{s}{r}_{3,2} & \AESRND{s}{r}_{3,3} \\
} ;

\node [inner sep={-2pt},fit=(S-1-1) (S-2-2),fill={red},   fill opacity={0.2}] {} ;
\node [inner sep={-2pt},fit=(S-1-3) (S-2-4),fill={green}, fill opacity={0.2}] {} ;
\node [inner sep={-2pt},fit=(S-3-1) (S-4-2),fill={blue},  fill opacity={0.2}] {} ;
\node [inner sep={-2pt},fit=(S-3-3) (S-4-4),fill={orange},fill opacity={0.2}] {} ;

\node at ([xshift={-0.25cm}] S.west) [anchor={east}] {$\AESRND{s}{r} = $} ;
\end{tikzpicture}
\]

We can now compute the next round state of any quadrant by sourcing
only two other quadrants (registers) at a time, thus keeping within
the $2$-read-$1$-write constraint.

The state matrix and must be re-arranged before and after applying
the round functions, which adds a small overhead to this variant.
Similarly, the KeySchedule words must also be re-arranged to allow
\AESFUNC{AddRoundKey} to be performed efficiently.
This can be done as a post-processing step in the key expansion.

A single encryption round for this variant requires
four load-word instructions to fetch the round key,
four {\tt xor} instructions to perform \AESFUNC{AddRoundKey},
four {\tt saes.v5.ersub.[lo|hi]} instructions to compute
    \AESFUNC{SubBytes}, \AESFUNC{ShiftRows} for each quadrant
and
four {\tt saes.v5.emix} instructions to compute \AESFUNC{MixColumns} for each
quadrant.
This would make it equivalent to variant 2, however we must also
account for the effort spent packing and un-packing the AES
state into the quadrant representation.
For the base ISA, this would take $12$ instructions to (un-)pack the state.
We note that if the {\tt pack[h]} instructions from the draft
Bit-manipulation extension were included, then packing and unpacking
would be reduced to four instructions.
All packing and un-packing occurs outside the performance critical
loop sections.
\REFFIG{fig:round:v5} shows an example AES encrypt round function
using this variant.

% =============================================================================



% -----------------------------------------------------------------------------

\subsection{Implementation}
\label{sec:ise:imp}
% =============================================================================

The evaluation of each ISE considers two different RISC-V compliant base
micro-architectures, which constitute two different host cores:

\begin{itemize}
\item The \CORE{2}\footnote{%
        \ifbool{anonymous}{Details of this core have been anonymised to comply with the TCHES submission guidelines.}{\url{https://github.com/scarv/scarv}}
      } core 
      supports the 
      RV32IMC 
      instruction set, i.e.,
      the 
             $32$-bit~\cite[Section 2]{RV:ISA:I:19} 
      base integer ISA plus 
      standard 
      Multiplication ~\cite[Section  7]{RV:ISA:I:19}
      and
      Compressed ~\cite[Section 16]{RV:ISA:I:19}
      extensions.
      Per the block diagram shown in~\REFFIG{fig:core:2:normal},
      the core 
      executes instructions using a $5$-stage, in-order pipeline.
      No branch prediction is supported.
      There are two memory interfaces for instruction fetch and data memory
      accesses.
      No instruction or data caches are supported.
      The core implements various performance counters,
      and
      elements of the
      RISC-V Privileged Resource Architecture (PRA)~\cite[Chapter 3]{RV:ISA:II:19}
      related to exception and interrupt handling.

\item The \CORE{1}~\cite{rocket:16} 
        core
      executes instructions using a $5$-stage, in-order pipeline
      which is highly configurable.
      We take advantage of this, considering two variants whose
      exact configuration is outlined in
      \REFFIG{fig:rocket:32} 
      and 
      \REFFIG{fig:rocket:64}:
      the variants represent single $32$-bit and $64$-bit cores respectively,
      and so
      support  the 
      RV32IMC 
      (resp. RV64IMC)
      instruction set, i.e.,
      the 
             $32$-bit~\cite[Section 2]{RV:ISA:I:19} 
      (resp. $64$-bit~\cite[Section 5]{RV:ISA:I:19})
      base integer ISA plus 
      standard 
      Multiplication ~\cite[Section  7]{RV:ISA:I:19}
      and
      Compressed ~\cite[Section 16]{RV:ISA:I:19}
      extensions.
      Each variant is configured to support
      an instruction cache, 
      a  data        cache,
      and
      a  branch prediction mechanism,
      but 
      no floating-point support.

\end{itemize}

\noindent
To support each ISE, two modifications were made to each host core:
the instruction decoder was modified to support
operand selection
and
an AES Functional Unit (AES-FU) was added to support execution of
ISE instructions.
The \CORE{2} core integrates the AES-FU directly into the pipeline,
while
the \CORE{1} core accesses the AES-FU via the
Rocket Custom Coprocessor (RoCC)~\cite[Section 4]{rocket:16}
interface.
Since \REFREQ{req:2}
(each instruction uses at most $2$ source and $1$ destination register)
is fulfilled,
neither micro-architecture required further structural alteration.
A synthesis-time parameter was used to switch between different 
ISEs.

% =============================================================================


% -----------------------------------------------------------------------------

\subsection{Evaluation}

\paragraph{Hardware}
\label{sec:ise:eval:hw}
% =============================================================================

Each ISE variant was evaluated on the host cores
described in \REFSEC{sec:design}.
The 32-bit designs (V1,V2,V3,V5) were implemented on both the
\CORE{1} and \CORE{2} cores.
The 64-bit design (v4) was only evaluated on the 64-bit configuration
of the \CORE{1} core.
Table \ref{tab:eval:hw} shows the hardware implementation costs.

For variants 1, 2 and 5, two implementations are evaluated.
The {\em Size} optimised implementations instantiate only a single
Forward/Inverse S-box circuit and take multiple cycles
to produce a result.
The {\em Latency} optimised implementations instantiate $4$ S-box
\AESFUNC{MixColumn} circuits to produce their results in a single processor 
clock cycle.

The {\em Size} columns of Table \ref{tab:eval:hw} 
record the number of NAND2 equivalent gates of each variant,
instantiated independently from any wider system.
The LTP column gives the Longest Topological Path of the synthesised
functional unit circuit from input to combinatorial output.
The \CORE{2} Size column gives the size in NAND2 equivalent gates of the
\CORE{2}, with the AES functional units integrated.
The ``Baseline'' row gives the size of the core without any of the
ISEs integrated.
We found that none of the proposed ISEs affected the critical
path of the \CORE{2} or \CORE{1} cores.


% =============================================================================


\paragraph{Software}
\label{sec:ise:eval:sw}
% =============================================================================

We evaluated each ISE variant by constructing an associated implementation
of AES (recalling this means AES-128), including
$\ALG{Enc}$
and
$\ALG{Dec}$,
{\em plus}
$\ALG{Enc-KeyExp}$
and
$\ALG{Dec-KeyExp}$;
a set of reference, {\em non}-ISE 
(e.g., T-table) 
implementations were used as a baseline.
The variants which assume  $w = 32$
(\ISE{1}, \ISE{2},     \ISE{3}, and \ISE{5})
uses an   rolled strategy wrt. loops:
 \ISE{1}, \ISE{2},              and \ISE{5}
use  $1$ round  per-iteration,
whereas
                       \ISE{3}
uses $2$ rounds per-iteration
to avoid needless register move operations.
The variant  which assumes $w = 64$
(\ISE{4})
uses an unrolled strategy.
In all cases the state is aligned\footnote{%
RISC-V does not mandate support for misaligned loads and stores, so
aligning the state this way ensures the best performance across all
cores.
} naturally, meaning any input (resp. output) can be loaded (resp. stored) 
using 
$4$ \VERB{lw} instructions on a $32$-bit core
or
$2$ \VERB{ld} instructions on a $64$-bit core.

\REFTAB{tab:eval:sw:size} 
records
memory footprint (i.e., code footprint, and static data footprint)
associated with the software implementations.
Note that entries for $\ALG{Dec-KeyExp}$ marked with $\star$ indicate that 
the equivalent inverse cipher construction~ \cite[Section 5.3.5]{FIPS:197}
is used: this allows $\ALG{Dec-KeyExp}$ to
1) call $\ALG{Enc-KeyExp}$,
   then
2) perform some additional post processing,
with the quoted footprint therefore reflecting the latter only.  
%Where the entry is empty
%$\ALG{Enc-KeyExp} = \ALG{Dec-KeyExp}$,
%meaning there is no additional implementation and so no overhead.

\REFTAB{tab:eval:sw:perf:scarv}
and
\REFTAB{tab:eval:sw:perf:rocket}
record
instruction (i.e., iret) cycle count
associated with the software implementations,
as executed on the \CORE{2} and \CORE{1} cores respectively.

% =============================================================================


\paragraph{Discussion}
\label{sec:ise:eval:discuss}
% =============================================================================

\REFTAB{tab:eval:hw}
demonstrates that all ISE variants
imply a modest area overhead relative to an associated core.
The RV32 \CORE{1} results are not listed, as the ISE overhead compared to
the area of a synthesised Rocket Tile with caches was less than $1\%$ in all
cases.
\REFTAB{tab:eval:sw:size},
demonstrates that all ISE variants
imply a similar, low memory (i.e., both code and static data) footprint.
Beyond this, and per 
\REFSEC{sec:ise:design},
the primary metric of interest is
improvement in execution latency per area:
this metric draws on data from
\REFTAB{tab:eval:hw}
plus either
\REFTAB{tab:eval:sw:perf:2}
or
\REFTAB{tab:eval:sw:perf:1}
for the \CORE{2} or \CORE{1} core respectively,
and, for each variant, is computed by dividing the improvement in execution 
latency (relative to the T-table baseline) by the normalised area (i.e., the 
ISE area column of \REFTAB{tab:eval:hw}).  We deliberately omit the area of
the host core, because this fixed overhead will always dominate: it detracts 
from the comparison between ISEs themselves, therefore.

\REFTAB{tab:eval:results} 
captures the results for the \CORE{1} core, although the same conclusion can 
be drawn for the \CORE{2} core.  Qualitatively, we place more of a weight on 
\ALG{Enc} 
and 
\ALG{Dec} 
vs.
\ALG{Enc-KeyExp} 
and 
\ALG{Dec-KeyExp},
because
 (few) invocations of the latter 
are usually amortised by 
(many) invocations of the former.
For a $32$-bit core, our conclusion is that
\ISE{3} 
is the best option;
for a $64$-bit core,
\ISE{4} 
is the best option, which is somewhat obvious because it specifically makes
use of the wider data-path.
With reference to
\REFTAB{tab:eval:sw:perf:1}, 
note that the number of cycles per instruction executed is relatively low.
This fact stems from use of the ROCC interface, in that forwarding of the 
result from an ISE instruction (that uses the ROCC) incurs an overhead vs. 
an ISE instruction; fine-grained integration of the AES-FU could therefore
incrementally improve the results.

% -----------------------------------------------------------------------------

\begin{adjustbox}{center,caption={Hardware implementation metrics 
                                  (e.g., area and LTP)
                                  for each ISE variant.
                                 },label={tab:eval:hw},float={table}[!p]}
\centering
\begin{tabular}{|c|c@{\;}c|rr|rr|}
\hline
  \multicolumn{1}{|c|}{ISA}
& \multicolumn{1}{ c }{Variant}
& \multicolumn{1}{ c|}{(Goal)}
& \multicolumn{1}{ c|}{             ISE area}
& \multicolumn{1}{ c|}{             ISE LTP }
& \multicolumn{2}{ c|}{\CORE{2} $+$ ISE area}
\\
\hline
\hline
 RV32IMC &          &     &$          $&$          $&$     37375 $&$     (1.00\times) $ \\
 RV32IMC & \ISE{1}  & (L) &$     3472 $&${\bf   19}$&$     41723 $&$     (1.12\times) $ \\
 RV32IMC & \ISE{1}  & (A) &$     2174 $&$       22 $&$     40161 $&$     (1.07\times) $ \\
 RV32IMC & \ISE{2}  & (L) &$     3547 $&${\bf   19}$&$     41199 $&$     (1.10\times) $ \\
 RV32IMC & \ISE{2}  & (A) &$     1381 $&$       21 $&$     38885 $&$     (1.04\times) $ \\
 RV32IMC & \ISE{3}  &     &${\bf 1157}$&$       30 $&${\bf 38610}$&${\bf (1.03\times)}$ \\
 RV32IMC & \ISE{5}  & (L) &$     4121 $&$       22 $&$     42070 $&$     (1.13\times) $ \\
 RV32IMC & \ISE{5}  & (A) &$     1927 $&$       23 $&$     39251 $&$     (1.05\times) $ \\
\hline
\hline
  \multicolumn{1}{|c|}{ISA}
& \multicolumn{1}{ c }{Variant}
& \multicolumn{1}{ c|}{(Goal)}
& \multicolumn{1}{ c|}{             ISE area}
& \multicolumn{1}{ c|}{             ISE LTP }
& \multicolumn{2}{ c|}{\CORE{1} $+$ ISE area}
\\
\hline
 RV64IMC &          &     &$          $&$          $&$   3717607 $&$     (1.000\times)$ \\
 RV64IMC & \ISE{4}  &     &$     8312 $&$       27 $&$   3733786 $&$     (1.004\times)$ \\
\hline
\end{tabular}
\end{adjustbox}

\begin{adjustbox}{center,caption={Software implementation metrics 
                                  (i.e., memory footprint measured in bytes)
                                  for each ISE variant.
                                 },label={tab:eval:sw:size},float={table}[!p]}
\centering
\begin{tabular}{|c|c|r|r|r|r|r|}
\hline
  \multicolumn{1}{|c|}{ISA}
& \multicolumn{1}{ c|}{Variant}
& \multicolumn{1}{ c|}{$\ALG{Enc}$}
& \multicolumn{1}{ c|}{$\ALG{Dec}$}
& \multicolumn{1}{ c|}{$\ALG{Enc-KeyExp}$}
& \multicolumn{1}{ c|}{$\ALG{Dec-KeyExp}$}
& \multicolumn{1}{ c|}{.data} 
\\
\hline
\hline
%RV32IMC & Byte    &$         $&$         $&$     312 $&$       0 $&$ 522$ \\
 RV32IMC & T-table &$     804 $&$     804 $&$     154 $&$     174 $&$5120$ \\
 RV32IMC & \ISE{1} &$     424 $&$     424 $&${\bf  68}$&$       0 $&$  10$ \\
 RV32IMC & \ISE{2} &${\bf 234}$&${\bf 238}$&${\bf  68}$&$      62 $&$  10$ \\
 RV32IMC & \ISE{3} &$     290 $&$     290 $&$      86 $&$      64 $&$  10$ \\
 RV32IMC & \ISE{5} &$     266 $&$     278 $&$     290 $&$       0 $&$  10$ \\
\hline
 RV64IMC & \ISE{4} &$     268 $&$     268 $&$     168 $&$     100 $&$   0$ \\
\hline
\end{tabular}
\end{adjustbox}

\begin{adjustbox}{center,caption={Execution metrics
                                  for each ISE variant on the \CORE{2} core.
                                  Note that the $64$-bit \ISE{4} is absent, since there is no $64$-bit \CORE{2} core.
                                 },label={tab:eval:sw:perf:2},float={table}[!p]}
\centering
\begin{tabular}{|c|c@{\;}c|rr|rr|rr|rr|}
\hline
  \multicolumn{1}{|c|}{ISA}
& \multicolumn{1}{ c }{Variant}
& \multicolumn{1}{ c|}{(Goal)}
& \multicolumn{2}{ c|}{$\ALG{Enc}$}
& \multicolumn{2}{ c|}{$\ALG{Dec}$}
& \multicolumn{2}{ c|}{$\ALG{Enc-KeyExp}$}
& \multicolumn{2}{ c|}{$\ALG{Dec-KeyExp}$}
\\
\cline{4-11}
&
&
& \multicolumn{1}{ c|}{iret}
& \multicolumn{1}{ c|}{cycles}
& \multicolumn{1}{ c|}{iret}
& \multicolumn{1}{ c|}{cycles}
& \multicolumn{1}{ c|}{iret}
& \multicolumn{1}{ c|}{cycles}
& \multicolumn{1}{ c|}{iret}
& \multicolumn{1}{ c|}{cycles}
\\
\hline
\hline
%RV32IMC & Byte    &     &$          $&$          $&$          $&$          $&$          $&$          $&$          $&$          $\\
 RV32IMC & T-table &     &$      998 $&$     1076 $&$      998 $&$     1103 $&$      466 $&$      554 $&$     1747 $&$     2346 $\\ 
 RV32IMC & \ISE{1} & (L) &$      518 $&$      593 $&$      518 $&$      607 $&${\bf  198}$&${\bf  291}$&${\bf  204}$&${\bf  310}$\\
 RV32IMC & \ISE{1} & (A) &$      518 $&$      753 $&$      518 $&$      775 $&${\bf  198}$&$      331 $&${\bf  204}$&$      350 $\\
 RV32IMC & \ISE{2} & (L) &${\bf  221}$&$      301 $&${\bf  222}$&$      303 $&${\bf  198}$&$      302 $&$      335 $&$      616 $\\
 RV32IMC & \ISE{2} & (A) &${\bf  221}$&$      538 $&${\bf  222}$&$      540 $&${\bf  198}$&$      332 $&$      335 $&$      754 $\\
 RV32IMC & \ISE{3} &     &$      238 $&${\bf  291}$&$      238 $&${\bf  286}$&$      219 $&$      312 $&$      659 $&$     1118 $\\
 RV32IMC & \ISE{5} & (L) &$      233 $&$      304 $&$      233 $&$      309 $&$      332 $&$      447 $&$      338 $&$      466 $\\
 RV32IMC & \ISE{5} & (A) &$      233 $&$      556 $&$      233 $&$      550 $&$      332 $&$      477 $&$      338 $&$      496 $\\
\hline
\end{tabular}                
\end{adjustbox}

\begin{adjustbox}{center,caption={Execution metrics
                                  for each ISE variant on the \CORE{1} core.
                                  Note that the $64$-bit \ISE{4} uses the $64$-bit \CORE{1} core; all others use the $32$-bit \CORE{1} core.
                                 },label={tab:eval:sw:perf:1},float={table}[!p]}
\centering
\begin{tabular}{|c|c@{\;}c|rr|rr|rr|rr|}
\hline
  \multicolumn{1}{|c|}{ISA}
& \multicolumn{1}{ c }{Variant}
& \multicolumn{1}{ c|}{(Goal)}
& \multicolumn{2}{ c|}{$\ALG{Enc}$}
& \multicolumn{2}{ c|}{$\ALG{Dec}$}
& \multicolumn{2}{ c|}{$\ALG{Enc-KeyExp}$}
& \multicolumn{2}{ c|}{$\ALG{Dec-KeyExp}$}
\\
\cline{4-11}
&
&
& \multicolumn{1}{ c|}{iret}
& \multicolumn{1}{ c|}{cycles}
& \multicolumn{1}{ c|}{iret}
& \multicolumn{1}{ c|}{cycles}
& \multicolumn{1}{ c|}{iret}
& \multicolumn{1}{ c|}{cycles}
& \multicolumn{1}{ c|}{iret}
& \multicolumn{1}{ c|}{cycles}
\\
\hline
\hline
%RV32IMC & Byte    &     &$          $&$          $&$          $&$          $&$          $&$          $&$          $&$          $\\
 RV32IMC & T-table &     &$      948 $&$     1143 $&$      949 $&$     1025 $&$      444 $&$      478 $&$     1726 $&$     1977 $\\
 RV32IMC & \ISE{1} & (L) &$      528 $&$      685 $&$      529 $&$      680 $&${\bf  212}$&$      341 $&${\bf  214}$&${\bf  290}$\\
 RV32IMC & \ISE{1} & (A) &$      528 $&$      804 $&$      529 $&$      744 $&${\bf  212}$&$      357 $&${\bf  214}$&$      335 $\\
 RV32IMC & \ISE{2} & (L) &${\bf  231}$&${\bf  359}$&${\bf  233}$&$      368 $&${\bf  212}$&${\bf  315}$&$      350 $&$      508 $\\
 RV32IMC & \ISE{2} & (A) &${\bf  231}$&$      511 $&${\bf  233}$&$      520 $&$      212 $&$      345 $&$      350 $&$      646 $\\
 RV32IMC & \ISE{3} &     &$      253 $&$      445 $&$      254 $&$      445 $&$      233 $&$      470 $&$      674 $&$     2425 $\\
 RV32IMC & \ISE{5} & (L) &$      243 $&$      414 $&$      244 $&${\bf  319}$&$      346 $&$      427 $&$      348 $&$      424 $\\
 RV32IMC & \ISE{5} & (A) &$      243 $&$      585 $&$      244 $&$      543 $&$      346 $&$      504 $&$      348 $&$      454 $\\
\hline
 RV64IMC & \ISE{4} &     &$       81 $&$      119 $&$       82 $&$      125 $&$       66 $&$      204 $&$      136 $&$      306 $\\
\hline
\end{tabular}
\end{adjustbox}

% -----------------------------------------------------------------------------

\begin{adjustbox}{center,caption={Comparison of improvement per unit-area 
                                  for each ISE variant. 
                                 },label={tab:eval:results},float={table}[!t]}
\centering
\begin{tabular}{|c|c@{\;}c|rr|rr|rr|rr|}
\hline
  \multicolumn{1}{|c|}{ISA}
& \multicolumn{1}{ c }{Variant}
& \multicolumn{1}{ c|}{(Goal)}
& \multicolumn{2}{ c|}{$\ALG{Enc}$}
& \multicolumn{2}{ c|}{$\ALG{Dec}$}
& \multicolumn{2}{ c|}{$\ALG{Enc-KeyExp}$}
& \multicolumn{2}{ c|}{$\ALG{Dec-KeyExp}$}
\\
\cline{4-11}
&
&
& \multicolumn{1}{ c|}{iret}
& \multicolumn{1}{ c|}{cycles}
& \multicolumn{1}{ c|}{iret}
& \multicolumn{1}{ c|}{cycles}
& \multicolumn{1}{ c|}{iret}
& \multicolumn{1}{ c|}{cycles}
& \multicolumn{1}{ c|}{iret}
& \multicolumn{1}{ c|}{cycles}
\\
\hline
\hline
RV32IMC & \ISE{1} & (L) &$      4.61 $&$      4.34 $&$      4.61 $&$      4.35 $&$      5.39 $&$      6.06 $&$     20.50 $&$     18.12 $ \\
RV32IMC & \ISE{1} & (A) &$      7.37 $&$      5.46 $&$      7.37 $&$      5.44 $&$      8.61 $&$      6.40 $&${\bf 32.74}$&${\bf 25.63}$ \\
RV32IMC & \ISE{2} & (L) &$     10.58 $&$      8.38 $&$     10.53 $&$      8.53 $&$      5.28 $&$      4.30 $&$     12.22 $&$      8.92 $ \\
RV32IMC & \ISE{2} & (A) &$     27.18 $&$     12.04 $&$     27.06 $&$     12.29 $&$     13.56 $&$     10.04 $&$     31.39 $&$     18.73 $ \\
RV32IMC & \ISE{3} &     &${\bf 30.12}$&${\bf 26.56}$&${\bf 30.12}$&${\bf 27.71}$&${\bf 14.63}$&${\bf 12.76}$&$     19.04 $&$     15.08 $ \\
RV32IMC & \ISE{5} & (L) &$      8.64 $&$      7.14 $&$      8.64 $&$      7.20 $&$      2.71 $&$      2.50 $&$     10.43 $&$     10.15 $ \\
RV32IMC & \ISE{5} & (A) &$     18.48 $&$      8.35 $&$     17.64 $&$      8.30 $&$      5.79 $&$      5.01 $&$     22.29 $&$     20.40 $ \\
\hline
RV64IMC & \ISE{4} &     &$     12.32 $&$      9.04 $&$     12.17 $&$      8.82 $&$      6.76 $&$      2.72 $&$     12.85 $&$      7.67 $ \\
\hline
\end{tabular}
\end{adjustbox}

% =============================================================================


% =============================================================================

\section{Using ISEs to implement AES-GCM}
\label{sec:gcm}
%\input{tex/body-modes.tex}
% =============================================================================

\begin{table}[p]
\centering
\begin{tabular}{|c|c|c|rrrrrr|}
\hline
ISA    & Karatsuba & Reduce & \VERB{grev}
                            & \VERB{xor}
                            & \VERB{s[lr]li}
                            & \VERB{clmul} 
                            & \VERB{clmulh}
                            & Total \\
\hline
\hline
RV32IB &        no &    mul &$  4$&$ 36$&$  0$&$ 20$&$ 20$&$ 80$ \\
RV32IB &        no &  shift &$  4$&$ 56$&$ 24$&$ 16$&$ 16$&$116$ \\
RV32IB &       yes &    mul &$  4$&$ 52$&$  0$&$ 13$&$ 13$&$ 82$ \\
RV32IB &       yes &  shift &$  4$&$ 72$&$ 24$&$  9$&$  9$&$118$ \\
\hline
RV64IB &        no &    mul &$  2$&$ 10$&$  0$&$  6$&$  6$&$ 24$ \\
RV64IB &        no &  shift &$  2$&$ 20$&$ 12$&$  4$&$  4$&$ 42$ \\
RV64IB &       yes &    mul &$  2$&$ 14$&$  0$&$  5$&$  5$&$ 26$ \\
RV64IB &       yes &  shift &$  2$&$ 24$&$ 12$&$  3$&$  3$&$ 44$ \\
\hline
\end{tabular}
\caption{Instruction counts for multiplication in $\F_{2^{128}}$ as used by \ALG{GHASH}.}
\label{tab:gcm:instrs}
\end{table}

\begin{table}[p]
\centering
\begin{tabular}{|c|c|c|rrrr|}
\hline
ISA    & Karatsuba & Reduce & $1$-cycle       & $2$-cycle       & $3$-cycle       & $6$-cycle       \\
       &           &        & \VERB{clmul[h]} & \VERB{clmul[h]} & \VERB{clmul[h]} & \VERB{clmul[h]} \\
\hline
\hline
RV32IB &        no &    mul &     \bftab  80  &            120  &            160  &            280  \\
RV32IB &        no &  shift &            116  &            148  &            180  &            276  \\
RV32IB &       yes &    mul &             82  &    \bftab  108  &     \bftab 134  &            212  \\
RV32IB &       yes &  shift &            118  &            136  &            154  &     \bftab 208  \\
\hline
RV64IB &        no &    mul &     \bftab  24  &    \bftab   36  &             48  &             84  \\
RV64IB &        no &  shift &             42  &             50  &             58  &             82  \\
RV64IB &       yes &    mul &             26  &    \bftab   36  &     \bftab  46  &             76  \\
RV64IB &       yes &  shift &             44  &             50  &             56  &     \bftab  74  \\
\hline
\end{tabular}
\caption{Modelled cycle counts for multiplication in $\F_{2^{128}}$ as used by \ALG{GHASH}.}
\label{tab:gcm:cycles}
\end{table}

\begin{table}[p]
\centering
\begin{tabular}{|c|l|rr|r|r|}
\hline
  \multicolumn{1}{|c|}{ISA}
& \multicolumn{1}{ c|}{Variant}
& \multicolumn{1}{ c|}{             ISE}
& \multicolumn{1}{ c|}{       ISE      }
& \multicolumn{1}{ c|}{\CORE{2}     CPU}
& \multicolumn{1}{ c|}{\CORE{1}     CPU}
\\
& \multicolumn{1}{ c|}{/ Goal       }
& \multicolumn{1}{ c|}{Area         }
& \multicolumn{1}{ c|}{Latency      }
& \multicolumn{1}{ c|}{$+$ ISE area }
& \multicolumn{1}{ c|}{$+$ ISE area }
\\
\hline
\hline
 RV32IMC & Baseline    &              &            &       37325  ($1.00\times$) &       3501576 ($1.000\times$) \\
 RV32IMC & \ISE{1} (L) &        1605  & \bftab 17  &       39154  ($1.05\times$) &       3506224 ($1.001\times$) \\
 RV32IMC & \ISE{1} (A) &        1038  &        23  &       38561  ($1.05\times$) &       3505695 ($1.001\times$) \\
 RV32IMC & \ISE{2} (L) &        1611  & \bftab 17  &       40337  ($1.03\times$) &       3506729 ($1.001\times$) \\
 RV32IMC & \ISE{2} (A) &         780  &        21  &       38479  ($1.08\times$) &       3505910 ($1.001\times$) \\
 RV32IMC & \ISE{3}     & \bftab  630  &        25  &\bftab 38301  ($1.03\times$) &       3506097 ($1.001\times$) \\
 RV32IMC & \ISE{5} (L) &        1852  &        23  &       40626  ($1.03\times$) &       3507518 ($1.001\times$) \\
 RV32IMC & \ISE{5} (A) &        1048  &        23  &       38749  ($1.09\times$) &       3506816 ($1.001\times$) \\
\hline
\hline
 RV64IMC & Baseline &          &          &  N/A  & 3717607 (1.000$\times$) \\
 RV64IMC & \ISE{4}  &  3790    &    27    &  N/A  & 3728235 (1.003$\times$) \\
\hline
\end{tabular}
\caption{
Hardware implementation metrics for each ISE variant with only encrypt instructions implemented.
Area is measured in NAND2 gate equivalents and latency in gate delays.
}
\label{tab:eval:hw:dec}
\end{table}

% -----------------------------------------------------------------------------

\noindent
The Galois/Counter Mode (GCM) ~\cite{NIST:sp.800.38d}
is a block cipher mode of operation which 
supports authenticated encryption.
AES-GCM refers to an instantiation using AES as the underlying block cipher, 
which is the only case mandated by TLS 1.3~\cite[Section 9.1]{rfc:8446}; the
importance of this construction means GCM and AES are frequently considered 
together from an implementation and evaluation perspective.
The computational core of AES-GCM is formed from two components.
\ALG{GCTR} ~\cite[Section 6.5]{NIST:sp.800.38d}
is responsible for 
    encryption
using AES,
and
\ALG{GHASH}~\cite[Section 6.4]{NIST:sp.800.38d}
is responsible for
authentication.
Having dealt with efficient implementation of AES and hence \ALG{GCTR} in
\REFSEC{sec:ise}, we turn our attention to \ALG{GHASH}.  
Rather than further 
embellish the ISE for AES, we instead focus on re-use of the proposed
standard 
bit-manipulation extension~\cite[Section 17]{RV:ISA:I:19}
(at the time of writing, the draft extension proposal is found in~\cite{riscv:bitmanip:draft}).
This approach is attractive for two reasons.
AES-GCM is a very common construction, but AES is not the only block
cipher which can be used with GCM.
Likewise, AES may not always be used with GCM, so separation of
the two constructs from an instruction set point of view is prudent.

% -----------------------------------------------------------------------------

\paragraph{Implementation.}

\ALG{GHASH}~\cite[Section 6.4]{NIST:sp.800.38d} is a universal hash defined 
over the finite field $\F_{2^{128}}$ constructed as
$
\F_{2}[\IND{x}] / ( \IND{x}^{128} + \IND{x}^{7} + \IND{x}^{2} + \IND{x} + 1 ) .
$
Conversion of the input into the correct endianness can be realised using
the 
\VERB{grev} (or generalised reverse)
instruction,
which can reverse the bits in each byte of an input word:
$4$ (resp. $2$) 
\VERB{grev} 
instructions are therefore required on RV32IB (resp. RV64IB).
Beyond this, operations in $\F_{2^{128}}$ dominate.
Addition       in $\F_{2^{128}}$ 
is equivalent to XOR: thus
$4$ (resp. $2$) 
\VERB{xor} 
instructions are required on RV32IB (resp. RV64IB).
Multiplication in $\F_{2^{128}}$ 
can be split into two steps:
a $( 128 \times 128 )$-bit polynomial multiplication, 
followed by 
a reduction of the $256$-bit result modulo
$
\IND{x}^{128} + \IND{x}^{7} + \IND{x}^{2} + \IND{x} + 1 .
$

The multiplication step 
can be realised using pairs of ``carry-less'' multiplication instructions
\VERB{clmul} and \VERB{clmulh}.
These compute the least significant (resp. most-significant) 
half of a carry-less product (i.e., product over $\F_2$).
Pairs of 
\VERB{clmul} and \VERB{clmulh}
should be scheduled adjacently, allowing capable micro-architectures
to fuse them.
Use of a school book approach 
requires
$16$ (resp. $4$) pairs 
on RV32IB (resp. RV64IB).
Optimisation using the Karatsuba method
requires
$ 9$ (resp. $3$) such pairs 
on RV32IB (resp. RV64IB),
plus some additional \VERB{xor} instructions.

The reduction step
can be implemented in two ways:
a shift-based reduction, made possible by the low Hamming weight of the
primitive polynomial,
or
a multiplication-based reduction, analogous to the Montgomery or Barret
methods.
The most efficient approach depends on the relative execution 
latency of
\VERB{clmul[h]}
vs.
\VERB{xor} and \VERB{s[lr]li}.
Note that the entire \ALG{GHASH} operation, including \VERB{clmul[h]},
{\em must} exhibit data-oblivious execution latency 
(e.g., avoid data-dependent optimisations like early-termination)
to avoid associated side-channel attacks (cf.~\cite{GOPT:09}).

% -----------------------------------------------------------------------------

\paragraph{Discussion.}

\REFTAB{tab:gcm:instrs} 
lists instruction counts for 
multiplication in $\F_{2^{128}}$,
implemented using combinations of the base ISA, and approaches
for the polynomial multiplication and reduction steps.
\REFTAB{tab:gcm:cycles} 
then models the execution latency 
(measured in cycles)
assuming \VERB{grev}, \VERB{xor}, and \VERB{s[lr]li} take $1$ cycle.
Although the model only considers an in-order core in line with those used
in \REFSEC{sec:ise} and is focused on execution latency
(vs. other pertinent metrics, such as code footprint),
there are two obvious conclusions:
if
\VERB{clmul[h]}
has $2$ (or more) times the latency of
\VERB{xor} and \VERB{s[lr]li},
a 
Karatsuba
polynomial multiplication
is preferable.
If
\VERB{clmul[h]}
has $6$ (or more) times the latency of
\VERB{xor} and \VERB{s[lr]li},
a shift-based 
reduction 
is preferable.

We recommend the carry-less multiply instructions
specified in the proposed RISC-V bit-manipulation extension also be included
in the RISC-V cryptography extension.
Implementers would otherwise need to implement (a subset of) the B
extension, potentially adding functionality and cost that is not
necessary.

An important consideration for the GCTR component of GCM is that it only
requires the encryption function for a block cipher.
Given this, we re-evaluate the hardware costs of each ISE, assuming that
only the encryption instructions are implemented.
These results are shown in \REFTAB{tab:eval:hw:dec}.
Compared to the hardware results for encrypt and decrypt being implemented in
\REFTAB{tab:eval:hw:encdec}, the area overhead for all ISE variants
is approximately halved, and there is a small reduction in circuit depth.
For our recommended variants, \ISE{3} and \ISE{4},
the area savings when only encryption instructions are implemented
are $0.46\times$ and $0.54\times$ respectively.
For very constrained devices which have exact functionality
requirements, we believe that making implementation of the decryption
instruction optional could be beneficial.
If these systems {\em do} require AES decryption, it could still be
implemented in software, with a performance and code size similar
to the baseline implementations in
\REFTAB{tab:eval:sw:perf:2}
and
\REFTAB{tab:eval:sw:perf:1}.

% =============================================================================


% =============================================================================

\section{Hardening an AES ISE against DPA attack}
\label{sec:sca}

In embedded, IoT class devices to which an attacker may have
physical access,
Differential Power Analysis (DPA) attacks on cryptographic implementations
\cite{KJJ:99} can be devastating.
DPA attacks rely on the power consumption of an attacked device being
corrolated with secret values (i.e. cryptographic keys) being manipulated.
By observing the power consumption of the device while processing
attacker controlled inputs (plaintexts), one can model the {\em expected}
power consumption based on simple hamming-weight or hamming-distance
calculations of expected instruction results.
The secret value is recovered by correlating the predicted power
consumption with the observed traces.

While ISEs give a notable increase in efficiency, they can also create
attractive targets for DPA attacks.
This stems from there being only one way to sensibly implement
AES using the ISE, thus reducing the number of target variables an
attacker needs to consider.
It is hence important to consider how an implementer might
further extend an cryptographic ISE to secure it against DPA attacks.

Having identified ISE \ISE{3} as a strong standardisation candidate
for embedded $32$-bit RISC-V cores, we now show a possible
way of extending the ISE further to add 
$1$'st order DPA side-channel resistance.

\subsection{Design and implementation}

We base our design on boolean-masking, and represent the secret
key as two boolean masked shares.

An implementation of the AES block encrypt/decrypt function 
using \ISE{3} requires eight registers:
four for the current round state,
four to load the next round key into
and
then accumulate the next round state.
See \REFFIG{fig:v3:round} for an AES round function implementation
using \ISE{3}

Storing shares of each secret variable
in the General Purpose Register (GPR) file is un-reasonable,
requiring drastic modifications to the instruction definitions and
register file to read four registers (two sources, of two shares each) and
write two registers.
This would break the RISC-V $2$-read-$1$-write principle.
Storing corresponding shares in the GPRs is also a security
risk, as they may be accidentally combined due to
careless instruction use, or implicit register accesses by the
CPU micro-architecture.

Instead, we define a new, $8$-element ``Mask Register File'' (MRF).
Each mask register $M_i$ is $w=32$-bits wide, and stores the mask for
one of the GPRs.
We use a fixed mapping between GPRs and mask registers;
not all GPRs have a corresponding mask register.
We use the mapping $\{a0..a3,t0..t4\} \Rightarrow \{m0..m7\}$.

Share $0$ of each secret value is loaded into the GPRs using the
standard RISC-V Load Word ({\tt lw}) instruction.
We define a new Load Mask instruction {\tt lm rd, imm(rs1)} which
loads {\em the mask for GPR {\tt rd}}
(i.e. Share $1$)
from memory into the corresponding MRF entry.
A corresponding Store Mask instruction {\tt sm rs2, imm(rs1)} writes
the mask corresponding to GPR {\tt rs2} to memory.
The {\tt sm} instruction is only used for context switches, and
destructively reads the MRF register value to prevent it being
leaked to other applications running on the same core.\footnote{
    In this case, destructive could mean set to zero (which could
    leak the hamming weight of the mask) or randomising it's value.}
We require the secret values be stored in shared form in memory
(rather than splitting them into shares upon being loaded)
to extend the SCA protection boundary outside the CPU.
Otherwise, the hamming weight of un-masked secret values would be
leaked by memory-hierarchy registers outside the CPU.
Executing an {\tt lm} instruction such that {\tt rd} does not map to
a mask register raises an illegal opcode exception.
Likewise for {\tt sm} and {\tt rs2}.

When an ISE instruction is executed and its GPR source
registers map to an MRF register, both GPR and MRF are
read simultaneously and fed to the AES functional unit.
If any GPR source does not map to an MRF register, we assume that
operand is un-masked and represent the other share as $0$.

Within the AES FU the instruction result is computed entirely in it's
masked representation.
The result shares are then re-masked before being written back to the
GPRs and MRF.
This is necessary, because \ISE{3} instructions are designed
such that {\tt rs1=rd} for all use cases.
Without re-masking, overwriting a source with the result could cause 
$1$'st order hamming-distance leakage.

If the destination GPR has a corresponding
mask register, share $0$ is stored in the GPRs and share $1$ in the MRF.
If the destination GPR does not map to a mask register, the result is written
to the GPR un-masked.
This means that in the final encrypt/decrypt round, we can obtain
the un-masked results without having to store the shares to memory,
load them back and un-mask them.

We used the \CORE{2} core as the basis for our side-channel secure
implementation of \ISE{3}.
\REFFIG{fig:core:2:secure} shows a block diagram of the modifications
made to the core, and which data-paths carry masked data.
To avoid accidental un-masking of the two shares,
Share $1$ is stored in {\em bit-reversed} form in the MRF and pipeline
stage registers.
This means that any accidental multiplexing between pipeline operand
registers causes toggles between non-corresponding bits of each share.
Share $1$ is only un-reversed immediately prior to entering the
AES functional unit, and is re-reversed before exiting it.
Bit-reversal has zero logic gate cost and some minor routing complexity.

While the architectural state stores a $2$-share representation
of the secret material, we use a $3$-share implementation of the
AES S-box.
This was driven by experiments showing 
leakage from a $2$-share design in our FPGA platform.
The additional share is generated by a simple $32$-bit LFSR and added
dynamically by the hardware, and is never visible to the programmer.
This is suitable for a proof of concept (evident in the experimental
results) but would need to be used in conjunction with a true random
number source (E.g. a set of ring-oscillators) in a deployed system.
Only the S-box is implemented using $3$-shares.
Subsequent \AESFUNC{MixColumns} logic is only implemented using $2$ shares.

\subsection{Evaluation}

The modified \CORE{2} core was implemented on a
Sasebo GIII \cite{HKSS:12}
side-channel analysis platform, containing two Xilinx FPGAs:
a Kintex-7 
(model {\tt xc7k160tfbg676})
target
and
a supporting Spartan-6
(model {\tt xc6slx45}).
Only the Kintex-7 was used.
The design was synthesised using Xilinx Vivado 2019.2 with
default synthesis and implementaiton strategies.
The Kintex-7 FPGA uses a 200MHz differential external clock source, which is
transformed into a 50MHz internal clock used by the entire
design.

Trace capture uses a standard pipeline of components:
a MiniCircuits BLK+89 D/C blocker,
an Agilent 8447D amplifier (with a $\SI{100}{\kilo\hertz}$ to $\SI{1.3}{\giga\hertz}$ range, and $\SI{25}{\decibel}$ gain),
and
a  PicoScope 5000 series oscilloscope.
The oscilloscope uses a 250 MHz sample rate, with a 12-bit resolution.
The capture process is coordinated using a laptop.

We performed a generic Test Vector Leakage Assessment (TVLA) \cite{TVLA:13}
flow to evaluate
the effectivness of the side-channel hardened implementation;
using the AES-128 block encrypt function as the target operation.
The un-protected and protected implementation results are shown in
\REFFIG{fig:sca:unprotected} and
\REFFIG{fig:sca:protected} respectivley.
The protected implementaiton is effective at removing $1$'st
order side-channel leakage upto $100$K traces.
The peaks at the beginning and end of \REFFIG{fig:sca:protected}
are caused by the un-masked block input and output data being loaded/stored.

\REFTAB{tab:sca:sw-hw} show the hardware and software overheads.
The ISE Size/LTP rows are inclusive of the S-box Size/LTP rows.
Likewise, the CPU Size rows are inclusive of the ISE Size rows.
The static code size and instruction count overheads are
$\approx 20\%$; considerably less than a non-ISE based software masking
approach.
The hardware overheads are dominated by the increased size of the
S-box (owing to the 3-share design), and the MRF.
Although the overhead to the dedicated
ISE logic is $~4x$, this drops to $~1.2x$ when the entire
CPU sub-system is considered.
Measured against an entire SoC, the overheads are modest.

\begin{table}[]
\centering
\begin{tabular}{|l|r|r|r|}
\hline
Metric  & Un-protected & Protected  & Overhead \\
\hline
\hline
Static Code Size (Bytes)        & 290         & 358    & $1.23\times$   \\
Instructions Executed           & 238         & 287    & $1.21\times$   \\
CPU Clock Cycles                & 291         & 331    & $1.14\times$   \\
\hline
S-box Size (NAND2 Equivalent)   & 554         & 3245   & $5.86\times$   \\
S-box LTP                       & 19          & 22     & $1.16\times$   \\
ISE Size (NAND2 Equivalent)     & 1157        & 4616   & $3.99\times$   \\
ISE LTP                         & 30          & 37     & $1.23\times$   \\
CPU Size (NAND2 Equivalent)     & 38610       & 45141  & $1.16\times$   \\
CPU Size  LUTs                  & 4017        & 4956   & $1.23\times$   \\
CPU Size  FFs                   & 2078        & 2420   & $1.16\times$   \\
FPGA Timing Slack @50MHz        & 8.12ns      & 7.05ns & $0.87\times$   \\
\hline
\end{tabular}
\caption{
Software and hardware overheads for the protected ISE implementation
of AES-128 block encryption.
The ``ISE Size'' row does not include the cost of the mask register file
for the protected implementation.
This is included in the CPU size measurements, since the exact method
of mask delivery and storage is an implmentation option.
}
\label{tab:sca:sw-hw}
\end{table}

\begin{figure}
\centering
\begin{subfigure}[t]{0.95\textwidth}
\centering
\includegraphics[width=\textwidth]{graphs/aes-vanilla-enc-default-ttest.png}
\caption{
    Un-protected implementation TVLA results after $10$K traces.
}
\label{fig:sca:unprotected}
\end{subfigure}
\begin{subfigure}[t]{0.95\textwidth}
\centering
\includegraphics[width=\textwidth]{graphs/aes-secure-enc-default-ttest.png}
\caption{
    Side-channel protected implementation TVLA results after $100$K traces.
}
\label{fig:sca:protected}
\end{subfigure}
\caption{
TVLA results for the baseline and protected implementations.
The blue trace is the absoloute result of the TVLA evaluation, the green
trace is the average power consumption for each TVLA trace set.
}
\end{figure}


% =============================================================================

\section{Conclusion}
\label{sec:outro}
% =============================================================================

We have surveyed and evaluated several ISEs for accelerating
the AES block cipher in the context of RISC-V.
We find that \ISE{3} has clear advantages for embedded class
$32$-bit cores, while \ISE{4} is a natural choice for taking
advantage of the wider data-path on $64$-bit systems.
For the $32$-bit case, we have also shown that with reasonable additional
hardware, it is possible to create a $1$'st order masked implementation with
modest performance and resource overheads.

% =============================================================================


% ============================================================================

\ifbool{anonymous}{}{%
\section*{Acknowledgements}

This work was undertaken as part of the ongoing standardisation of
RISC-V. We are grateful to {\em all} members of
the Cryptographic Extensions Task Group
who contributed to related discussions.
The opinions expressed in this paper are the author's alone, not
of their respective employers or the RISC-V International Foundation.
The RISC-V cryptography extension is in the process of being standardised
at the time of writing. The purpose of this work is to support that process.

We would like to thank the anonymous reviewers for their helpful and 
constructive comments.

This work has been supported in part by EPSRC via grant EP/R012288/1, 
under the RISE (\url{http://www.ukrise.org}) programme.
}%

% =============================================================================

\bibliographystyle{alpha}
\bibliography{paper}

% =============================================================================

\appendix

\clearpage
\section{\ISE{1}:       additional technical detail}
\label{sec:pseudo:v1}
% =============================================================================

\vspace*{\fill}

\begin{figure}[!h]
\begin{lstlisting}[language=pseudo,style=block]
saes.v1.encs rd, rs1 : v1.SubBytes(rd, rs1, fwd=1)
saes.v1.decs rd, rs1 : v1.SubBytes(rd, rs1, fwd=0)
saes.v1.encm rd, rs1 : v1.MixColumn(rd, rs1, fwd=1)
saes.v1.decm rd, rs1 : v1.MixColumn(rd, rs1, fwd=0)
\end{lstlisting}
\caption{
  Instruction mnemonics, and their mapping onto pseudo-code functions, for \ISE{1}.
}
\label{fig:v1:mnemonics}
\end{figure}

\begin{figure}[!h]
\begin{lstlisting}[language=pseudo,style=block]
v1.SubByte(rd, rs1, fwd):
    rd.8[i] = AESSBox[rs1.8[i]] if fwd else AESInbSBox[rs1.8[i]] for i=0..3

v1.MixColumn(rd, rs1, fwd)
    for i=0..3:
        tmp.32  = ROTL32(rs1.32, 8*i)
        rd.8[i] = AESMixColumn(tmp.32) if fwd else AESInvMixColumn(tmp.32)
\end{lstlisting}
\caption{
  Instruction pseudo-code functions for \ISE{1}.
}
\label{fig:v1:pseudo}
\end{figure}

\begin{figure}[!h]
\begin{lstlisting}[language=pseudo,style=block]
lw           a0,  0(a4)       // Load Round Key
lw           a1,  4(a4)
lw           a2,  8(a4)
lw           a3, 12(a4)
xor          a4, a4, a0       // Add Round Key
xor          a5, a5, a1
xor          a6, a6, a2
xor          a7, a7, a3
saes.v1.encs a0, a4           // SubBytes
saes.v1.encs a1, a5
saes.v1.encs a2, a6
saes.v1.encs a3, a7
                              // Shift Rows
and          a4, t0, t6   ; and   a5, t1, t6
and          a6, t2, t6   ; and   a7, t3, t6
slli         t4, t6, 0x8  ; and   t5, t0, t4
or           a7, a7, t5   ; and   t5, t3, t4
or           a6, a6, t5   ; and   t5, t2, t4
or           a5, a5, t5   ; and   t5, t1, t4
or           a4, a4, t5   ; slli  t4, t4, 0x8
and          t5, t2, t4   ; or    a4, a4, t5
and          t5, t3, t4   ; or    a5, a5, t5
and          t5, t0, t4   ; or    a6, a6, t5
and          t5, t1, t4   ; or    a7, a7, t5
slli         t4, t4, 0x8  ; and   t5, t3, t4
or           a4, a4, t5   ; and   t5, t0, t4
or           a5, a5, t5   ; and   t5, t1, t4
or           a6, a6, t5   ; and   t5, t2, t4
or           a7, a7, t5
saes.v1.encm t0, a4           // MixColumns
saes.v1.encm t1, a5
saes.v1.encm t2, a6
saes.v1.encm t3, a7
\end{lstlisting}
\caption{
  An AES encryption round implemented using \ISE{1}.
}
\label{fig:v1:round}
\end{figure}

\vspace*{\fill}

% =============================================================================

\clearpage
\section{\ISE{2}:       additional technical detail}
\label{sec:pseudo:v2}
% =============================================================================

\vspace*{\fill}

\begin{figure}[!h]
\begin{lstlisting}[language=pseudo,style=block]
saes.v2.encs rd, rs1, rs2 : v2.SubBytes(rd, rs1, rs2, fwd=1)
saes.v2.decs rd, rs1, rs2 : v2.SubBytes(rd, rs1, rs2, fwd=0)
saes.v2.encm rd, rs1, rs2 : v2.MixColumns(rd, rs1, rs2, fwd=1)
saes.v2.decm rd, rs1, rs2 : v2.MixColumns(rd, rs1, rs2, fwd=0)
\end{lstlisting}
\caption{
  Instruction mnemonics, and their mapping onto pseudo-code functions, for \ISE{2}.
}
\label{fig:v2:mnemonics}
\end{figure}

\begin{figure}[!h]
\begin{lstlisting}[language=pseudo,style=block]
v2.SubBytes(rd, rs1, rs2, fwd):
  t1.32  = {rs1.8[0], rs2.8[1], rs1.8[2], rs2.8[3]}
  rd.8[i]= AESSBox[t1.8[i]] if fwd else AESInvSBox[t1.8[i]] for i=0..3

v2.MixColumns(rd, rs1, rs2, fwd):
  t1.32  = {rs1.8[0], rs1.8[1], rs2.8[2], rs2.8[3]}
  for i=0..3:
      tmp.32 = ROTL32(rs1.32, 8*i)
      rd.8[i]= AESMixColumn(tmp.32) if fwd else AESInvMixColumn(tmp.32)
\end{lstlisting}
\caption{
  Instruction pseudo-code functions for \ISE{2}.
}
\label{fig:v2:pseudo}
\end{figure}

\begin{figure}[!h]
\begin{lstlisting}[language=pseudo,style=block]
lw              a0,  0(a4)     // Load Round Key
lw              a1,  4(a4)
lw              a2,  8(a4)
lw              a3, 12(a4)
xor             t0, t0, a0     // Add Round Key
xor             t1, t1, a1
xor             t2, t2, a2
xor             t3, t3, a3
saes.v2.sub.enc a0, t0, t1     // SubBytes / ShiftRows
saes.v2.sub.enc a1, t2, t3
saes.v2.sub.enc a2, t1, t2
saes.v2.sub.enc a3, t3, t0
saes.v2.mix.enc t0, a0, a1     // ShiftRows / MixColumns
saes.v2.mix.enc t1, a2, a3
saes.v2.mix.enc t2, a1, a0
saes.v2.mix.enc t3, a3, a2
\end{lstlisting}
\caption{
  An AES encryption round implemented using \ISE{2}.
}
\label{fig:v2:round}
\end{figure}

\vspace*{\fill}

% -----------------------------------------------------------------------------

\newpage

\vspace*{\fill}

\begin{figure}[!h]
\centering
\includegraphics[width={0.5\textwidth}]{diagrams/ise-datapath-v2.png}
\caption{
  A diagramatic description of the functional unit required to support \ISE{2}.
}
\label{fig:v2:fu}
\end{figure}

\vspace*{\fill}

% =============================================================================

\clearpage
\section{\ISE{3}:       additional technical detail}
\label{sec:pseudo:v3}
% =============================================================================

\vspace*{\fill}

\begin{figure}[!h]
\begin{lstlisting}[language=pseudo,style=block]
saes.v3.encs  rd, rs1, rs2, bs : v3.Proc(rd, rs1, rs2, bs, fwd=1, mix=0)
saes.v3.encsm rd, rs1, rs2, bs : v3.Proc(rd, rs1, rs2, bs, fwd=1, mix=1)
saes.v3.decs  rd, rs1, rs2, bs : v3.Proc(rd, rs1, rs2, bs, fwd=0, mix=0)
saes.v3.decsm rd, rs1, rs2, bs : v3.Proc(rd, rs1, rs2, bs, fwd=0, mix=1)
\end{lstlisting}
\caption{
  Instruction mnemonics, and their mapping onto pseudo-code functions, for \ISE{3}.
}
\label{fig:v3:mnemonics}
\end{figure}

\begin{figure}[!h]
\begin{lstlisting}[language=pseudo,style=block]
v3.Proc(rd, rs1, rs2, bs, fwd, mix):
  x     = AESSBox[rs2.8[bs]] if fwd else AESInvSBox[rs2.8[bs]]
  if   mix and  fwd: t1.32 = {GFMUL(x, 3),      x    ,      x   ,GFMUL(x, 2)}
  elif mix and !fwd: t1.32 = {GFMUL(x,11),GFMUL(x,13),GFMUL(x,9),GFMUL(x,14)}
  else             : t1.32 = {0, 0, 0, x}
  rd.32 = ROTL32(t1.32, 8*bs) ^ rs1
\end{lstlisting}
\caption{
  Instruction pseudo-code functions for \ISE{3}.
}
\label{fig:v3:pseudo}
\end{figure}

\begin{figure}[!h]
\begin{lstlisting}[language=pseudo,style=block]
lw              a0, 16(RK)      // Load Round Key
lw              a1, 20(RK)
lw              a2, 24(RK)
lw              a3, 28(RK)      // t0,t1,t2,t3 contains current round state.
saes.v3.encsm   a0, a0, t0, 0   // Next state for column 0.
saes.v3.encsm   a0, a0, t1, 1   // Current column 0 in t0.
saes.v3.encsm   a0, a0, t2, 2   // Next column 0 accumulates in a0
saes.v3.encsm   a0, a0, t3, 3
saes.v3.encsm   a1, a1, t1, 0   // Next state for column 1.
saes.v3.encsm   a1, a1, t2, 1
saes.v3.encsm   a1, a1, t3, 2
saes.v3.encsm   a1, a1, t0, 3
saes.v3.encsm   a2, a2, t2, 0   // Next state for column 2.
saes.v3.encsm   a2, a2, t3, 1
saes.v3.encsm   a2, a2, t0, 2
saes.v3.encsm   a2, a2, t1, 3
saes.v3.encsm   a3, a3, t3, 0   // Next state for column 3.
saes.v3.encsm   a3, a3, t0, 1
saes.v3.encsm   a3, a3, t1, 2
saes.v3.encsm   a3, a3, t2, 3   // a0,a1,a2,a3 contains new round state
\end{lstlisting}
\caption{
  An AES encryption round implemented using \ISE{3}.
}
\label{fig:v3:round}
\end{figure}

\vspace*{\fill}

% -----------------------------------------------------------------------------

\newpage

\vspace*{\fill}

\begin{figure}[!h]
\centering
\includegraphics[width={0.5\textwidth}]{diagrams/ise-datapath-v3.png}
\caption{
  A diagramatic description of the functional unit required to support \ISE{3}.
}
\label{fig:v3:fu}
\end{figure}

\vspace*{\fill}

% =============================================================================

\clearpage
\section{\ISE{4}:       additional technical detail}
\label{sec:pseudo:v4}
\input{tex/appx-design_v4.tex}
\clearpage
\section{\ISE{5}:       additional technical detail}
\label{sec:pseudo:v5}
% =============================================================================

\vspace*{\fill}

\begin{figure}[!h]
\begin{lstlisting}[language=pseudo,style=block]
saes.v5.esrsub.lo rd, rs1, rs2 : rd = v5.SrSub(rs1, rs2, fwd=1, hi=0)
saes.v5.esrsub.hi rd, rs1, rs2 : rd = v5.SrSub(rs1, rs2, fwd=1, hi=1)
saes.v5.dsrsub.lo rd, rs1, rs2 : rd = v5.SrSub(rs1, rs2, fwd=0, hi=0)
saes.v5.dsrsub.hi rd, rs1, rs2 : rd = v5.SrSub(rs1, rs2, fwd=0, hi=1)
saes.v5.emix      rd, rs1, rs2 : rd = v5.Mix(rs1, rs2, fwd=1)
saes.v5.dmix      rd, rs1, rs2 : rd = v5.Mix(rs1, rs2, fwd=0)
saes.v5.sub       rd, rs1      : rd = SubBytes(rs1.8[i])         for i=0..3
\end{lstlisting}
\caption{
  Instruction mnemonics, and their mapping onto pseudo-code functions, for \ISE{5}.
}
\label{fig:v5:mnemonics}
\end{figure}

\begin{figure}[!h]
\begin{lstlisting}[language=pseudo,style=block]
v5.SrSub(rd, rs1, rs2, fwd, hi):
  if(fwd):
    if hi: tmp.32 = {rs1.8[3], rs2.8[0], rs2.8[1], rs2.8[2]}
    else : tmp.32 = {rs2.8[3], rs1.8[1], rs1.8[0], rs1.8[2]}
    tmp.8[i]      =    AESSBox[tmp.8[i]] for i=0..3
  else:
    if hi: tmp.32 = {rs2.8[3], rs2.8[0], rs1.8[1], rs2.8[2]}
    else : tmp.32 = {rs1.8[3], rs2.8[1], rs1.8[0], rs1.8[2]}
    tmp.8[i]      = InvAESSBox[tmp.8[i]] for i=0..3
  if(hi): rd.32 = {tmp.8[2],tmp.8[3],tmp.8[0],tmp.8[1]}
  else  : rd.32 = {tmp.8[1],tmp.8[3],tmp.8[0],tmp.8[2]}

v5.mix(rd, rs1, rs2, fwd):
  col0.32 = {rs1.8[2], rs1.8[3], rs2.8[2], rs2.8[3]}
  col1.32 = {rs1.8[0], rs1.8[1], rs2.8[0], rs2.8[1]}
  n0.8    = AESMixColumn(       col0   ) if fwd else AESInvMixColumn(       col0   )
  n1.8    = AESMixColumn(ROTL32(col0,8)) if fwd else AESInvMixColumn(ROTL32(col0,8))
  n2.8    = AESMixColumn(       col1   ) if fwd else AESInvMixColumn(       col1   )
  n3.8    = AESMixColumn(ROTL32(col1,8)) if fwd else AESInvMixColumn(ROTL32(col1,8))
  rd.32 = {n2, n3, n0, n1}
\end{lstlisting}
\caption{
  Instruction pseudo-code functions for \ISE{5}.
}
\label{fig:v5:pseudo}
\end{figure}

\begin{figure}[!h]
\begin{lstlisting}[language=pseudo,style=block]
lw                a0,  0(a4)   // Load Round Key
lw                a1,  4(a4)
lw                a2,  8(a4)
lw                a3, 12(a4)
xor               t0, t0, a0   // Add Round Key
xor               t1, t1, a1
xor               t2, t2, a2
xor               t3, t3, a3
saes.v5.esrsub.lo a0, t0, t1   // Quad 0: SubBytes / ShiftRows
saes.v5.esrsub.lo a1, t1, t0   // Quad 1
saes.v5.esrsub.hi a2, t2, t3   // Quad 2
saes.v5.esrsub.hi a3, t3, t2   // Quad 3
saes.v5.emix      t0, a0, a2   // Quad 0: ShiftRows / MixColumns
saes.v5.emix      t1, a1, a3   // Quad 1
saes.v5.emix      t2, a2, a0   // Quad 2
saes.v5.emix      t3, a3, a1   // Quad 3
\end{lstlisting}
\caption{
  An AES encryption round implemented using \ISE{5}.
}
\label{fig:v5:round}
\end{figure}

\vspace*{\fill}

% =============================================================================


\clearpage
\section{\mbox{\CORE{2}} core: additional technical detail}
\label{sec:core:2}
% =============================================================================

\vspace*{\fill}

\begin{figure}[!h]
\centering
\includegraphics[scale={0.45},angle={90}]{diagrams/scarv-cpu-uarch.png}
\caption{
  \CORE{2} core: vanilla  micro-architecture.
}
\label{fig:core:2:normal}
\end{figure}

\begin{figure}[!h]
\centering
\includegraphics[scale=0.45,angle=90]{diagrams/scarv-cpu-uarch-sca.png}
\caption{
  \CORE{2} core: hardened micro-architecture, 
  extending \ISE{3} for improved security against side-channel attack.
  Connections coloured red are security-critical, in the sense they relate to masks.
}
\label{fig:core:2:secure}
\end{figure}

\vspace*{\fill}

% =============================================================================

\clearpage
\section{\mbox{\CORE{1}} core: additional technical detail}
\label{sec:core:1}
% =============================================================================

\vspace*{\fill}

\begin{figure}[!h]
\begin{lstlisting}[style={block},language={scala}]
class AESVanilla32 extends Config (
  new freechips.rocketchip.subsystem.WithNoMMIOPort ++
  new freechips.rocketchip.subsystem.WithNoSlavePort ++
  new freechips.rocketchip.subsystem.WithInclusiveCache ++
  new freechips.rocketchip.subsystem.WithRV32 ++
  new freechips.rocketchip.subsystem.WithNExtTopInterrupts(0) ++
  new freechips.rocketchip.subsystem.WithNBigCores(1) ++
  new freechips.rocketchip.subsystem.WithoutFPU ++
  new freechips.rocketchip.system.BaseConfig
)
\end{lstlisting}
\caption{$32$-bit \CORE{1} core configuration.}
\label{fig:rocket:32}
\end{figure}

\begin{figure}[!h]
\begin{lstlisting}[style={block},language={scala}]
class AESVanilla64 extends Config(
  new freechips.rocketchip.subsystem.WithNoMMIOPort ++
  new freechips.rocketchip.subsystem.WithNoSlavePort ++
  new freechips.rocketchip.subsystem.WithInclusiveCache ++
  new freechips.rocketchip.subsystem.WithNExtTopInterrupts(0) ++
  new freechips.rocketchip.subsystem.WithNBigCores(1) ++
  new freechips.rocketchip.subsystem.WithoutFPU ++
  new freechips.rocketchip.system.BaseConfig
)
\end{lstlisting}
\caption{$64$-bit \CORE{1} core configuration.}
\label{fig:rocket:64}
\end{figure}

\vspace*{\fill}

% =============================================================================


%\clearpage
%\section{Additional algorithms}
%\label{sec:alg}
%% =============================================================================

\begin{algorithm}
\KwData  {A  cipher key             $k$,
          an initialisation vector $iv$,
          and
          an $n$-block  plaintext   $m$.
}
\KwResult{An $n$-block ciphertext   $c$.
}
\BlankLine
\KwFn{$\mbox{\SCOPE{\ID{AES-CBC}}{\ALG{Enc}}}( k, iv, m )$}{
    $c_0 \ASN \SCOPE{\ID{AES}}{\ALG{Enc}}( k, m_0   \XOR iv       )$ \;
  \For{$i = 0$ {\bf upto} $n-1$}{
    $c_i \ASN \SCOPE{\ID{AES}}{\ALG{Enc}}( k, m_i   \XOR  c_{i-1} )$ \;
  }
  \KwRet{$c$} \;
}
\caption{AES-CBC~\cite[Section 6.2]{NIST:sp.800.38a} encryption.}
\label{alg:cbc:enc}
\end{algorithm}

\begin{algorithm}
\KwData  {A  cipher key             $k$,
          an initialisation vector $iv$,
          and
          an $n$-block ciphertext   $c$.
}
\KwResult{An $n$-block  plaintext   $m$.
}
\BlankLine
\KwFn{$\mbox{\SCOPE{\ID{AES-CBC}}{\ALG{Dec}}}( k, iv, c )$}{
    $m_0 \ASN \SCOPE{\ID{AES}}{\ALG{Dec}}( k, c_0 ) \XOR iv        $ \;
  \For{$i = 0$ {\bf upto} $n-1$}{
    $m_i \ASN \SCOPE{\ID{AES}}{\ALG{Dec}}( k, c_i ) \XOR  c_{i-1}  $ \;
  }
  \KwRet{$m$} \;
}
\caption{AES-CBC~\cite[Section 6.2]{NIST:sp.800.38a} decryption.}
\label{alg:cbc:dec}
\end{algorithm}

% =============================================================================

\begin{algorithm}
\KwData  {A  cipher key             $k$,
          an initialisation vector $iv$,
          an increment function     $f$,
          and
          an $n$-block  plaintext   $m$.
}
\KwResult{An $n$-block ciphertext   $c$.
}
\BlankLine
\KwFn{$\mbox{\SCOPE{\ID{AES-CTR}}{\ALG{Enc}}}( k, iv, m )$}{
  $t_0 \ASN iv$ \;
  \For{$i = 0$ {\bf upto} $n-1$}{
    $t_{i+1} \ASN f( t_{i} )$ \;
    $c_{i  } \ASN \SCOPE{\ID{AES}}{\ALG{Enc}}( k, t_{i+1} ) \XOR m_{i}$ \;
  }
  \KwRet{$c$} \;
}
\caption{AES-CTR~\cite[Section 6.5]{NIST:sp.800.38a} encryption.}
\label{alg:ctr:enc}
\end{algorithm}

\begin{algorithm}
\KwData  {A  cipher key             $k$,
          an initialisation vector $iv$,
          an increment function     $f$,
          and
          an $n$-block ciphertext   $c$.
}
\KwResult{An $n$-block  plaintext   $m$.
}
\BlankLine
\KwFn{$\mbox{\SCOPE{\ID{AES-CTR}}{\ALG{Dec}}}( k, iv, m )$}{
  $t_0 \ASN iv$ \;
  \For{$i = 0$ {\bf upto} $n-1$}{
    $t_{i+1} \ASN f( t_{i} )$ \;
    $m_{i  } \ASN \SCOPE{\ID{AES}}{\ALG{Enc}}( k, t_{i+1} ) \XOR c_{i}$ \;
  }
  \KwRet{$m$} \;
}
\caption{AES-CTR~\cite[Section 6.5]{NIST:sp.800.38a} decryption.}
\label{alg:ctr:dec}
\end{algorithm}

% =============================================================================

\begin{algorithm}
\KwData  {A  cipher key             $k$,
          an initialisation vector $iv$,
          an increment function     $f$,
          and
          an $n$-block sequence     $x$.
}
\KwResult{An $n$-block sequence     $y$.
}
\BlankLine
\KwFn{$\mbox{\SCOPE{\ID{GCM}}{\ALG{GCTR}}}( k, iv, f, x )$}{
  $t_0 \ASN iv$ \;
  \For{$i = 0$ {\bf upto} $n-1$}{
    $t_{i+1} \ASN f( t_{i} )$ \;
    $y_{i  } \ASN \SCOPE{\ID{AES}}{\ALG{Enc}}( k, t_{i+1} ) \oplus x_i$ \;
  }
  \KwRet{$y$} \;
}
\caption{The GCTR  component of AES-GCM~\cite[Algorithm 3]{NIST:sp.800.38d}.}
\label{alg:gctr}
\end{algorithm}

% -----------------------------------------------------------------------------

\begin{algorithm}
\KwData  {A  hash   key             $h$,
          and
          an $n$-block sequence     $x$.
}
\KwResult{A         tag             $t$.
}
\BlankLine
\KwFn{$\mbox{\SCOPE{\ID{GCM}}{\ALG{GHASH}}}( h, x )$}{
  $t \ASN 0$ \;
  \For{$i = 0$ {\bf upto} $n-1$}{
    $t \ASN ( t \oplus x_{i} ) \otimes h$
  }
  \KwRet{$t$}
}
\caption{The GHASH component of AES-GCM~\cite[Algorithm 2]{NIST:sp.800.38d}.}
\label{alg:ghash}
\end{algorithm}

% -----------------------------------------------------------------------------

\begin{algorithm}
\KwData  {A  cipher key       $k$,
          a   plaintext block $m$,
          a       tweak block $i$,
          and
          a       block index $j$.
}
\KwResult{A  ciphertext block $c$.
}
\BlankLine
\KwFn{$\mbox{\SCOPE{\ID{XTS-AES}}{\ALG{Enc}}}( k, m, i, j )$}{
  parse $k = k_1 \CONS k_2$ \;
  $t  \ASN \SCOPE{\ID{AES}}{\ALG{Enc}}( k_2, i  ) \otimes \alpha^k$ \;
  $m' \ASN m  \oplus t                                            $ \;
  $c' \ASN \SCOPE{\ID{AES}}{\ALG{Enc}}( k_1, m' )                 $ \;
  $c  \ASN c' \oplus t                                            $ \;
  \KwRet{$c$} \;
}
\caption{XTS-AES~\cite{NIST:sp.800.38e} encryption.}
\label{alg:xts:enc}
\end{algorithm}

\begin{algorithm}
\KwData  {A  cipher key       $k$,
          a  ciphertext block $c$,
          a       tweak block $i$,
          and
          a       block index $j$.
}
\KwResult{A   plaintext block $m$.
}
\BlankLine
\KwFn{$\mbox{\SCOPE{\ID{XTS-AES}}{\ALG{Dec}}}( k, c, i, j )$}{
  parse $k = k_1 \CONS k_2$ \;
  $t  \ASN \SCOPE{\ID{AES}}{\ALG{Enc}}( k_2, i  ) \otimes \alpha^k$ \;
  $c' \ASN c  \oplus t                                            $ \;
  $m' \ASN \SCOPE{\ID{AES}}{\ALG{Enc}}( k_1, c' )                 $ \;
  $m  \ASN m' \oplus t                                            $ \;
  \KwRet{$m$} \;
}
\caption{XTS-AES~\cite{NIST:sp.800.38e} decryption.}
\label{alg:xts:dec}
\end{algorithm}

% =============================================================================


% =============================================================================

\end{document}
